Where the project laid out in \S \ref{sec:p2} will study the feasibility of
using the Jao Gap to date stellar populations; this project will apply this
technique to observational data. We will use Gaia photometry for the solar
neighborhood to study whether the Jao Gap's theoretical age-dependence is
observable.

Our sample will be pulled from the GCNS. It is known that the Jao Gap is
observable in this sample \citep[Figure 17 in][]{GaiaCollaboration2021};
however, we are less interested in if the gap shows up at all and instead
interested in if subsets of the population show the gap in different locations.
Using Gaia reported vertical velocity displacments and follwing methods from
\citep{Lu2021}  we will slice the GCNS into rough age groups (e.g. $< 2$ Gyr, $\geq
2$ and $< 6$ Gyr, and $\geq 6$ Gyr). Then, following binning methods laied
out in \citet{Jao2018} we will measure the Jao Gap location for each group.

Both binarity and variations in metallicity will smear the Jao Gap. The
Renormalised Unit Weight Error (RUWE) reported by Gaia, which tracks excess
astrometric noise, can be used to filter for wide binaries with mass ratios
near 1 \citep[e.g.][]{Deacon2020, Kervella2022}. Recent work has shown that the
RUWE is biased in the GCNS and that a further renormalization of the unit
weight error is needed \citep{Penoyre2022}. Therefore, in this work we will
adopt the Local Unit Weight Error (LUWE) from \citet{Penoyre2022} to clean our
sample of binaries. Additionally, we can model populations assuming a local M
Dwarf binary fraction \citep{Winters2019} to place bounds on gap and gap shift
detectability. Similarly we will adopt a metallicity function for the solar
neighborhood (though this is approximatly constant) \citep{Holmberg2009} when
modeling the gap to place bounds on it and its shift detectability.
