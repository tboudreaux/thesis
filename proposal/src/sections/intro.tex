\section{INTRODUCTION}
Over the last half of the 19th and first decade of the 20th centuries Lane,
Ritter, and Emden codified the earliest mathematical model of stellar
structure, the polytrope (Equation \ref{eqn:polytrope}), in \textit{Gaskugeln}
(Gas Balls) \citep{Emden1907}.

\begin{align}\label{eqn:polytrope}
	\frac{d}{d\xi}\left(\xi^{2}\frac{d\theta}{d\xi}\right) = -\xi^{2}\theta^{n}
\end{align}

Where $\xi$ and $\theta$ are dimensionless parameterizations of radius and
temperature respectively, and $n$ is known as the polytropic index. Despite this
early work, it wasn't until the late 1930s and early 1940s that the full set of
equations needed to describe the structure of a steady state,
radially-symmetric, star (known as the equations of stellar structure) began to
take shape as proton-proton chains and the Carbon-Nitrogen-Oxygen cycle were,
for the first time, seriously considered as energy generation mechanisms
\citep{Cowling1966}. Since then, and especially with the proliferation of
computers in astronomy, the equations of stellar structure have proven
themselves an incredibly predictive set of models.  

Low mass stars form an important component of the stellar population, with stars
less than [MASS HERE] making up more than 70\% of stars in the galaxy. 

There are currently many stellar structure codes \citep[e.g.][]{Dotter2008,
Kovetz2009, Paxton2011} which integrate the equations of stellar structure ---
in addition to equations of state and lattices of nuclear reaction rates ---
over time to track the evolution of an individual star. The Dartmouth Stellar
Evolution Program (DSEP) \citep{Chaboyer2001, Bjork2006, Dotter2008} is one
such, well tested, stellar evolution program.

DSEP solves the equations of stellar structure using the Henyey method
\citep{Henyey1964}. This is a relaxation technique making use of a
Newton–Raphson root finder and therefore requires some initial guess to relax
towards a solution. This guess will be either some initial, polytropic, model
or the solution from the previous timestep.  In order to evolve a model through
time DSEP alternates between solving for reaction rates and the structure
equations. At some temperature and pressure from the solution to the structure
equations DSEP finds the energy generation rate due to proton-proton chains,
the CNO cycle, and the tripe-alpha process from known nuclear cross sections.
These reaction rates yield both photon and neutrino luminosities as well as
chemical changes over some small time step. Thermodynamic variables are
calculated using an equation of state routine which is dependent on the initial
model mass. All the updated physical quantities (pressure, luminosity, mean
molecular mass, temperature) are then used to solve the structure equations
again. This process of using a solution to the structure equations to calculate
reaction rates which then inform the next structure solution continues until
DSEP can no longer find a solution.  This can happen as the stellar structure
equations are extremely stiff. In addition, for finite radial mesh sizes,
discontinuities can occur.

While other stellar evolution programs, such as the widely used Modules for
Experimentation in Stellar Astrophysics (MESA) \citep{Paxton2011}, consider a
more complex handling of nuclear reaction rate calculations, and are
consequently more applicable to a wider range of spectral classes than DSEP,
DSEP has certain advantages over these other programs that make it well suited
for certain tasks, such as low-mass modeling. For one, DSEP generally can
evolve models much more rapidly than MESA and has a smaller memory footprint
while doing it. This execution time difference is largely due to the fact that
DSEP makes some simplifying assumptions due to its focus only on models with
initial masses between 0.1 and 5 M$_{\odot}$ compared to MESA’s more general
approach.  Moreover, MESA elects to take a very careful handling of numeric
uncertainty, going so far as to guarantee byte-to-byte similarity of the same
model run on different architectures \citep{Paxton2011}. DSEP on the other hand
makes no such guarantee. Rather, models evolved using DSEP will be accurate
down to some arbitrary, user controllable, tolerance but beyond that point may
vary from one computer to another. Despite this trade off in generality and
precision, the current grid of isochrones generated by DSEP \citep{Dotter2008},
has been heavily cited since its initial release in 2008, proving that there is
a place for a code as specific as DSEP.

As DSEP pushes a star along its evolutionary track the radiative opacity must
be known for a wide range of temperatures, pressures, and compositions.
Specifically, opacity is a key parameter in the equation of energy transport.
With current computational tools it's infeasible to compute opacities on the
fly; rather, Rossland Mean opacity ($\kappa_{R}$) for individual elements must
be pre-tabulated over a wide range of temperatures and densities. These
opacities can then be somewhat arbitrarily mixed together and interpolated to
form opacity lookup-tables. Multiple groups have preformed these calculations
and subsequently made tables available to the wider community, these include
the Opacity Project \citep[OP][]{Seaton1994}, Laurence Livermore National Labs
OPAL opacity tables \citep{Iglesias1996}, and Los Alamos National Labs OPLIB
opacity tables \citep{Colgan2016}.

The OPAL opacity tables in particular are very widely used by current
generation stellar evolution programs (in addition to current generation
stellar model and isochrone grids). However, they are no longer the most up
date elemental opacities. Moreover, the generation mechanism for these tables,
a webform, is no longer reliably online.  Consequently, it makes sense to
transition to more modern opacity tables with a more stable generation
mechanism.

Here we will present work transitioning DSEP from OPAL opacities to opacities
based on measurements from Los Alamos national Labs T-1 group
\citep[OPLIB][]{Colgan2016}. Moreover, we will present two projects which are
in large part reliant on these updated opacities. For the first project we
investigate the affects of chemically self consistent modeling of multiple
populations within the globular cluster NGC 2808, and for the second project we
present the effects of the OPLIB opacities on the location of the recently
discovered Gaia M-dwarf gap.

This paper is organized as follows. In Section \ref{sec:opac} we outline some
basic information about OPLIB opacities, how we query them, and how we modify
them to work with DSEP. In Section \ref{sec:2808} we discuss scientific
background of the first project along with the current work done towards its
goal. Finally, in Section \ref{sec:Jao} we present our findings on the effects
of OPLIB opacities on the location of the Gaia M-dwarf gap.





