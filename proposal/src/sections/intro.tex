\section{INTRODUCTION}
Throughout last half of the 19th and first decade of the 20th centuries Lane,
Ritter, and Emden codified the earliest mathematical model of stellar
structure, the polytrope (Equation \ref{eqn:polytrope}), in \textit{Gaskugeln}
(Gas Balls) \citep{Emden1907}.

\begin{align}\label{eqn:polytrope}
	\frac{d}{d\xi}\left(\xi^{2}\frac{d\theta}{d\xi}\right) = -\xi^{2}\theta^{n}
\end{align}

Where $\xi$ and $\theta$ are dimensionless parameterizations of radius and
temperature respectively, and $n$ is known as the polytropic index. Despite this
early work, it wasn't until the late 1930s and early 1940s that the full set of
equations needed to describe the structure of a steady state,
radially-symmetric, star (known as the equations of stellar structure) began to
take shape as proton-proton chains and the Carbon-Nitrogen-Oxygen cycle were,
for the first time, seriously considered as energy generation mechanisms
\citep{Cowling1966}. Since then, and especially with the proliferation of
computers in astronomy, the equations of stellar structure have proven
themselves an incredibly predictive set of models.  

There are currently many stellar structure codes \citep[e.g.][]{Dotter2008,
Kovetz2009, Paxton2011} which integrate the equations of stellar structure ---
in addition to equations of state and lattices of nuclear reaction rates ---
over time to track the evolution of an individual star. The Dartmouth Stellar
Evolution Program (DSEP) \citep{Chaboyer2001, Bjork2006, Dotter2008} is one
such, well tested, stellar evolution program.

Here we propose to model low-mass stars in both the local solar neighborhood
and in globular clusters using DSEP. This work will primarily extend our
understanding of stellar physics in two areas: the effects of chemical
self-consistency on stellar models \citep[e.g.][]{Dotter2014} and the time
evolution of the core-convective instabilities which ultimately are believed to
result in the observed paucity of stars at a Gaia G magnitude of $\sim$10
\citep{Jao2018, Feiden2020}. 

% Low mass stars form an important component of the stellar population, with
% stars less than [MASS HERE] making up more than 70\% of stars in the galaxy
% [CITE]. In globular clusters, where all stars are coeval to one of a limited
% number of old populations, low mass stars provide the vast majority of data points
% when fitting ischrones [CITE]. Additionally, stars around the fully-convective
% transition mass show age-dependent core-convective instabilities [CITE].

\subsection{Globular Clusters}
Globular clusters \citep[GC,][]{Herschel1814} are among the oldest groupings of
stars in the Universe, with typical ages greater than 10 Gyr. They are
characterized by their compact size --- typical half-light radius $<$ 10 pc but
up to 10s of pc --- and high surface brightness --- M$_{V} \sim -7$. For
decades, prevailing thought had it globular clusters were composed of a single
stellar population born from a pristine interstellar media. Single stellar
populations had been assumed --- as opposed to multiple stellar population
(MPs) due to spectroscopically uniform iron abundances \citep{Gratton2012} and
very narrow principal sequences \citep{Stetson1988}, both of which are
indicative of a single stellar population.  

These early studies either did not handle or had very large photometric
uncertainties masking subtly distinguished features within the CMD. Moreover,
given these studies were ground based they were limited to optical bands where
colors do not respond strongly to all chemical changes within a star's
atmosphere.

\begin{figure}
	\centering
	\includegraphics[width=0.75\textwidth]{src/Figures/Gould53.png}
	\caption{$m_{pg}$ - $m_{pv}$ color-magnitude diagram for the globular cluster M3.}
	\label{fig:M3CMD}
\end{figure}

Despite the canonical view of single populations composing GCs, there has been
spectroscopic evidence for chemical inhomogeneities in GCs since the early
1970s \citep[e.g.][]{Osborn1971} and by the late 1980s, as higher resolution
photometry became available, multiple clusters were known which exhibited
features in their CMDs consistent with either bimodal or multimodal stellar
populations \citep[e.g.][]{Norris1987}.

The first conclusive evidence for MPs came with Hubble Space Telescope (HST)
high precision crowded field photometry in which three distinct main sequences
in NGC 2808 were identified \citep{Piotto2007}. Since this discovery, split
main sequences have been found in nearly all Milky Way globular clusters
studied by HST \citep{anderson2009,milone2011}. Split stellar populations are
believed to be due to enhanced helium abundances in the stellar populations
formed after the primordial population of stars \citep{d2005,Piotto2007}. When
compared to primordial helium mass fractions ($Y$) of $Y\sim 0.25$
\citep{collaboration2016planck} or solar helium abundances $Y\sim0.27$
\citep{vinyoles2017new} these populations have mass fractions as high as $Y\sim
0.4$. Helium enhancement is strongly suspected to be the result of an earlier,
more massive population dying off, enriching the interstellar medium
\citep{Gratton2001, Gratton2004, Gratton2012}; however, precise formation channels
for split stellar populations remain contentious. The primary open question then
is not why some populations are enhanced in helium; rather, it is to what
extent they are enhanced. 

Two chapters of this thesis will further constrain the helium enhancement of
MPs within globular clusters by modeling their stellar populations in a fully
chemically self-consistent manner. Sections \ref{sec:ngc2808} and
\ref{sec:7tucngc6752} address the details of these projects in more detail.

\subsection{Local Solar Neighborhood}
\citet{Jao2018} discovered a novel feature in the Gaia Gp-Rp color-magnitude
diagram. Around $M_{G}=10$ there is an approximatly 17\% decreas in stellar
density of the volume complete sample of stars \citeauthor{Jao2018} considered.
Subsequently, this has become known as either the Jao Gap, or Gaia M dwarf Gap.
Section \ref{} will go into more detail regarding the physics belived to
underpin this feature; however, in brief convective instabilities in the core
are belived to form for stars straddeling the fully convective transition mass.
These instabilities result in stars preferentially falling to either side of
the gap location.

Stellar modeling has been sucsessful in reproducing the Jap Gap and, with these
models, we have begun to constrain parameters which constrain gap location. For
example, it is now well documented that a stars metallicity can affect the gap
color by up to [HOW MUCH DID GREG FIND/CHECK FOR OTHER PAPERS ON THIS]. 

Initial testting which we have done using DSEP along with work by [PAPER] also
indicated the Jao Gap's color sensitivity to age. We observe that as models age
the Jao Gap moves [DIRECTION OF MOVMENT IN MAG AND COLOR SPACE]. Sections \ref{}
and \ref{} of this proposal lay out a plan to use this observed age-dependence
to date low-mass stars in the local-solar neighborhood.

\subsection{OPAL (move?)}
The OPAL opacity tables in particular are very widely used by current
generation stellar evolution programs (in addition to current generation
stellar model and isochrone grids). However, they are no longer the most up
date elemental opacities. Moreover, the generation mechanism for these tables,
a webform, is no longer reliably online.  Consequently, it makes sense to
transition to more modern opacity tables with a more stable generation
mechanism.

Here we will present work transitioning DSEP from OPAL opacities to opacities
based on measurements from Los Alamos national Labs T-1 group
\citep[OPLIB][]{Colgan2016}. Moreover, we will present two projects which are
in large part reliant on these updated opacities. For the first project we
investigate the affects of chemically self consistent modeling of multiple
populations within the globular cluster NGC 2808, and for the second project we
present the effects of the OPLIB opacities on the location of the recently
discovered Gaia M-dwarf gap.

This paper is organized as follows. In Section \ref{sec:opac} we outline some
basic information about OPLIB opacities, how we query them, and how we modify
them to work with DSEP. In Section \ref{sec:2808} we discuss scientific
background of the first project along with the current work done towards its
goal. Finally, in Section \ref{sec:Jao} we present our findings on the effects
of OPLIB opacities on the location of the Gaia M-dwarf gap.





