\section{DSEP}
DSEP solves the equations of stellar structure using the Henyey method
\citep{Henyey1964}. This is a relaxation technique making use of a
Newton–Raphson root finder and therefore requires some initial guess to relax
towards a solution. This guess will be either some initial, polytropic, model
or the solution from the previous timestep.  In order to evolve a model through
time DSEP alternates between solving for reaction rates and the structure
equations. At a given temperature and pressure from the solution to the structure
equations DSEP finds the energy generation rate due to proton-proton chains,
the CNO cycle, and the tripe-alpha process from known nuclear cross sections.
These reaction rates yield both photon and neutrino luminosities as well as
chemical changes over some small time step. Thermodynamic variables are
calculated using an equation of state routine which is dependent on the initial
model mass. All the updated physical quantities (pressure, luminosity, mean
molecular mass, temperature) are then used to solve the structure equations
again. This process of using a solution to the structure equations to calculate
reaction rates which then inform the next structure solution continues until
DSEP can no longer find a solution.  This can happen as the stellar structure
equations are extremely stiff. In addition, for finite radial mesh sizes,
discontinuities can occur.
