\section{DSEP}
DSEP solves the equations of stellar structure using the Henyey method
\citep{Henyey1964}. This is a relaxation technique making use of a
Newton–Raphson root finder and therefore requires some initial guess to relax
towards a solution. This guess will be either some initial, polytropic, model
or the solution from the previous timestep.  In order to evolve a model through
time DSEP alternates between solving for reaction rates and the structure
equations. At some temperature and pressure from the solution to the structure
equations DSEP finds the energy generation rate due to proton-proton chains,
the CNO cycle, and the tripe-alpha process from known nuclear cross sections.
These reaction rates yield both photon and neutrino luminosities as well as
chemical changes over some small time step. Thermodynamic variables are
calculated using an equation of state routine which is dependent on the initial
model mass. All the updated physical quantities (pressure, luminosity, mean
molecular mass, temperature) are then used to solve the structure equations
again. This process of using a solution to the structure equations to calculate
reaction rates which then inform the next structure solution continues until
DSEP can no longer find a solution.  This can happen as the stellar structure
equations are extremely stiff. In addition, for finite radial mesh sizes,
discontinuities can occur.

As DSEP pushes a star along its evolutionary track the radiative opacity must
be known for a wide range of temperatures, pressures, and compositions.
Specifically, opacity is a key parameter in the equation of energy transport.
With current computational tools it's infeasible to compute opacities on the
fly; rather, Rossland Mean opacity ($\kappa_{R}$) for individual elements must
be pre-tabulated over a wide range of temperatures and densities. These
opacities can then be somewhat arbitrarily mixed together and interpolated to
form opacity lookup-tables. Multiple groups have preformed these calculations
and subsequently made tables available to the wider community, these include
the Opacity Project \citep[OP][]{Seaton1994}, Laurence Livermore National Labs
OPAL opacity tables \citep{Iglesias1996}, and Los Alamos National Labs OPLIB
opacity tables \citep{Colgan2016}.
