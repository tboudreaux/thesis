In addition to NGC 2808, collaborators will generate \texttt{MARCS} atmospheric
boundary conditions for multiple populations within the clusters NGC 6752 and
47 Tuc. Using these surface boundary conditions along with new opacity tables
we query from OPLIB, we will conduct the same, self-consistent, modeling for
these clusters as we do for NGC 2808.

NGC 6752 has been self-consistently modeled in the past. \citet{Dotter2015} perform
self-consistent chemical modeling of this cluster using both \texttt{ATLAS} and
\texttt{PHOENIX} atmospheric boundary conditions computed from abundance of the
photometrically identified poulations A and C reported by \citet{Milone2013}.
\citeauthor{Dotter2015} additionally make use of OPAL high temperature opacity
tables and \texttt{PHOENIX} low temperature opacity tables.  

\citeauthor{Dotter2015} find slight difference between best fit ishcohrone to
NGC 6752 populations A and E for both blue and UV synthetic CMDs, with
\texttt{ATLAS} atmospheres preforming slightly better. Moreover, they find that
the inferred helium abundacnce is sensative to atmospheric boundary conditions,
with \texttt{ATLAS} models showing stronger heliumn sensitivity than
\texttt{PHOENIX} models.

We will replicate portions of the work \citeauthor{Dotter2015} conduct on NGC
6752 and extend this work to 47 Tuc. Of particular interest is \texttt{MARCS}
historical focus on earlier spectral class stars and commensurately on bluer
portions of the model atmosphere when compared to \texttt{PHOENIX}
\citep{Plez2011}, and how this will affect infered elemental abundacnces.
\citeauthor{Dotter2015} find the primary difference betweeen \texttt{ATLAS} and
\texttt{PHOENIX} models originating from UV difference in their spectra.
