Whereas, people have have often tried to categorized objects as GCs by making
cuts along half-light radius, density, and surface brightness profile, in fact
many objects which are generally thought of as GCs don't cleanly fit into these
cuts. Consequently, \citet{Carretta2010} proposed a definition of GC based on
observed chemical inhomogeneities in their stellar populations. The modern
understanding of GCs then is not simply one of a dense cluster of stars which
may have chemical inhomogeneities and multiple populations; rather, it is one
where those chemical inhomogeneities and multiple populations themselves are
the defining element of a GC.

All globular clusters older than 2 Gyr studied in detail show populations
enriched in He, N, and Na while also being deplete in O and C
\citep{Piotto2015,Bastian2018}. These light element abundance patterns also are
not strongly correlated with variations in heavy element abundances. One
consequence of this fact is the spectroscopically uniform Fe abundances
mentioned in \S\ref{sec:intro_GC}. Further, high-resolution spectral studies
reveal anti-correlations between N-C abundances, Na-O abundances, and
potentially Al-Mg \citep{Sneden1992, Gratton2012}. Typical stellar fusion
reactions can deplete core oxygen; however, the observed abundances of Na, Al,
and Mg cannot be explained by the likes of the CNO cycle \citep{Prantzos2007}.

Formation channels for these multiple populations remain a point of debate
among astronomers. Most proposed formation channels consist of some older,
more massive, population of stars polluting the pristine cluter media before a
second population forms, now enriched in heavier elements which they themselves could
not have generated \citep[for a detailed review see ][]{Gratton2012}. The four
primary candidates for these polluters are asymptotic giant branch stars
\citep[AGBs,][]{Ventura2001,DErcole2010}, fast rotating massive stars
\citep[FRMSs,][]{Decressin2007}, super massive stars
\citep[SMSs,][]{Denissenkov2014}, and massive interacing binaries
\citep[MIBs,][]{deMink2009, Bastian2018}. 

Hot hydrogen burning (proton capture), material transport to the surface, and
material ejection into the intra-cluster media are features of each of these
models and consequently they can all be made to {\it qualitatively} agree with
the observed elemental abundances. However, none of the standard models can
currently account for all specific abundances \citep{Gratton2012}. AGB and FRMS
models are the most promising; however, both models have difficulty reproducing
severe O depletion \citep{Ventura2009,Decressin2007}. Moreover, AGB and FRMS
models require signifigant mass loss ($\sim 90\%$) between cluster formation
and the current epoch --- implying that a signifigant fraction of halo stars
formed in GCs \citep{Renzini2008,DErcole2008,Bastian2015}.

In addition to the light-element anti-correlations observed it is also known
that younger populations are signifigantly enhanced in Helium
\citep{Piotto2007, Piotto2015, Latour2019}. Depending on the cluster, Helium
mass fractions as high as $Y=0.4$ have been inferred \citep[e.g][]{Milone2015}.
However, due to the relatively high and tight temperature range of partial
ionization for He it cannot be observed in globular clusters; consequently, the
evidence for enhanced He in GCs originates from comparison of theoretical
stellar isochrones to the observed color-magnitude-diagrams of globular
clusters. Therefore, a careful handling of chemistry is essential when modeling
with the aim of discriminating between MPs; yet, only a very limited number of
GCs have yet been studied with chemically self-consistent (structure and
atmosphere) isochrones \citep[e.g.][NGC 6752]{Dotter2015}.

This thesis will contain chapters where we expand the number of clusters which
have been self-consistently modeled. In this chapter we will focus on
chemically self-consistent modeling of the two extreme population of NGC 2808
identified by \citep{Milone2015}, A and E.

One key element of NGC 2808 modeling is the incorporation of new atmospheric
models, generated from the \texttt{MARCS} grid of model atmospheres \citep{Plez2008},
which match interior elemental abundances. \texttt{MARCS} provides one-dimensional,
hydrostatic, plane-parallel and spherical LTE atmopsheric models
\citep{Gustafsson2008}. Members of our collaboration have generated atmospheric 
models for populations A and E.  Integration of these new model atmospheres
into DSEP is ongoing. 

\subsubsection{Population Opacities}
For similar reasons as discussed in \S\ref{sec:p1} we conduct this research
with OPLIB high-temperature opacity tables as opposed to OPAL tables. We will
also generate low temperature opacity tables using the \texttt{MARCS}.
Moreover, we confirm that the atmosphere and structure meet in an optically
thick region of the star by shifting the atmospheric fitting point from an optical
depth of $\tau = 2/3$ (used by DSEP currently for \texttt{PHOENIX} model atmospheres) to
some higher $\tau$. We will experiment to identify the best optical depth to
fit at.. 

These population have been studied in depth by Feiden and their chemical
compositions were determined in \citet{Milone2015} (see Table 2 in that paper).
While we cannot yet evolve DSEP models with these new boundary conditions, we
can make a first pass investigation of the affect of OPLIB opacities (Figure
\ref{fig:NGC2808ISO}). Note how the models generated using OPLIB opacity tables
have a systematically lower luminosity. This discrepancy is consistent with
the overall lower opacities of the OPLIB tables. 

\begin{figure}
	\centering
	\includegraphics[width=0.75\textwidth]{src/Figures/033ZIsosOPALOPLIB.pdf}
	\caption{10 Gyr \& Y=0.33 isochrones for models generated with OPAL and
	OPLIB opacities tables (top). $\log(L)$ Residuals between isochrones (bottom).}
	\label{fig:NGC2808ISO}
\end{figure}

\subsubsection{Additional Consistency}
The isochrones generally used to infer the degree of helium enhancements assume that
convection operates in the same manner in metal-poor stars as it does in the
Sun. However, observations from \textit{Kepler} of metal-poor red giants
\citep{Bonaca2012, tayar2017correlation}, in concert with interferometric
radius determination of the metal-poor sub-giant HD 140283
\citep{creevey2015benchmark}, have shown that the efficiency of convection
changes with iron content. As the final portion of our work to more carefully
handle a star's chemistry, we will modify DSEP to capture this variation in
convective efficiency. 
