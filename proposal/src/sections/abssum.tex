Over its approximately 100 year history stellar modeling has become an
essential tool for understanding certain astrophysical phenomena which are not
directly observable. Modeling allows for empirical constraints --- such as
elemental abundances, luminosities, and effective temperatures --- to strongly
inform non-observables such a core temperature and pressure. Here we propose a
thesis in five parts, related through their use of both modeling and the
Dartmouth Stellar Evolution Program (DSEP) to conduct this modeling. In two of
the parts of this thesis we will use DSEP, in conjunction with atmospheric
boundary conditions generated by collaborators, to build chemically
self-consistent models of multiple populations (MPs) in the globular clusters
NGC 2808, 47 Tuc, and NGC 6752. We will infer helium abundances across MPs and
compare these inferred abundances to those from models which do not consider as
careful a handling of a star's chemistry. The remaining three parts of this
thesis will address a recently discovered feature in the Gaia $G_{BP} - G_{RP}$
color-magnitude-diagram (colloquially the Jao Gap). Throughout this series we
will update DSEP's high-temperature opacity tables to the most modern available
(OPLIB from Los Alamos) and show how this change affects the theoretical
location of the Jao Gap. Subsequently, we will use synthetic
color-magnitude-diagrams (CMDs) --- covering the Jao Gap regime --- in
conjunction with gyro-kinematically derived age distributions to test the
feasibility of population age-dating by measuring the Jao Gap's location in a
CMD. Finally, we will apply techniques developed in our theoretical testing of
Jao Gap based age-dating to the solar neighborhood, attempting to identify
coeval groups and roughly age-date them. These five parts will compose the
scientific chapters of a thesis to be submitted to the faculty and advising
committee no later than the summer term of 2024.
