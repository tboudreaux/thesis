%%
%% Beginning of file 'sample62.tex'
%%
%% Modified 2018 January
%%
%% This is a sample manuscript marked up using the
%% AASTeX v6.2 LaTeX 2e macros.
%%
%% AASTeX is now based on Alexey Vikhlinin's emulateapj.cls 
%% (Copyright 2000-2015).  See the classfile for details.

%% AASTeX requires revtex4-1.cls (http://publish.aps.org/revtex4/) and
%% other external packages (latexsym, graphicx, amssymb, longtable, and epsf).
%% All of these external packages should already be present in the modern TeX 
%% distributions.  If not they can also be obtained at www.ctan.org.

%% The first piece of markup in an AASTeX v6.x document is the \documentclass
%% command. LaTeX will ignore any data that comes before this command. The 
%% documentclass can take an optional argument to modify the output style.
%% The command below calls the preprint style  which will produce a tightly 
%% typeset, one-column, single-spaced document.  It is the default and thus
%% does not need to be explicitly stated.
%%
%%
%% using aastex version 6.2
\documentclass[twocolumn]{aastex62}

\usepackage{enumitem}
\usepackage{amsmath}

\newcommand{\vdag}{(v)^\dagger}
\newcommand\aastex{AAS\TeX}
\newcommand\latex{La\TeX}


\shorttitle{Updated High Temperature Opacities for DSEP}
\shortauthors{Boudreaux et al.}

\begin{document}

% \title{Self Consistent Models of He Enhancment in different populations of NGC 2808}
\title{Updated High-Temperature Opacities for the Dartmouth Stellar Evolution Program}

\author[0000-0002-2600-7513]{Thomas M. Boudreaux}
\affiliation{Department of Physics and Astronomy, Dartmouth College, Hanover, NH 03755, USA}

\author[0000-0003-3096-4161]{Brian C. Chaboyer}
\affiliation{Department of Physics and Astronomy, Dartmouth College, Hanover, NH 03755, USA}

\author[0000-0002-2012-7215]{Gregory A. Feiden}
\affiliation{Department of Physics and Astronomy, University of North Georgia, GA 30597, USA}



%% Mark off the abstract in the ``abstract'' environment. 
\begin{abstract}
	The equations of stellar structure have proven astonishingly predictive
	when describing stars interior structures.  In their most basic form they
	constitute 4 ordinary, first-order, differential equations. However, they
	are not on their own well enough constrained to solve. In addition to the
	four ODEs, an equation of state, thermal conductivities, nuclear reaction
	rates, and opacities are all required when modeling a star. Some of these
	additional constraints can be computed on the fly; however, as yet there is
	no effective way to compute opacities at run time. Rather, stellar
	structure programs use pre-tabulated opacities over a range of
	temperatures, densities, and chemical compositions. The Dartmouth Stellar
	Evolution Program (DSEP) has used OPAL opacities for the last decade and a
	half; however, there are now more up to date elemental opacity tables from
	OPLIB. Moreover, OPAL opacities can no longer be reliably generated for
	different chemical compositions.  Here we present an overview of how we
	update DSEP to use opacities from OPLIB in addition to preliminary results
	from two studies making use of these updated opacities.
	\vspace{1cm}
\end{abstract}


\keywords{}

%%%%%%%%%%%%%%%%%
%% INTRODUCTION
%%%%%%%%%%%%%%%%%
Example Content, 2


\acknowledgments{
	This research has made use of NASA's astrophysical data system (ADS). We
	acknowledge the support of an NASA grant (No. 80NSSC18K0634). Additionally,
	we would like to thank James Colgan for his assistance with the OPLIB
	opacity tables. We would like to thank Aaron Dotter, and Elisabeth
	Newton for their assistance. Finally, we thank our colleagues and peers in
	for their continuing and appreciated support.
}


%% If you wish to include an acknowledgments section in your paper,
%% separate it off from the body of the text using the \acknowledgments
%% command.
\acknowledgments

% \bibliography{src/Bibliography/ms}


\end{document}
