%%
%% Beginning of file 'sample62.tex'
%%
%% Modified 2018 January
%%
%% This is a sample manuscript marked up using the
%% AASTeX v6.2 LaTeX 2e macros.
%%
%% AASTeX is now based on Alexey Vikhlinin's emulateapj.cls 
%% (Copyright 2000-2015).  See the classfile for details.

%% AASTeX requires revtex4-1.cls (http://publish.aps.org/revtex4/) and
%% other external packages (latexsym, graphicx, amssymb, longtable, and epsf).
%% All of these external packages should already be present in the modern TeX 
%% distributions.  If not they can also be obtained at www.ctan.org.

%% The first piece of markup in an AASTeX v6.x document is the \documentclass
%% command. LaTeX will ignore any data that comes before this command. The 
%% documentclass can take an optional argument to modify the output style.
%% The command below calls the preprint style  which will produce a tightly 
%% typeset, one-column, single-spaced document.  It is the default and thus
%% does not need to be explicitly stated.
%%
%%
%% using aastex version 6.2
\documentclass[11pt, manuscript]{aastex62}

\usepackage{enumitem}
\usepackage{amsmath}

\marginparwidth 0pt
\oddsidemargin  0pt
\evensidemargin  0pt
\marginparsep 0pt
%\topmargin   -0.5in
\textwidth   7in
\textheight  9in %total height of text on page

\newcommand{\vdag}{(v)^\dagger}
\newcommand\aastex{AAS\TeX}
\newcommand\latex{La\TeX}


\shorttitle{Thesis Proposal}
\shortauthors{Boudreaux et al.}

\begin{document}

% \title{Self Consistent Models of He Enhancment in different populations of NGC 2808}
\begin{titlepage}
    \begin{center}
        \Large
        \vspace*{1cm}
		\textbf{Simulation go Burrr....} \\
        \vspace{0.75cm}
        \large
        A Thesis Proposal \\
        \vspace{0.25cm}
        Submitted to the Faculty \\
        % \\
        in partial fulfillment of the requirements for the \\
        % \\
        degree of \\
        \vspace{0.25cm}
        Doctor of Philosophy \\
        \vspace{0.25cm}
        in \\
        \vspace{0.25cm}
        Physics and Astronomy \\
        \vspace{0.5cm}
        by \\
        \vspace{0.25cm}
        Thomas M. Boudreaux \\
        \vspace{0.3cm}
        DARTMOUTH COLLEGE \\
        Hanover, NH \\
        May 10, 2022
     \end{center}
\vspace{1cm}
\begin{flushright}
The Examining Committee: \hspace{\parindent}
\vspace{0.4cm}
\end{flushright}
\begin{flushright}
\line(180,0){140}\\
        Dr. Brian Chaboyer
        \end{flushright}
        
        \begin{flushright}
        \line(180,0){140}\\
        Dr. Elisabeth E. Newton
        \end{flushright}
        
        \begin{flushright}
        \line(180,0){140}\\
        Dr. Aaron Dotter
        \end{flushright}

\end{titlepage}


%% Mark off the abstract in the ``abstract'' environment. 
\begin{abstract}
	The equations of stellar structure have proven astonishingly predictive
	when describing stars interior structures.  In their most basic form they
	constitute 4 ordinary, first-order, differential equations. However, they
	are not on their own well enough constrained to solve. In addition to the
	four ODEs, an equation of state, thermal conductivities, nuclear reaction
	rates, and opacities are all required when modeling a star. Some of these
	additional constraints can be computed on the fly; however, as yet there is
	no effective way to compute opacities at run time. Rather, stellar
	structure programs use pre-tabulated opacities over a range of
	temperatures, densities, and chemical compositions. The Dartmouth Stellar
	Evolution Program (DSEP) has used OPAL opacities for the last decade and a
	half; however, there are now more up to date elemental opacity tables from
	OPLIB. Moreover, OPAL opacities can no longer be reliably generated for
	different chemical compositions.  Here we present an overview of how we
	update DSEP to use opacities from OPLIB in addition to preliminary results
	from two studies making use of these updated opacities.
	\vspace{1cm}
\end{abstract}


\keywords{}

%%%%%%%%%%%%%%%%%
%% INTRODUCTION
%%%%%%%%%%%%%%%%%
\section{INTRODUCTION}
Over the last half of the 19th and first decade of the 20th centuries Lane,
Ritter, and Emden codified the earliest mathematical model of stellar
structure, the polytrope (Equation \ref{eqn:polytrope}), in \textit{Gaskugeln}
(Gas Balls) \citep{Emden1907}.

\begin{align}\label{eqn:polytrope}
	\frac{d}{d\xi}\left(\xi^{2}\frac{d\theta}{d\xi}\right) = -\xi^{2}\theta^{n}
\end{align}

Where $\xi$ and $\theta$ are dimensionless parameterizations of radius and
temperature respectively, and $n$ is known as the polytropic index. Despite this
early work, it wasn't until the late 1930s and early 1940s that the full set of
equations needed to describe the structure of a steady state,
radially-symmetric, star (known as the equations of stellar structure) began to
take shape as proton-proton chains and the Carbon-Nitrogen-Oxygen cycle were,
for the first time, seriously considered as energy generation mechanisms
\citep{Cowling1966}. Since then, and especially with the proliferation of
computers in astronomy, the equations of stellar structure have proven
themselves an incredibly predictive set of models.  

There are currently many stellar structure codes \citep[e.g.][]{Dotter2008,
Kovetz2009, Paxton2011} which integrate the equations of stellar structure ---
in addition to equations of state and lattices of nuclear reaction rates ---
over time to track the evolution of an individual star. The Dartmouth Stellar
Evolution Program (DSEP) \citep{Chaboyer2001, Bjork2006, Dotter2008} is one
such, well tested, stellar evolution program.

Here we propose to model low-mass stars in both the local solar neighborhood
and in globular clusters using DSEP. This work will primarily extend our
understanding of stellar physics in two areas: the effects of chemically
self-consistency on stellar models and time evolution of the core-convective
instabilities which ultimatly are belived to result in the observed paucity of
stars at a Gaia G magnitude of $\sim$10. [NEED CITATIONS IN THIS PARAGRAPH]

Low mass stars form an important component of the stellar population, with
stars less than [MASS HERE] making up more than 70\% of stars in the galaxy
[CITE]. Moreover, due to their long lives, low-mass stars provide essential
constraints on ages of various stellar populations [CITE]. In globular
clusters, where all stars are coeval to one of a limited number of populations,
low mass stars provide the vast majority of constraints when fitting ischrones [CITE].
Additionally, stars around the fully-convective transition mass show
age-dependent core-convective instabilities [CITE].

\subsection{Globular Clusters}

Globular clusters in the local universe are primarly composed of old and
consequently low-mass stars. For decades, prevaling thought had it globular
clusters were composed of a single stellar population born from a preisten
interstellar medium. This was supported by visibly tight main sequences and
clear main sequence turn offs in optical CMDs \citep[Figure
\ref{fig:M3CMD}][]{Sandage1953}. These early studies either did not handel or
had very large photometric uncertanties and therefore they were unable to
discriminate beteween CMD features with small separations,

\begin{figure}
	\centering
	\includegraphics[width=0.75\textwidth]{src/Figures/Gould53.png}
	\caption{$m_{pg}$ - $m_{pv}$ color-magnitude diagram for the globular cluster M3.}
	\label{fig:M3CMD}
\end{figure}

[SOMETHING ABOUT EARLY SPECTROSCOPIC INDICATIONS OF MPs]

With the presicion photometric measurements, degenerecies between noise and
intrinsic scatter were broken and it became clear that globular clusters are
almost universally composed of multiple stellar populations (MPs). [GRAB SOME
TEXT FROM THE NGC 2808 SECTION FOR HERE].

\subsection{Local Solar Neighborhood}
\citet{Jao2018} discovered a novel feature in the Gaia Gp-Rp color-magnitude
diagram. Around $M_{G}=10$ there is an approximatly 17\% decreas in stellar
density of the volume complete sample of stars \citeauthor{Jao2018} considered.
Subsequently, this has become known as either the Jao Gap, or Gaia M dwarf Gap.
Section \ref{} will go into more detail regarding the physics belived to
underpin this feature; however, in brief convective instabilities in the core
are belived to form for stars straddeling the fully convective transition mass.
These instabilities result in stars preferentially falling to either side of
the gap location.

Stellar modeling has been sucsessful in reproducing the Jap Gap and, with these
models, we have begun to constrain parameters which constrain gap location. For
example, it is now well documented that a stars metallicity can affect the gap
color by up to [HOW MUCH DID GREG FIND/CHECK FOR OTHER PAPERS ON THIS]. 

Initial testting which we have done using DSEP along with work by [PAPER] also
indicated the Jao Gap's color sensitivity to age. We observe that as models age
the Jao Gap moves [DIRECTION OF MOVMENT IN MAG AND COLOR SPACE].

The OPAL opacity tables in particular are very widely used by current
generation stellar evolution programs (in addition to current generation
stellar model and isochrone grids). However, they are no longer the most up
date elemental opacities. Moreover, the generation mechanism for these tables,
a webform, is no longer reliably online.  Consequently, it makes sense to
transition to more modern opacity tables with a more stable generation
mechanism.

Here we will present work transitioning DSEP from OPAL opacities to opacities
based on measurements from Los Alamos national Labs T-1 group
\citep[OPLIB][]{Colgan2016}. Moreover, we will present two projects which are
in large part reliant on these updated opacities. For the first project we
investigate the affects of chemically self consistent modeling of multiple
populations within the globular cluster NGC 2808, and for the second project we
present the effects of the OPLIB opacities on the location of the recently
discovered Gaia M-dwarf gap.

This paper is organized as follows. In Section \ref{sec:opac} we outline some
basic information about OPLIB opacities, how we query them, and how we modify
them to work with DSEP. In Section \ref{sec:2808} we discuss scientific
background of the first project along with the current work done towards its
goal. Finally, in Section \ref{sec:Jao} we present our findings on the effects
of OPLIB opacities on the location of the Gaia M-dwarf gap.







\section{Thesis Structure}\label{sec:thesis}
The thesis here proposed will be split into 5 chapters. Each chapter will
consist of work focusing on models low mass stars.

\subsection{Jao Gap \& Updated High Temperature Opacities}\label{sec:p1}
Due to initial mass requirements of the molecular clouds which collapse to form
stars, star formation is strongly biased towards lower mass, later spectral
class stars when compared to higher mass stars. Partly as a result of this
bias and partly as a result of their extremely long main-sequence lifetimes,
M-dwarfs make up approximately 70 percent of all stars in the galaxy.
Moreover, some planet search campaigns have focused on M-dwarfs due to the
relative ease of detecting small planets in their habitable zones
\citep[e.g.][]{Nut08}. M-dwarfs then represent both a key component of the
galactic stellar population as well as the possible set of stars which may host
habitable exo-planets. Given this key location M-dwarfs occupy in modern
astronomy it is important to have a thorough understanding of their structure
and evolution.

\subsubsection{Observations and Instability}
Gaia Data Release 2 (DR2) revealed a previously unknown structure in in the
$G_{\text{BP}}-G_{\text{RP}}$, $M_{G}$ color-magnitude diagram (Figure
\ref{fig:JaoGap}) corresponding to stars with a mass near that where a star
transitions from fully convective to having both convective and radiative
regions within (the fully convective transition mass --- 0.3 to 0.35
M$_{\odot}$) \citep{Jao2018,Baraffe2018}. The so-called Gaia M-dwarf gap, or
Jao gap, represents a decrease in luminosity and commensurately a decrease in
stellar density --- by approximately 17\% --- over this mass range.


\begin{figure}
	\centering
	\includegraphics[width=0.45\textwidth]{src/Figures/JaoGap.png}
	\caption{Figure 1 from \citet{Jao2018} showing the so called ``Jao Gap'' at
	$M_{G}\approx$ 10}
	\label{fig:JaoGap}
\end{figure}

A theoretical explanation for such a density deficiency comes from
\citet{van2012}, who propose that directly above the transition mass between a
star with a radiative core and convective envelope and a fully convective star,
due to asymmetric production and destruction of He$^{3}$ during the
proton-proton I chain (ppI), periodic luminosity variations can be induced.
This process is known as convective-kissing instability.  Take for example a
star with a mass right on the fully convective transition.  Such a star will
descend the pre-MS with a radiative core; however, as the star reaches the zero
age main sequence (ZAMS) and as the core temperature exceeds $7\times 10^{6}$
K, enough energy will be produced by the ppI chain that the core becomes
convective. At this point the star exists with both a convective core and
envelope, in addition to a thin, radiative, layer separating the two.
Subsequently, asymmetries in ppI affect the evolution of the star's convective
core.

The proton-proton I chain constitutes three reactions 
\begin{enumerate} 
	\item $p + p \longrightarrow d + e^{+} + \nu_{e}$
	\item $p + d \longrightarrow \ ^{3}\text{He} + \gamma$
	\item $^{3}\text{He} + ^{3}\text{He} \longrightarrow \ ^{3}\text{He} + 2p$ 
\end{enumerate} 
Because reaction 3 of ppI consumes $^{3}$He at a slower rate than it is
produced by reaction 2, $^{3}$He abundance increases in the core increasing
energy generation. The core convective zone will therefore expand as more of
the star becomes unstable to convection. This expansion will continue until the
core connects with the convective envelope. At this point convective mixing can
transport material throughout the entire radius of the star and the high
concentration of $^{3}$He will rapidly diffuse outward, away from the core,
again decreasing energy generation as reaction 3 slows down. Ultimately, this
leads to the convective region around the core pulling back away from the
convective envelope, leaving in place the radiative transition zone, at which
point $^{3}$He concentrations build up in the core until it once again expands to
meet the envelope.  This process repeats until chemical equilibrium is reached
throughout the star and the core can sustain high enough nuclear reaction rates
to maintain contact with the envelope, resulting in a fully convective star.


\subsubsection{Modeling the Gap}
Since the identification of the Gaia M-dwarf gap, stellar modeling has been
conducted to better constrain its location, effects, and exact cause.
Both \citet{Mansfield2021} and \citet{Feiden2021} identify that the gap's mass
location is correlated with model metallicity --- the mass-luminosity
discontinuity in lower metallicity models being at a commensurately lower mass.
\citet{Feiden2021} suggests this dependence is due to the steep relation of
the radiative temperature gradient, $\nabla_{rad}$, on temperature and in turn,
on stellar mass.

\begin{align}\label{eqn:radGrad}
	\nabla_{rad} \propto \frac{L\kappa}{T^{4}}
\end{align}

As metallicity decreases so does opacity, which, by Equation \ref{eqn:radGrad},
dramatically lowers the temperature where radiation will dominate energy transport
\citep{Chabrier1997}. Since main sequence stars are virialized the core
temperature is proportional to the core density and total mass (Equation
\ref{eqn:TMRelation}). Therefore, if the core temperature where
convective-kissing instability is expected decreases with metallicity, so too
will the mass of stars which experience such instabilities.

\begin{align}\label{eqn:TMRelation}
	T_{c} \propto \rho_{c}M^{2}
\end{align}

This strong opacity dependence presents a slight problem where modeling is 
concerned. With current computational tools it is infeasible to compute opacities on the
fly; rather, Rossland Mean opacity ($\kappa_{R}$) for individual elements must
be pre-tabulated over a wide range of temperatures and densities. These
opacities can then be somewhat arbitrarily mixed together and interpolated to
form opacity lookup-tables. Multiple groups have performed these calculations
and subsequently made tables available to the wider community, these include
the Opacity Project \citep[OP][]{Seaton1994}, Laurence Livermore National Labs
OPAL opacity tables \citep{Iglesias1996}, and Los Alamos National Labs OPLIB
opacity tables \citep{Colgan2016}.

The OPAL opacity tables in particular are very widely used by current
generation stellar evolution programs (in addition to current generation
stellar model and isochrone grids). However, they are no longer the most
up-to-date elemental opacities or numerically precise. Moreover, the generation
mechanism for these tables, a webform, is no longer reliably online.  

Given its strong theoretical opacity dependence, it is reasonable to expect
updated opacity tables to affect, the Jap Gap's mass range. Therefore, as part
of this project we have transitioned DSEP from OPAL high temperature opacities
to opacities based on measurements from Los Alamos national Labs T-1 group
\citep[OPLIB][]{Colgan2016}. This chapter in the thesis will detail the opacity
transition, provide validation of the new opacity tables, and conduct an
in-depth statistical comparison between the locations of Jao Gaps from
populations modeled using both OPAL and OPLIB opacities.

Preliminary work shows populations modeled with OPLIB opacities have Jao Gaps
at consistently lower masses than those modeled using OPAL opacities (Figure
\ref{fig:JaoGapOPALOPLIB} and Table \ref{tab:fineMassRange}). This is in line
with expectations based on OPLIB opacities begin uniformly lower than OPAL
opacities for temperature above $10^{6}$ K (Figure \ref{fig:opacComp}) --- with
the lower opacities requiring commensurately lower core temperatures before
radiation dominates energy transport. 

\begin{figure}
	\centering
	\includegraphics[width=0.85\textwidth]{src/Figures/OPALOPLIB_popsynth_comp.pdf}
	\caption{Synthetic CMDs derived from simple population synthesis code.
	(Left) CMD showing the Jao Gap for a GS98 composition stellar population
	generate from models evolved using OPAL opacity tables. (Right) CMD showing
	the Jao Gap for a GS98 stellar population generated from models evolved
	using the OPLIB opacity tables. Note how the OPLIB derived Jao Gap is
	slightly brighter than the OPAL Jao Gap.}
	\label{fig:JaoGapOPALOPLIB}
\end{figure}

\begin{wrapfigure}{r}{0.5\textwidth}
	\centering
	\includegraphics[width=0.48\textwidth]{src/Figures/OpacityComparision.pdf}
	\caption{Rossland Mean Opacity from both OPAL (black solid) and OPLIB (red
	dashed). Note how above 10$^{6}$ K OPLIB opacities are uniformly smaller
	than those from OPAL.}
	\label{fig:opacComp}
\end{wrapfigure}

% \subsubsection{OPLIB Opacities}
% In order to address the two main issues with using OPAL opacity tables we use
% our OPLIB opacity table web scraper to generate a set of tables that
% consistently model lower metallicities. Specifically, we generate tables for
% $Z_{\odot}=0.017$, $Z=0.01$, $Z=0.001$, and $Z=0.0001$. Compositions are
% derived from the GS98 solar composition, with the mass fractions between metals
% remaining constant, and only the total metal mass fraction is allowed to vary.
% Moreover, Helium mass fraction is held constant as extra mass from the reduced
% metallicity is put into additional Hydrogen. 
%
% For each metallicity 101, uniformly spaced, models from 0.3 to 0.5 M$_{\odot}$
% (spacing of 0.001 M$_{\odot}$) are evolve with both the GS98 OPAL opacity table
% and OPLIB tables, hereafter these are the ``coarse'' models. For each set of
% coarse models the discontinuity in the mass-luminosity relation is identified
% at an age of 7 Gyr (Figures \ref{fig:coarseMassLum} \& \ref{fig:coarseTeffLum}
% shows a characteristic example).
%
% Immediately, the difference in mass where the discontinuity manifests is clear.
% For each metallicity the discontinuity in the OPLIB models is approximately one
% one-hundredth of a solar mass lower than the discontinuity in the OPAL models. We can
% validate that this discontinuity is indeed correlated with the convective
% transition mass; Figure \ref{fig:convTransition} shows an example of the model
% forming radiative zones at approximately the same masses where the discontinuity
% in the mass-luminosity function exists.
%
% At this resolution only a few models exist within the
% mass range of the discontinuity. In order to better constrain its location we
% run a series of ``fine'' models, with a mass step of 0.0001 M$_{\odot}$ and
% ranging from where the mass derivative first exceeds two sigma away from the
% mean derivative value up to the mass where it last exceeds two sigma away from
% the mean. A characteristic fine mass-luminosity relation is shown in Figure
% \ref{fig:fineMassLum}.


% \begin{figure}
% 	\centering
% 	\vspace{5mm}
% 	\includegraphics[width=0.45\textwidth]{src/Figures/ConvectiveMassFraction.pdf}
% 	\caption{Convective Mass Fraction vs. initial model mass for Z=0.01 at 7
% 	Gyr (top), Derivative of luminosity with respect to mass for the OPAL and
% 	OPLIB models (bottom).  Note how the model transitions from fully
% 	convective at approximately the same mass where the discontinuity exists.}
% 	\label{fig:convTransition}
% \end{figure}
%
%
% \begin{figure}
% 	\centering
% 	\vspace{5mm}
% 	\includegraphics[width=0.45\textwidth]{src/Figures/MassLumDisconZ001Paper-fine.pdf}
% 	\caption{Mass-Luminosity relation for Z=0.01 at 7 Gyr for models run with
% 	both OPAL and OPLIB high-temperature opacity tables and a mass step between
% 	them of 0.0001 M$_{\odot}$ (top). Derivative of luminosity with respect to
% 	mass for the OPAL and OPLIB models (bottom).}
% 	\label{fig:fineMassLum}
% \end{figure}

% Using the fine models we identify the location of the discontinuity in the same
% manner as before, results of this are presented in Table
% \ref{tab:fineMassRange}. Of note with the mass ranges we measure for the
% discontinuity is that are generally not in agreement with those measured in
% \citet{Mansfield2021}. However, the luminosity difference from over the gap
% ($\approx 0.1 mag$) is similar to both the observational difference and that
% reported in \citet{Mansfield2021}. Currently, it is not clear why our mass
% range is not in agreement with the \citet{Mansfield2021} mass range and further
% investigation is therefore needed.

\begin{table*}
	\centering
	\begin{tabular}{r | c c c c}
		\hline
		$Z=$ & Z$_{\odot}$ & 0.01 & 0.001 & 0.0001 \\
		\hline
		\hline
		OPAL & 0.3803 - 0.384 & 0.3583 - 0.3631 & 0.34 - 0.3448 & 0.362 - 0.3663 \\
		OPLIB & 0.374 - 0.3767 & 0.3526 - 0.3567 & 0.3358 - 0.3406 & 0.3577 - 0.3621
	\end{tabular}
	\caption{Mass ranges for the discontinuity in OPAL and OPLIB models. Masses
	are given in solar masses.}
	\label{tab:fineMassRange}
\end{table*}


\subsection{Jao Gap Ageing - 1}\label{sec:p2}
Following the integration of updated high temperature opacities detiled in \S
\ref{sec:p1} we will investigate using the Jao-Gap color to age the local
solar neighboorhood.

Models predict that the location of the Jap Gap will shift with population age
[CITATION]; in fact, we see this behavior, gap colors reddening as populations
age, in populations evolved with DSEP [FIGURE] which span the mass range of the
gap.

[DETAILS ON KINEMATIC AGEING]

We propose to model a population of stars of various ages and mettalicities
sampled from the local stellar neighboorhood. Each of these stars will be
assigned kinematics --- again sampled from empirical distributions. We will
then extract kinematically derived ages from this population and use these to
segregate stars into rough age bins. Finally, we will measure if difference in
Jao gap locations are statistically distinquishable between these rough age bins.


\subsection{Jao Gap Ageing - 2}\label{sec:p3}
Where the project laid out in \S \ref{sec:p2} study the feasibility of using the
Jao Gap to date stellar populations; this project will apply this technique to
observational data. Specifically, we measure the Jao Gap location in populations
separated by gyro-kinematic age-dating in the solar neighboorhood.


\subsection{NGC 2808}\label{sec:p4}
\section{INTRODUCTION}
Over the last half of the 19th and first decade of the 20th centuries Lane,
Ritter, and Emden codified the earliest mathematical model of stellar
structure, the polytrope (Equation \ref{eqn:polytrope}), in \textit{Gaskugeln}
(Gas Balls) \citep{Emden1907}.

\begin{align}\label{eqn:polytrope}
	\frac{d}{d\xi}\left(\xi^{2}\frac{d\theta}{d\xi}\right) = -\xi^{2}\theta^{n}
\end{align}

Where $\xi$ and $\theta$ are dimensionless parameterizations of radius and
temperature respectively, and $n$ is known as the polytropic index. Despite this
early work, it wasn't until the late 1930s and early 1940s that the full set of
equations needed to describe the structure of a steady state,
radially-symmetric, star (known as the equations of stellar structure) began to
take shape as proton-proton chains and the Carbon-Nitrogen-Oxygen cycle were,
for the first time, seriously considered as energy generation mechanisms
\citep{Cowling1966}. Since then, and especially with the proliferation of
computers in astronomy, the equations of stellar structure have proven
themselves an incredibly predictive set of models.  

There are currently many stellar structure codes \citep[e.g.][]{Dotter2008,
Kovetz2009, Paxton2011} which integrate the equations of stellar structure ---
in addition to equations of state and lattices of nuclear reaction rates ---
over time to track the evolution of an individual star. The Dartmouth Stellar
Evolution Program (DSEP) \citep{Chaboyer2001, Bjork2006, Dotter2008} is one
such, well tested, stellar evolution program.

Here we propose to model low-mass stars in both the local solar neighborhood
and in globular clusters using DSEP. This work will primarily extend our
understanding of stellar physics in two areas: the effects of chemically
self-consistency on stellar models and time evolution of the core-convective
instabilities which ultimatly are belived to result in the observed paucity of
stars at a Gaia G magnitude of $\sim$10. [NEED CITATIONS IN THIS PARAGRAPH]

Low mass stars form an important component of the stellar population, with
stars less than [MASS HERE] making up more than 70\% of stars in the galaxy
[CITE]. Moreover, due to their long lives, low-mass stars provide essential
constraints on ages of various stellar populations [CITE]. In globular
clusters, where all stars are coeval to one of a limited number of populations,
low mass stars provide the vast majority of constraints when fitting ischrones [CITE].
Additionally, stars around the fully-convective transition mass show
age-dependent core-convective instabilities [CITE].

\subsection{Globular Clusters}

Globular clusters in the local universe are primarly composed of old and
consequently low-mass stars. For decades, prevaling thought had it globular
clusters were composed of a single stellar population born from a preisten
interstellar medium. This was supported by visibly tight main sequences and
clear main sequence turn offs in optical CMDs \citep[Figure
\ref{fig:M3CMD}][]{Sandage1953}. These early studies either did not handel or
had very large photometric uncertanties and therefore they were unable to
discriminate beteween CMD features with small separations,

\begin{figure}
	\centering
	\includegraphics[width=0.75\textwidth]{src/Figures/Gould53.png}
	\caption{$m_{pg}$ - $m_{pv}$ color-magnitude diagram for the globular cluster M3.}
	\label{fig:M3CMD}
\end{figure}

[SOMETHING ABOUT EARLY SPECTROSCOPIC INDICATIONS OF MPs]

With the presicion photometric measurements, degenerecies between noise and
intrinsic scatter were broken and it became clear that globular clusters are
almost universally composed of multiple stellar populations (MPs). [GRAB SOME
TEXT FROM THE NGC 2808 SECTION FOR HERE].

\subsection{Local Solar Neighborhood}
\citet{Jao2018} discovered a novel feature in the Gaia Gp-Rp color-magnitude
diagram. Around $M_{G}=10$ there is an approximatly 17\% decreas in stellar
density of the volume complete sample of stars \citeauthor{Jao2018} considered.
Subsequently, this has become known as either the Jao Gap, or Gaia M dwarf Gap.
Section \ref{} will go into more detail regarding the physics belived to
underpin this feature; however, in brief convective instabilities in the core
are belived to form for stars straddeling the fully convective transition mass.
These instabilities result in stars preferentially falling to either side of
the gap location.

Stellar modeling has been sucsessful in reproducing the Jap Gap and, with these
models, we have begun to constrain parameters which constrain gap location. For
example, it is now well documented that a stars metallicity can affect the gap
color by up to [HOW MUCH DID GREG FIND/CHECK FOR OTHER PAPERS ON THIS]. 

Initial testting which we have done using DSEP along with work by [PAPER] also
indicated the Jao Gap's color sensitivity to age. We observe that as models age
the Jao Gap moves [DIRECTION OF MOVMENT IN MAG AND COLOR SPACE].

The OPAL opacity tables in particular are very widely used by current
generation stellar evolution programs (in addition to current generation
stellar model and isochrone grids). However, they are no longer the most up
date elemental opacities. Moreover, the generation mechanism for these tables,
a webform, is no longer reliably online.  Consequently, it makes sense to
transition to more modern opacity tables with a more stable generation
mechanism.

Here we will present work transitioning DSEP from OPAL opacities to opacities
based on measurements from Los Alamos national Labs T-1 group
\citep[OPLIB][]{Colgan2016}. Moreover, we will present two projects which are
in large part reliant on these updated opacities. For the first project we
investigate the affects of chemically self consistent modeling of multiple
populations within the globular cluster NGC 2808, and for the second project we
present the effects of the OPLIB opacities on the location of the recently
discovered Gaia M-dwarf gap.

This paper is organized as follows. In Section \ref{sec:opac} we outline some
basic information about OPLIB opacities, how we query them, and how we modify
them to work with DSEP. In Section \ref{sec:2808} we discuss scientific
background of the first project along with the current work done towards its
goal. Finally, in Section \ref{sec:Jao} we present our findings on the effects
of OPLIB opacities on the location of the Gaia M-dwarf gap.






\input{chapters/ngc2808/subsections/observations}
\section{Modeling}\label{sec:modeling}
One of the most pressing questions related to this work is whether or not the
increased star-to-star variability in the activity metric and the Jao Gap,
which are coincident in magnitude, are driven by the same underlying mechanism.
The challenge when addressing this question arises from current computational
limitations. Specifically, the kinds of three dimensional
magneto-hydrodynamical simulations --- which would be needed to derive the
effects of convective kissing instabilities on the magnetic field of the star
--- are infeasible to run over gigayear timescales while maintaining thermal
timescale resolutions needed to resolve periodic mixing events.

In order to address this and answer the specific question of \textit{could
kissing instabilities result in increased star-to-star variability of the
magnetic field}, we adopt a very simple toy model. Kissing instabilities result
in transient radiative zone separating the core of a star (convective) from its
envelope (convective). When this radiative zone breaks down two important
things happen: one, the entire star becomes mechanically coupled, and two,
convective currents can now move over the entire radius of the star.
\citet{Jao2023} propose that this mechanical coupling may allow the stars core
to act as an angular momentum sink thus accelerating a stars spin down and
resulting in anomalously low H$\alpha$ emission. 

Regardless of the exact mechanism by which the magnetic field may be effected,
it it reasonable to expect that both the mechanical coupling and the change to
the scale of convective currents will have some effect on the stars magnetic
field. On a microscopic scale both of these will change how packets of charge
within a star move and may serve to disrupt a stable dynamo. Therefore, in the
model we present here we make only one primary assumption: \textit{every mixing
event may modify the stars magnetic field by some amount}. Within our model
this assumption manifests as a random linear perturbation applied to some base
magnetic field at every mixing event. The strength of this perturbation is 
sampled from a normal distribution with some standard deviation, $\sigma_{B}$.

Synthetic stars are sampled from a grid of stellar models evolved using the
Dartmouth Stellar Evolution Program (DSEP). Each stellar model was evolved
using a high temporal resolution (timesteps no larger than 10,000 years
{\color{red} Check this}) and typical numerical tolerances of one part in
$10^5$. Each model was based on a GS98 \citep{Grevesse1998} solar
composition with a mass range from 0.3 M$_{\odot}$ to 0.4 M$_{\odot}$. Finally,
models adopt OPLIB high temperature radiative opacities, Ferguson 2004 low
temperature radiative opacities, and include both atomic diffusion and
gravitational settling. A Kippenhan-Iben diagram showing the structural
evolution of a model within the gap is shown in Figure \ref{fig:kippenhan}.

\begin{figure*}
  \centering
  \includegraphics[width=0.9\textwidth]{figures/jaoMagActivity/Kippenhan_clamped.pdf}
  \caption{Kippenhan-Iben diagram for a 0.345 solar mass star. Note the
  periodic mixing events (where the plotted curves peak).}
  \label{fig:kippenhan}
\end{figure*}

Each synthetic star is assigned some base magnetic activity ($B_{0} \sim
\mathcal{N}(1, \sigma_{B})$) and then the number of mixing events before some age $t$
are counted based on local maxima in the core temperature. The toy magnetic
activity at age $t$ for the model is given in Equation \ref{eqn:activity}. An
example of the magnetic evolution resulting from this model is given in Figure
\ref{fig:simpleB}. Fundamentally, this model presents magnetic
activity variation due to mixing events as a random walk and therefore results will
increasingly divergence over time.

\begin{align}\label{eqn:activity}
  B(t) = B_{0} + \sum_{i}B_{i} \sim \mathcal{N}(1, \sigma_{B}) 
\end{align}

\begin{figure}
  \centering
  \includegraphics[width=0.45\textwidth]{figures/jaoMagActivity/simpleBEvolution.pdf}
  \caption{Example of the toy model presented here resulting in increased
  divergence between stars magnetic fields. The shaded region represents the
  maximum spread in the two point correlation function at each age.}
  \label{fig:simpleB}
\end{figure}

Applying the same analysis to these models as was done to the observations as
described in Section {\color{red} X.X} we find that this simple model results
in a qualitatively similar trend in the standard deviation vs. Magnitude graph
(Figure \ref{fig:model}). In order to reproduce the approximately 50 percent
change to the spread of the activity metric observed in the combined dataset in
section \ref{sec:results} a distribution with a standard deviation of 0.1 is required when sampling the change in the magnetic activity metric at each mixing event. This corresponds to 68\% of mixing events modifying the activity strength by 10 percent or less. The interpretation here is important, what
this qualitative similarity demonstrates is that it may be reasonable to expect
kissing instabilities to result in the observed increased star-to-star
variation. Importantly, we are not able to claim that kissing instabilities
\textit{do} lead to these increased variations, only that they reasonably
could. Further modeling, observational, and theoretical efforts will be needed
to more definitively answer this question.

\begin{figure}
  \centering
  \includegraphics[width=0.45\textwidth]{figures/jaoMagActivity/SpreadModel.pdf}
  \caption{Toy model results showing a qualitatively similar discontinuity in the star-to-star magnetic activity variability.}
  \label{fig:model}
\end{figure}

\subsection{Limitations}
The model presented in this paper is very limited and it is important to keep
those limitations in mind when interpreting the results presented here. Some of
the main challenges which should be leveled at this model are the assumption
that the magnetic field will be altered by some small random perturbation at
every mixing event. This assumption was informed by the large number of free
parameters available to a physical star during the establishment of a large scale 
magnetic field and the associated likely stochastic nature of that process.
However, it is similarly believable that the magnetic field will tend to alter in
a uniform manner at each mixing event. For example, since differential rotation
is generally proportional to the temperature gradient within a star and activity is
strongly coupled to differential rotation then it may be that as the radiative zone reforms over thermal timescales the homogenization of angular momentum throughout the star results in overall lower amounts of differential rotation each after mixing event than would otherwise be present.

Moreover, this model does not consider how other degenerate sources of magnetic evolution such as stellar spin down, relaxation, or coronal heating may effect star-to-star variability. These could conceivably lead to a similar increase in star-to-star variability which is coincident with the Jao Gap magnitude as the switch from fully to partially convective may effect efficiency of these process.

Additionally, there are challenges with this toy model that originate from the stellar evolutionary model. Observations of the Jao Gap show that the feature is not perpendicular to the magnitude axis; rather, it is inversely proportional to the color. No models of the Jao Gap published at the time of writing capture this color dependency and \textit{what causes this color dependency} remains one of the most pressing questions relating to the underlying physics. This non captured physics is one potential explanation for why the magnitude where our model predicts the increase in variability is not in agreement with where the variability jump exists in the data.

Finally, we have not considered detailed descriptions of the dynamos of stars. The magnetohydrodynamical modeling which would be required to model the evolution of the magnetic field of these stars at thermal timescale resolutions over gigayears is currently beyond the ability of practical computing. Therefore future work should focus on limited modeling which may inform the evolution of the magnetic field directly around the time of a mixing event.

\section{Chemical Consistency}\label{sec:const}
There are three primary areas in which must the stellar models must be made chemically
consistent: the atmospheric boundary conditions, the opacities, and interior
abundances. The interior abundances are relatively easily handled by adjusting
parameters within our stellar evolutionary code. However, the other two areas are
more complicated to bring into consistency. Atmospheric boundary conditions and
opacities must both be calculated with a consistent set of chemical abundances
outside of the stellar evolution code. For evolution we use the Dartmouth Stellar Evolution Program (DSEP) \citep{Dotter2008}, a well tested 1D stellar evolution code which has a particular focus on modelling low mass stars ($\le 2$ M$_{\odot}$)

\subsection{Atmospheric Boundary Conditions}\label{sec:atm}
Certain assumptions, primarily that the radiation field is at equilibirum and radiative transport is diffusive \citep{Salaris2005}, made in stellar structure
codes, such as DSEP, are valid when the optical depth of a star is small.
However, in the atmospheres of stars, the number density of particles drops low
enough and the optical depth consequently becomes large enough that these
assumptions break down, and separate, more physically motivated, plasma modeling code is required.
Generally structure code will use tabulated atmospheric boundary conditions
generated by these specialized codes ATLAS9 \citep{Kurucz1993}, PHEONIX \citep{Husser2013}, MARCS \citep{Gustafsson2008}, and MPS-ATLAS \citep{Kostogryz2023}. Often, as the boundary conditions are both expensive to compute
and not the speciality of stellar structure researchers, the boundary
conditions are not updated as as light-element interior abundance varies. 

One key element when chemically consistently modeling NGC 2808 modeling is the
incorporation of new atmospheric models with the same elemental abundances as
the structure code. We use atmospheres generated from the \texttt{MARCS} grid
of model atmospheres \citep{Plez2008}. \texttt{MARCS} provides one-dimensional,
hydrostatic, plane-parallel and spherical LTE atmospheric models
\citep{Gustafsson2008}. Model atmospheres are made to match the
spectroscopically measured elemental abundances of populations A and E.
Moreover, for each populations, atmospheres with  various helium mass fractions
are generated. These range from Y=0.24 to Y=0.36 in steps of 0.02. All
atmospehric models are computed to an optical depth of $\tau = 100$ where their
temperature and pressures serves as boundary conditions for the strudcure code.
A comparison of the pressure and temperature throughout the atmospheres of the
two populations with helium abundances representative of literature values is
shown
in Figure \ref{fig:AEAtmComp}.

\begin{figure}
	\centering
	\includegraphics[width=0.85\textwidth]{figures/ngc2808/notebookFigures/AtmosphereComparison.pdf}
	\label{fig:AEAtmComp}
	\caption{Comparison of the MARCS model atmospheres generated for the two
	extreme populations of NGC 2808. These lines shows population A and E with
	the same Helium abundance; though, we fit a grid of models over various
	helumn abundances. Dashed lines show the temperature of the boundary
	condition while sold lines show the pressure.}
\end{figure}


\subsection{Opacities}\label{sec:opac}
In addition to the atmospheric boundary conditions, both the high and low
temperature opacities used by DSEP must be made chemically consistent. Here we
use OPLIB high temperature opacity tables \citep{Colgan2016} retrieved using
the TOPS web-interface. Low temperature opacity tables are retrieved from the
Aesopus 2.0 web-interface \citep{Marigo2009, Marigo2022}. Ideally, these
opacities would be the same used in the atmospheric models. However, the
opacities used in the MARCS models are not publicly available. As such, we use
the opacities provided by the TOPS and Aesopus 2.0 web-interfaces.

\section{fidanka}\label{sec:fidanka}
When fitting isochrones to the data we have four main criteria for any method

\begin{itemize}
	\item The method must be robust enough to work along the entire main sequence, turn off, and much of the subgiant and red giant branchs.
	\item Any method should consider photometric uncertainty in the fitting process.
	\item The method should be model independent, weighting any n number of populations equally.
	\item The method should be automated and require minimal intervention from the user.
\end{itemize}


We do not believe that any currently available software is a match for
our use case. Therefore, we elect to develop our own software suite, \fidanka.
\fidanka is a python package designed to automate much of the process of
measuring fiducial lines in CMDs, adhering to the four criteria we lay out
above. Primary features of \fidanka may be separated into three
categories: fiducial line measurement, stellar population synthesise, and
isochrone optimization/fitting. Additionally, there are utility functions which
are detailed in the \fidanka documentation.

\subsection{Fiducial Line Measurement}
\fidanka takes a iterative approach to measuring fiducial lines, the first step
of which is to make a ``guess'' as to the fiducial line. . This initial guess
is calculated by splitting the CMD into magnitude bins, with uniform numbers of
stars per bin (so that bins are cover a small magnitude range over densely
populated regions of the CMD while covering a much larger magnitude range in
sparsely populated regions of the CMD, such as the RGB). A unimodal Gaussian
distribution is then fit to the color distribution of each bin, and the
resulting mean color is used as the initial fiducial line guess. This rough
fiducial line will approximately trace the area of highest density. The initial
guess will be used to verticalze the CMD so that further algorithms can work in
1-D magnitude bins without worrying about weighting issues caused by varying
projections of the evolutionary sequence onto the magnitude axis.
Verticalization is preformed taking the difference between the guess fiducial
line and the color of each star in the CMD.

If \fidanka were to simply apply the same algorithm to the verticalized CMD
then the resulting fiducial line would likely be a re-extraction of the initial
fiducial line guess. To avoid this, we take a more robust, number density based
approach, which considers the distribution of stars in both color and magnitude
space simultaneously. For each star in the CMD we first using a
\texttt{introselect} partitioning algorithm to select the 50 nearest stars in F814W vs. F275W-F814W space.
To account for the case where the star is at an extreme edge of the CMD, those
50 stars include the star itself (such that we really select 49 stars + 1). We
use \texttt{qhull}\footnote{https://www.qhull.com}\citep{Barber1996, } to
calculate the convex hull of those 50 points. The number density at each star
then is defined as $50/A_{hull}$, where $A_{hull}$ is the area of the convex
hull. Because we use a fixed number of points per star, and a partitioning
algorithm as opposed to a sorting algorithm, this method scales like
$\mathcal{O}(n)$, where n is the number of stars in the CMD. This method also
intrinsically weights the density of of each star equally as the counting
statistics per bin are uniform. We are left with a CMD where each star
has a defined number density (Figure \ref{fig:densityMapDemo}).

\begin{figure*}
	\centering
	\includegraphics[width=0.9\textwidth]{figures/ngc2808/notebookFigures/DensityMapDemo.png}
	\label{fig:densityMapDemo}
	\caption{Density map demo showing density estimate over different parts of
	the evolutionary sequence. The left panel shows the density map over the
	entire evolutionary sequence, while the middle panel shows the density map
	over the main sequence and the right most panel shows the density map over
	the RGB. Figures in the top row are the raw CMD, while figures in the
	bottom row are colored by the density map.}
\end{figure*}

\fidanka can now exploit this density map to fit a better fiducial line to the
data, as the density map is far more robust to outliers. There are multiple
algorithms we implement to fit the fiducial line to the color-density profile
in each magnitude bin (Figure \ref{fig:densityBinsDemo}); they are explained in more detail in the \fidanka
documentation. However, of most relevance here is the Bayesian Gaussian Mixture
Modeling (BGMM) method. BGMM is a clustering algorithm which, for some fixed
number of n-dimensional Gaussian distributions, $K$, determines the mean, covariance, and
mixing probability (somewhat analogous to amplitude) of each $k^{th}$
distribution, such that the local lower bound of the evidence of each star
belonging strongly to a single distribution is maximized. 

\begin{figure}
	\centering
	\includegraphics[width=0.85\textwidth]{figures/ngc2808/notebookFigures/DensityBinsDemo.png}
	\label{fig:densityBinsDemo}
	\caption{CMD where points are colored by density. Lines show the
	density-color profile in each magnitude bin. In this figure adaptive
	binning targeted 1000 stars per bin}
\end{figure}

Maximization is preformed using the Dirichlet process, which is a
non-parametric Bayesian method of determining the number of Gaussian distributions, $K$,
which best fit the data \citep{Ferguson1973, scikit-learn}. Use of the Dirichlet process
allows for dynamic variation in the number of inferred populations from
magnitude bin to magnitude bin. Specifically, populations are clearly visually
separated from the lower main sequence through the turn off; however, at the
turn off and throughout much of the subgiant branch, the two visible
populations overlap due to their extremely similar ages \citep[i.e.][]{Jordan2002}. The Dirichlet process allows for the BGMM method to infer a single
population in these regions, while inferring two populations in regions where
they are clearly separated. More generally, the use of the Dirichlet process
removes the need for a prior on the exact number of populations to fit. Rather,
the user specifies a upper bound on the number of populations within the
cluster. An example bin (F814W = 20.6) is shown in Figure \ref{fig:BGMMDist}.

\begin{figure*}
	\centering
	\includegraphics[width=0.9\textwidth]{figures/ngc2808/BGMMMixingBin.pdf}
	\caption{Example of BGMM fit to a magnitude bin. The grey line shows the
	underlying color-density profile, while the black dashed-line shows the
	joint distribution of each BGMM component. The solid black lines show the
	two selected components.}
	\label{fig:BGMMDist}
\end{figure*}

\fidanka's BGMM method first breaks down the verticalized CMD into magnitude
bins with uniform numbers of stars per bin (here we adopt 250). Any stars left
over are placed into the final bin. For each bin a BGMM model with a maximum of
5 populations is fit to the color density profile. The number of populations is
then inferred from the weighting parameter (the mixing probability) of each
population. If the weighting parameter of any $k^{th}$ components less than
{\color{blue}0.05}, then that component is considered to be spurious and
removed. Additionally, if the number of populations in the bin above and the
bin below are the same, then the number of populations in the current bin is
forced to be the same as the number of populations in the bin above. Finally,
the initial guess fiducial line is added back to the BGMM inferred line. Figure
\ref{fig:vertFit} shows the resulting fiducial line(s) in each magnitude bin
for both a verticalized CMD and a non verticalized CMD.

\begin{figure}
	\centering
	\includegraphics[width=0.85\textwidth]{figures/ngc2808/vertFit.png}
	\caption{CMD where points are colored by density. Line trace the infered 
	fiducial line(s) in each magnitude bin.}
	\label{fig:vertFit}
\end{figure}

This method of fiducial line extraction effectively discriminated between
multiple populations long the main sequence and RGB of a cluster, while
simultaneously allowing for the presence of a single population along the MSTO
and subgiant branch. 

We can adapt this density map based BGMM method to consider photometric
uncertainties by adopting a simple Monte Carlo approach. Instead of measuring
the fiducial line(s) a single time, \fidanka can measure the fiducial line(s)
many times, resampling the data with replacement each time. For each resampling
\fidanka adds a random offset to each filter based on the photometric
uncertainties of each star. From these $n$ measurments the mean fiducial line
for each sequence can be identified along with upper and lower bound confidence
intervals in each magnitude bin.

\subsection{Stellar Population Synthesis}
In addition to measuring fiducial lines, \fidanka also includes a stellar
population synthesise module. This module is used to generate synthetic CMDs
from a given set of isochrones. This is of primary importance for binary
population modelling. The module is also used to generate synthetic CMDs for
the purpose of testing the fiducial line extraction algorithms against priors.

\fidanka uses MIST formatted isochrones \citep{Dotter2016} as input along
with distance modulus, B-V color excess, binary mass fraction, and bolometric
corrections. An arbitrarily large number of isochrones may be used to define an
arbitrary number of populations. Synthetic stars are samples from each
isochrone based on a definable probability (for example it is believed that
$\sim90\%$ of stars in globular clusters are younger population
\citep[e.g.][]{Suntzeff1996, Carretta2013}). Based on the metallicity, $\mu$, and E(B-V) of each
isochrone, bolometric corrections are taken from bolometric correction tables.
Where bolometric correction tables do not include exact metallicities or
extinctions a linear interpolation is preformed between the two bounding
values. 

\subsection{Isochrone Optimization}
The optimization routines in \fidanka will find the best fit distance modulus,
B-V color excess, and binary number fraction for a given set of isochrones. If
a single isochrone is provided then the optimization is done by minimizing the
$\chi^2$ of the perpendicular distances between an isochrone and a fiducial
line. If multiple isochrones are provided then those isochrones are first used
to run stellar population synthesis and generate a synthetic CMD. The
optimization is then done by minimizing the $\chi^2$ of both the perpendicular
distances between and widths of the observed fiducial line and the fiducial
line of the synthetic CMD.


\subsection{Fidanka Testing}
In order to validate fidanka we have run an series of injection recovery tests
using \fidanka's population synthesis routines to build various synthetic
populations and \fidanka's fiducial measurement routines to recover these
populations. Each population was generated using the initial mass function
given in \citep{Milone2012} for the redmost population ($\alpha=-1.2$).
Further, every population was given a binary population fraction of 10\%,
distance uniformly sampled between 5000pc and 15000pc, and a B-V color excess
uniformly sampled between 0 and 0.1. Finally, each synthetic population was
generated using a fixed age  uniformlly sampled between 7 Gyr and 14 Gyr. An
example synthetic population along with its associated best fit isochrone are
shown in Figure \ref{fig:ValidationBestFit}.

\begin{figure}
  \centering
  \includegraphics[width=0.85\textwidth]{figures/ngc2808/ExtractedIsoFit.pdf}
  \caption{Synthetic population generated by fidanka at 10000pc with E(B-V) = 0, and an age of 12 Gyr along with the best fitting isochrone. The best fit paremeters are derived to be $mu=15.13$, E(B-V)=0.001, and an age of 12.33 Gyr.}
  \label{fig:ValidationBestFit}
\end{figure}

For each trial we use \fidanka to measure the fiducial line and then optimize that fiducial line against the originating isochrone to esimate distance modulus, age, and color B-V excess. Figure \ref{fig:validationDist} is built from 1000 runs of these trials and show the mean and width of the percent error distributions for $\mu$, $E(B-V)$, and age. In general \fidanka is able to recover distance modulii effectively with age and E(B-V) reovery falling in line with other literature that does not cosider the CMD outside of the main sequence, main sequence turn off, sub giant, and red giant branches; specifically, it should be noted that \fidanka is not setup to model the horizontal branch.

\begin{figure}
  \centering
  \includegraphics[width=0.85\textwidth]{figures/ngc2808/DistributionOfErrors.pdf}
  \caption{Percent Error distribution for each of the three deriver parameters. Note that these values will be sensitive to the magnitude uncertainties of the photometry. Here we made use of the ACS artificial star tests to estimate the uncertanties. {\color{blue}Note that currently this is built with 100 runs, these take a long time so currently re running with 1000 runs.}}
  \label{fig:validationDist}
\end{figure}

\section{Isochrone Fitting}\label{sec:isoFit}
We fit pairs of isochrones to the HUGS data for NGC 2808 using \fidanka, as
described in \S \ref{sec:fidanka}. Two isochrones, one for Population A and one
for Population E are fit simultaneously. These isochrones are constrained to
have distance modulus, $\mu$, and color excess, E(B-V) which agree to within
0.5\% and an ages which agree to within 1\%. Moreover, we constrain the mixing
length, $\alpha_{ML}$, for any two isochrones in a set to be within 0.5 of one
and other. For every isochrone in the set of combination of which fulfilling
these constraints $\mu$, $E(B-V)$, Age$_{A}$, and $Age_{B}$ are optimized to
reduce the $\chi^{2}$ distance ($\chi^{2} = \sum\sqrt{\Delta \text{color}^{2} +
\Delta \text{mag} ^{2}}$) between the fiducial lines and the isochrones.
Because we fit fiducial lines directly, we do not need to consider the binary
population fraction, $f_{bin}$, as a free parameter.

The best fit isochrones are shown in Figure \ref{fig:BestFitResults} and
optimized parameters for these are presented in Table \ref{tab:BestFitResults}.
The initial guess for the age of these populations was locked to 12 Gyr
and the initial Extinction was locked to 0.5 mag. The initial guess for the
distance modulus was determined at run time using a dynamic time warping
algorithm to best align the morphologies of the fiducial line with the target
isochrone. This algorithm is explained in more detail in the \fidanka
documentation under the function called \texttt{guess\_mu}. We find helium mass
fractions that are consistent with those identified in past literature
\citep[e.g.][]{Milone2015}. Note that our helium mass fraction grid has a
spacing of 0.03 between grid points and we are therefore unable to resolve
between certain proposed helium mass fractions for the younger sequence (for
example between 0.37 and 0.39). We also note that the best fit mixing
length parameter which we derive for populations A and E do not agree within
their uncertainties. This is not surprising as the much high mean molecular mass
of population E --- when compared to population A, due to population E's larger
helium mass fraction --- will result in a steeper adiabatic temperature
gradient.

\begin{figure*}
  \centering
  \includegraphics[width=0.9\textwidth]{figures/ngc2808/BestFitResults.pdf}
  \caption{Best fit isochrone results for NGC 2808. The best fit population A
  and E models are shown as black lines. The following 50 best fit models are
  presented as gray lines. The solid black line is fit to population A, while
  the dashed black line is fit to population E.}
  \label{fig:BestFitResults}
\end{figure*}

\begin{table*}
  \centering
  \begin{tabular}{c | c c c c c c}
    \hline
    Population & Age & Distance Modulus & Extinction & Y & $\alpha_{ML}$ & $\chi^{2}_{\nu}$\\
    & [Gyr] & & [mag] & & &\\
    \hline
    \hline
    A & 12.996$^{+0.87}_{-0.64}$ & 15.021 & 0.54 & 0.24 & 2.050 & 0.021\\
    E & 13.061$^{+0.86}_{-0.69}$ & 15.007 & 0.537 & 0.39 & 1.600 & 0.033 \\
    \hline
  \end{tabular}
  \caption{Best fit parameters derived from fitting isochrones to the fiducual
  lines derived from the NCG 2808 photometry. The one sigma uncertainty
  reported on population age were determined from the 16th and 84th percentiles
  of the distribution of best fit isochrones ages.}
  \label{tab:BestFitResults}
\end{table*}


Past literature \citep[e.g. ][]{Milone2015, Milone2018} have found helium mass
fraction variation from the low red-most to blue-most populations of $\sim 0.12$.
Here we find a helium mass fraction variation of 0.15 which, given the spacing
of the helium grid we use {\em is consistent with these past results}.

\subsection{The Number of Populartions in NGC 2808}
In order to estimate the number of populations which ideally fit the NGC 2808
F275W-F814W photometry without over-fitting the data we make use of silhouette
analysis \citep[][and in a similar manner to how \citet{Valle2022} perform
their analysis of spectroscopic data]{ROUSSEEUW198753}. We find the average
silhouette score for all tagged clusters identified using BGMM in all magnitude
bins over the CMD using the standard python module \texttt{sklearn}. Figure
\ref{fig:clusterAn} shows the silhouette analysis results and that two
populations fit the photometry most ideally. This is in line with what our BGMM
model predicts for the majority of the CMD.

\begin{figure}
  \centering
  \includegraphics[width=0.85\textwidth]{figures/ngc2808/ClusterAnalysis.pdf}
  \caption{Silhouette analysis for NGC 2808 F275W-F814W photometry. The
  Silhouette scores are an average of score for each magnitude bin. Positive
  scores indicate that the clustering algorithm produced well distinguished
  clusters while negative scores indicate clusters which are not well
  distinguished.}
  \label{fig:clusterAn}
\end{figure}

While we make use a purely CMD based approach in this work, other literature
has made use of Chromosome Maps. These consist of implicitly verticalized
pseudo colors. In the chromosome map for NGC 2808 there may be evidence for
more than two populations; however, the process of transforming magnitude
measurements into chromosome space results in dramatically increased
uncertainties for each star. We find a mean fractional uncertainty for
chromosome parameters of $\approx1$ when starting with magnitude measurements
having a mean best-case (i.e. uncertainty assumed to only be due to Poisson
statistics) fractional uncertainty of $\approx 0.0005$ (Figure
\ref{fig:chromMapUn}). Because of how \fidanka operates, i.e. resampling a
probability distribution for each star in order to identify clusters, we are
unable to make statistically meaningful statements from the chromosome map


\begin{figure}
  \centering
  \includegraphics[width=0.85\textwidth]{figures/ngc2808/ChromosomeMapFractionalErrorDist.pdf}
  \caption{Fractional uncertainty distribution of the chromosome map parameter
  space for targets in NGC 2808. Note that fractional uncertainties of the
  magnitudes which went into the production of this chromosome map were on the
  order of 0.0005 (the blue vertical line in both plots marks this). Further,
  we assumed that there was no uncertainty on the placement of the red and blue
  fiducial lines. If there were uncertainty on those placements then the mean
  of this distribution would be higher.}
  \label{fig:chromMapUn}
\end{figure}

\subsection{ACS-HUGS Photometric Zero Point Offset}
The Hubble legacy archive photometry used in this work is calibrated to the
Vega magnitude system. However, we have found that the photometry has a
systematic offset of $\sim0.026$ magnitudes in the F814W band when
compared to the same stars in the ACS survey (Figure \ref{fig:offset}). The
exact cause of this offset is unknown, but it is likely due to a difference in
the photometric zero point between the two surveys. A full correction of this
offset would require a careful re-reduction of the HUGS photometry, which is
beyond the scope of this work. We instead recognize a 0.02 inherent uncertainty
in the inferred magnitude of any fit when comparing to the ACS survey. This
uncertainty is small when compared to the uncertainty in the
distance modulus and should not affect the conclusion of this
paper. 

\begin{figure*}
  \centering
  \includegraphics[width=0.90\textwidth]{figures/ngc2808/photometricOffset.pdf}
  \caption{(left) CMD showing the photometric offset between the ACS and HUGS
  data for NGC 2808. CMDs have been randomly subsampled and colored by point
  density for clarity. (right) Mean difference between the color of the HUGS
  and ACS fiducual lines at the same magnitude. Note that the ACS data is
  systematically bluer than the HUGS data.}
  \label{fig:offset}
\end{figure*}

The oberved photometric offset between ACS and HUGS reductions introduces a
systematic uncertainty when comparing parameters derived from isochrone fits
to ACS data vs those fit to HUGS data. Specifically, this offset introduces a
$\sim 2 Gyr$ uncertainty when comparing ages between ACS and HUGS. Moreover,
for two isochrone of the same age, only separated by helium mass fraction, a
shift of the main sequence turn off of is also expected. Figure \ref{fig:HeMO}
shows this shift. Note a change in the helium mass fraction of a model by 0.03
results in an approximate 0.08 magnitude shift to the main sequence turn off
location. This means that the mean 0.026 magnitude offset we find in between
ACS and HUGS data corresponds to an additional approximate 0.01 uncertainty
in the derived helium mass fraction when comparing between these two data sets. 

\begin{figure}
  \centering
  \includegraphics[width=0.85\textwidth]{figures/ngc2808/HeliumMeanOffset.pdf}
  \caption{Main sequence turn off magnitude offset from a gauge helium mass
  fraction (Y=0.30 chosen). All main sequence turn off locations are measured
  at 12.3 Gyr}
  \label{fig:HeMO}
\end{figure}

\section{Results}\label{sec:results}
Using \fidanka we fit pairs of Population A + E isochrones to the HUGS data for
NGC 2808. Each pair of isochrones is allowed to vary in distance modulus,
reddening, relative helium mass fraction (A/E), and age. Any population pairs which vary by more than 1\% in distance modulus or B-V color excess are rejected. The $\chi^{2}$
distribution for the isochrone pairs is shown in Figure {\color{red}[FIGURE]}.
The best fit isochrones are shown in Figure \ref{fig:BestFitResults} and optimized
parameters for these are presented in Table {\color{red}[TABLE]}.

{\color{red} Need to make the chi2 dist plot still. Have all these values but need to figure out best way to visualize it}

\begin{figure}
  \centering
  \includegraphics[width=0.45\textwidth]{figures/ngc2808/BestFitResults.pdf}
  \label{fig:BestFitResults}
  \caption{Best fit isochrone results for NGC 2808.}
\end{figure}

\begin{table*}
  \centering
  \begin{tabular}{c | c c c c c c}
    \hline
    population & age & distance modulus & extinction & Y & $\alpha_{ML}$ & $\chi^{2}_{\nu}$\\
    & [Gyr] & & mag & & &\\
    \hline
    \hline
    A & 12.3 & 14.91 & 0.54 & 0.24 & 1.901 & 0.014\\
    E & 14.3 & 14.96 & 0.54 & 0.39 & 1.750 & 0.017 \\
    \hline
  \end{tabular}
  \label{tab:BestFitResults}
  \caption{Best fit parameters derived from fitting isochrones to the fiducual lines derived from the NCG 2808 photometry.}
\end{table*}

{\color{blue} Currently are still seeing a discontinutiy in the isochrobne below the MSTO. This must be addressed before submission.}

\subsection{The Number of Populartions in NGC 2808}
\fidanka provides a somewhat straigtforward way to estimate the number of populations expected in a given magnitude bin given the observations. See Section \ref{sec:fidanka} for specific implimentaiton details. Here we preform an analysis of the number of populations seen in the NGC 2808 F814W-F274W vs F814W color-magnitude diagram. We find that for the majority of the main sequence and red giant branches BGMM prefers two populations; wherease, near the main sequnce turn off and on the majority of the subgiant branches BGMM prefers a single population model.

{\color{red}[FIGURE SHOWING BGMM population probability]}


\section{Conclusion}\label{sec:2808conclusion}
Here we have preformed the first chemically self-consistent modeling of the
Milky Way Globular Cluster NGC 2808. We find that, updated atmospheric boundary
conditions and opacity tables do not have a significant effect on the inferred
helium abundances of multiple populations. Specifically, we find that
population  has a helium mass fraction of 0.24, while population E has a helium
mass fraction of 0.39. Additionally, we find that the ages of these two populations 
agree within uncertainties. We only find evidence for two distinct stellar
populations, which is in agreement with recent work studying the number
of populations in NGC 2808 spectroscopic data.

We introduce a new software suite for globular cluster science,
\fidanka, which has been released under a permissive open source license.
\fidanka aims to provide a statistically robust set of tools for estimating the
parameters of multiple populations within globular clusters.



\subsection{47 Tuc \& NGC 6752}\label{sec:p5}
In addition to NGC 2808, Feiden has generated MARCS atmospheric models for the
clusters NGC 6752 and 47 Tuc. Using these surface boundary conditions along
with new opacity tables we querey from OPLIB, we will conduct the same,
self-consistent, modeling for these clusters as we do for NGC 2808.

NGC 6752 has been self-consistently modeled in past. \citet{Dotter2015} preform
self-consistent chemical modeling of this cluster using both ATLAS and PHOENIX
atmospheric boundary conditions computed from abundance of the photometrically
identified poulations A and C reported by \citet{Milone2013}.
\citeauthor{Dotter2015} additionally make use of OPAL high temperature opacity
tables and PHOENIX low temperature opacity tables.




\acknowledgments{
	This research has made use of NASA's astrophysical data system (ADS). We
	acknowledge the support of an NASA grant (No. 80NSSC18K0634). Additionally,
	we would like to thank James Colgan for his assistance with the OPLIB
	opacity tables. We would like to thank Aaron Dotter, and Elisabeth
	Newton for their assistance. Finally, we thank our colleagues and peers in
	for their continuing and appreciated support.
}


%% If you wish to include an acknowledgments section in your paper,
%% separate it off from the body of the text using the \acknowledgments
%% command.
\acknowledgments

% \bibliography{src/Bibliography/ms}


\end{document}
