\documentclass{GuariniThesis}
\usepackage{textgreek}
\usepackage[table]{xcolor}

% Define the details of your thesis
\thesisTitle{MODELS OF LOW MASS STARS AS PHYSICAL LABRATORIES}
\authorName{Emily M. Boudreaux}
\field{Physics and Astronomy}
\defenseDate{April 24th 2024}
\committeeChair{Brian C. Chaboyer}
\committeeMemberOne{Elisabeth R. Newton}
\committeeMemberTwo{Aaron Dotter}
\externalMember{Jamie Tayar}
\dedicationText{For Poppy, who taught me to build and supported as I did.}
\prefaceText{When I was in pre-school I told my father that I would be a professor of
astronomy. I remember the day, walking out of the Boyd school, having just
looked through a large picture book of the --- then --- nine planets in our
solar system. At the time I did not have a concept of what it meant to be an
astronomer. I did not understand what it meant to study space in any way other
than looking at images in a large cardboard book. In a very real way that day
was the most important in my life and it undoubtedly set a trajectory for me
which I have obsessively held onto for the past 22 years. I have held onto
that goal; however, my ability to, in some small part with this thesis, achieve
the goal that that 4 year old told their father has not primarily been a
function of my ability. Rather, many many people in my life have supported me,
helped me, loved me, let me lean on them, and been there for me when I need
them. I would not have been able to write this thesis without each and every
one of them and I am immeasurably grateful for the people in my life.

First of all I would like to speak to my mother and father, Karol and Don. You
are the most wonderful parents I could ever imagine. From the time when your
young child made the wild statement that they wanted to be an astronomer,
through many detentions and late nights at Westminster, through far too many
hours at Encore and comically long commutes in high school you have both been
the most supportive parents anyone ever could be. You have relentlessly
supported and loved me in ways that, I hope, have made me a better person, 
a better researcher, and a better child. I love you so much and I am so
grateful. Thank you.

Secondly, I would like to thank the educators, teachers, and professors who
I have interacted with in my life. I have been blessed and privileged to
have such a consistently good set of educational role models in my life. From
Mr. David Majewski in seventh and eighth grade science, to Ms. Leslie Zeigler 
in high school biology. My childhood teachers shaped my love of science and
helped me develop that goal that four year old me had into a firm sense of
what astronomy is. I want to thank Ms. Zeigler in particular for helping me 
grow my love of science and supporting that during a time in life when so 
many people get disillusioned with it. I truly believe that if not for her
I would not have remained interested in astronomy through high-school.

When I left high school for college I was worried that I would not be able to
build the same kind of relationships with educators there than I had had
until that point. I could not have been more wrong. I first met Dr. Brad Barlow
the day before his wedding, in October 2014, and from the moment we met he went
above and beyond any possible expectation to try to help me become an astronomer.
Brad brought me observing at SOAR when I was still a senior in High School, he
took me to AAS my first year in college, he co authored 3 papers with me and
took me around the world to various conferences. The friendship I developed with
Brad was the most important, by far, of my time in college and I immensely
grateful for that. More than the professional development opportunities which
Brad provided to me I want to thank him for his consistent willingness to
engage with me on random thoughts I had. I think chatting outside the Slane
center over lunch about various research ideas or topics I had just learned
about in class is what began, in earnest, to develop my ability to think about
science analytically and critically. That is the greatest gift that a professor
could ever give to their student. Thank you Brad.

Outside of education I have also been exceptionally privileged in the friends
whom I have met. I spent much of my childhood quite isolated, by choice.
However, working at Encore with Sarah, Annika, Kara, and Paddy was one of the
most important experiences of my life. Whereas Leslie and Brad helped shape who
I am as a scientist, my friends served to shape who I am as a person. There can
never be words strong enough to thank you all for that. I love you all very
deeply. 

Much as I was worried about my ability to form mentor-mentee connections in
college I was worried about my ability to make friends in grad school. This
concern could not have been more misplaced. I think there is likely no one in
grad school with a better set of friends and colleagues. My joint cohort mates
and house mates Aylin and Weishi are wonderful. We have lived together for 5
years now and I cannot imagine better people to live with. They are funny, fun,
engaging, and wonderful. I am so glad we will be able to keep living together
and I am so glad that we have. Aside from those I live with Steph and Rayna
have been incredible friends. Steph is always there for support or laughs, or
cooking and Rayna is always there for a good laugh or a sandwich. Thank you
all so much for being such wonderful friends and I cannot wait to see what you
accomplish with your careers.

Keighley Rockcliffe is one of the most amazing people I have ever met. I
remember being so scared of her when I first starting attending group meetings
my first year. She is a brilliant scientist, an incredibly empathetic and kind
person, and one of my best friends in the world. I could not have finished grad
school without her and I want to thank her from the bottom of my heart for
always being there for me. During grad school I went through a lot of change in
my life, I transitioned, I adopted a dog, I lost a dog, I struggled with
depression, and I struggled with anxiety. For every single struggle Keighley has
been there to provide a shoulder to cry on, a ear to listen, and an arm to
support. Thank you Keighley, you are a wonderful person, I love you so much and
I cannot wait to see the impact you make on the world.

Towards the end of my time in graduate school I was lucky enough to meet another
one of the most amazing people. I met Isabel in August of 2023 and even though
we have only known each other for 8 months, they have been some of the most
wonderful 8 months of my life. You have made the stress of the end of grad
school so much more bearable and you have made the stress of transitioning so
much more bearable. Shortly after we met I lost Jordy unexpectedly and
suddenly. That was likely the hardest day of my life and Isabel was there for
me in a way that is so much more than could ever have been expected. I could
not have continued in grad school after that without Isabel. I love you Isabel
and I am so excited for the coming years together.

I adopted Jordy in February of 2021 and she was the best Beagle, the best dog, and
the best companion someone could hope for. Jordy was the littlest beagle and the
stinkiest girl. She brought so much love and happiness to everyone life. I 
had planned to walk at graduation with her but unfortunately in October of 2023 
Jordy passed away from cancer. I miss you Jordy and I love you little girl.

Finally, this work was conducted under the supervision of Brian Chaboyer and I
would not have been able to complete this thesis without his continued support.
I would like to that Brian immensely for his continued support and for being a
wonderful teacher. I have not always been the fastest worker, spending far too
long getting distracted with small side projects. However, Brian has always
been both supportive and a provided a guiding hand to get those projects back
on track and turned into publishable material. Thank you Brian.
}
\abstractText{Low mass stars account for approximately 70 percent of the stellar populations
\citep{Conroy2012}; yet, due to their small sizes and cool temperatures they
account for only a small fraction of the galaxies luminosity function
\citep{Laughlin1997}. Moreover, due to the lack of labratory conditions
avalible to astronomy and astrophysics low mass stars can provide a rare
controlled enviroment for calibrations of numerical models. Consequently,
across multiple domains there has been signifigant interest in these key
astronomical objects. In this thesis I present three projects which have
further revealed properties of low mass stars and pushed the extent where
these low mass stars may be used as labratories. Firstly, I present chemically
self consistent stellar evolutionary models of the globular clusters NGC 2808.
Due to the age of this cluster, these models are dominated by low mass stars.
We find that chemical consistency between a stars structural and atmospheric
models makes only a trivial difference in model predictions. Secondly, I
present a detailed investigation into the Gaia M Dwarf Gap (the Jao Gap)
looking at how the Jao Gap's theoretical location is effected by high
temperature radiative opacity source and how the physics which drives the Jao
Gap's formation may also drive perturbations to stellar magnetic field
strength. A detailed understanding of the Jao Gap's underlying physics may
provide an important calibration point for M dwarf convectivive parameters. The
work presented in this thesis brings the field of astronomy closer to being
able to use those calibrations. Finally, this thesis investigates the relation
between the red giant branch bump (RGBB) in both NGC 2808 accross multiple
populations and across multiple opacity sources. Similar to the Jao Gap, the
RGBB provides a calibration point for convective parameters in stars on the red
giant branch. We find that the helium enriched population in NGC 2808 does not
show a detectable RGBB, validating previous theoretical studies of the RGBB
which did not consider multiple populations in their modeling. 

}

\newcommand\fidanka{\texttt{Fidanka} }
\newcommand\addcite{{\color{red}[CITATION HERE]}}

\begin{document}
\makePrelim

\part{Introduction}
\section{INTRODUCTION}
Over the last half of the 19th and first decade of the 20th centuries Lane,
Ritter, and Emden codified the earliest mathematical model of stellar
structure, the polytrope (Equation \ref{eqn:polytrope}), in \textit{Gaskugeln}
(Gas Balls) \citep{Emden1907}.

\begin{align}\label{eqn:polytrope}
	\frac{d}{d\xi}\left(\xi^{2}\frac{d\theta}{d\xi}\right) = -\xi^{2}\theta^{n}
\end{align}

Where $\xi$ and $\theta$ are dimensionless parameterizations of radius and
temperature respectively, and $n$ is known as the polytropic index. Despite this
early work, it wasn't until the late 1930s and early 1940s that the full set of
equations needed to describe the structure of a steady state,
radially-symmetric, star (known as the equations of stellar structure) began to
take shape as proton-proton chains and the Carbon-Nitrogen-Oxygen cycle were,
for the first time, seriously considered as energy generation mechanisms
\citep{Cowling1966}. Since then, and especially with the proliferation of
computers in astronomy, the equations of stellar structure have proven
themselves an incredibly predictive set of models.  

There are currently many stellar structure codes \citep[e.g.][]{Dotter2008,
Kovetz2009, Paxton2011} which integrate the equations of stellar structure ---
in addition to equations of state and lattices of nuclear reaction rates ---
over time to track the evolution of an individual star. The Dartmouth Stellar
Evolution Program (DSEP) \citep{Chaboyer2001, Bjork2006, Dotter2008} is one
such, well tested, stellar evolution program.

Here we propose to model low-mass stars in both the local solar neighborhood
and in globular clusters using DSEP. This work will primarily extend our
understanding of stellar physics in two areas: the effects of chemically
self-consistency on stellar models and time evolution of the core-convective
instabilities which ultimatly are belived to result in the observed paucity of
stars at a Gaia G magnitude of $\sim$10. [NEED CITATIONS IN THIS PARAGRAPH]

Low mass stars form an important component of the stellar population, with
stars less than [MASS HERE] making up more than 70\% of stars in the galaxy
[CITE]. Moreover, due to their long lives, low-mass stars provide essential
constraints on ages of various stellar populations [CITE]. In globular
clusters, where all stars are coeval to one of a limited number of populations,
low mass stars provide the vast majority of constraints when fitting ischrones [CITE].
Additionally, stars around the fully-convective transition mass show
age-dependent core-convective instabilities [CITE].

\subsection{Globular Clusters}

Globular clusters in the local universe are primarly composed of old and
consequently low-mass stars. For decades, prevaling thought had it globular
clusters were composed of a single stellar population born from a preisten
interstellar medium. This was supported by visibly tight main sequences and
clear main sequence turn offs in optical CMDs \citep[Figure
\ref{fig:M3CMD}][]{Sandage1953}. These early studies either did not handel or
had very large photometric uncertanties and therefore they were unable to
discriminate beteween CMD features with small separations,

\begin{figure}
	\centering
	\includegraphics[width=0.75\textwidth]{src/Figures/Gould53.png}
	\caption{$m_{pg}$ - $m_{pv}$ color-magnitude diagram for the globular cluster M3.}
	\label{fig:M3CMD}
\end{figure}

[SOMETHING ABOUT EARLY SPECTROSCOPIC INDICATIONS OF MPs]

With the presicion photometric measurements, degenerecies between noise and
intrinsic scatter were broken and it became clear that globular clusters are
almost universally composed of multiple stellar populations (MPs). [GRAB SOME
TEXT FROM THE NGC 2808 SECTION FOR HERE].

\subsection{Local Solar Neighborhood}
\citet{Jao2018} discovered a novel feature in the Gaia Gp-Rp color-magnitude
diagram. Around $M_{G}=10$ there is an approximatly 17\% decreas in stellar
density of the volume complete sample of stars \citeauthor{Jao2018} considered.
Subsequently, this has become known as either the Jao Gap, or Gaia M dwarf Gap.
Section \ref{} will go into more detail regarding the physics belived to
underpin this feature; however, in brief convective instabilities in the core
are belived to form for stars straddeling the fully convective transition mass.
These instabilities result in stars preferentially falling to either side of
the gap location.

Stellar modeling has been sucsessful in reproducing the Jap Gap and, with these
models, we have begun to constrain parameters which constrain gap location. For
example, it is now well documented that a stars metallicity can affect the gap
color by up to [HOW MUCH DID GREG FIND/CHECK FOR OTHER PAPERS ON THIS]. 

Initial testting which we have done using DSEP along with work by [PAPER] also
indicated the Jao Gap's color sensitivity to age. We observe that as models age
the Jao Gap moves [DIRECTION OF MOVMENT IN MAG AND COLOR SPACE].

The OPAL opacity tables in particular are very widely used by current
generation stellar evolution programs (in addition to current generation
stellar model and isochrone grids). However, they are no longer the most up
date elemental opacities. Moreover, the generation mechanism for these tables,
a webform, is no longer reliably online.  Consequently, it makes sense to
transition to more modern opacity tables with a more stable generation
mechanism.

Here we will present work transitioning DSEP from OPAL opacities to opacities
based on measurements from Los Alamos national Labs T-1 group
\citep[OPLIB][]{Colgan2016}. Moreover, we will present two projects which are
in large part reliant on these updated opacities. For the first project we
investigate the affects of chemically self consistent modeling of multiple
populations within the globular cluster NGC 2808, and for the second project we
present the effects of the OPLIB opacities on the location of the recently
discovered Gaia M-dwarf gap.

This paper is organized as follows. In Section \ref{sec:opac} we outline some
basic information about OPLIB opacities, how we query them, and how we modify
them to work with DSEP. In Section \ref{sec:2808} we discuss scientific
background of the first project along with the current work done towards its
goal. Finally, in Section \ref{sec:Jao} we present our findings on the effects
of OPLIB opacities on the location of the Gaia M-dwarf gap.







\part{Stellar Populations}
\chapter{Why Study Globular Clusters}
Globular clusters (GCs) are among the oldest observable objects in the universe
\citep{Pen11}. They are characterized by high densities with typical half-light
radii of $\le$10 pc \citep{Vanderburg2010}, and typical masses ranging from
$10^{4}$--$10^{5}$ M$_{\odot}$ \citep{Bro06} --- though some GCs are
significantly larger than these typical values \citep[e.g. $\omega$Cen,
][]{Richer1991}. GCs provide a unique way to probe stellar evolution
\citep{Bau03}, galaxy formation models \citep{Boy18,Kra05}, and dark matter
halo structure \citep{Hud18}.

The traditional view of Globular Clusters was, for a long time, that they
consisted of a single stellar population (SSP, in some publications this is
referred to as a Simple Stellar Population). This view was supported by
spectroscopically uniform heavy element abundances \citep{Carretta2010, Bastian2018} accross most clusters (M54 and $\omega$Cen are notable exceptions, see \citet{Marino2015} for further details), and the lack of evidence for multiple stellar populations
(MPs) in past color-magnitude diagrams of GCs \citep[i.e.][]{Sandage1953, Alcaino1975}. However, over the last 40 years non-trivial star-to-star light-element abundance variations have been observed \citep[i.e.][]{Smith1987} and, in
the last two decades, it has been definitively shown that most if not all Milky
Way GCs have MPs \citep{Gratton2004, Gratton2012, Piotto2015}. The lack of photometric evidence for MPs
can be attributed to the short color throw available to ground based
photometric surveys \citep{Milone2017}; specifically, lacking UV filters. While MPs are chemically distinct
from one another, that distinction is most prominent when observing with $U$
and $B$ filters \citep{Sbordone2011}.

The prevalence of multiple populations in GCs is so distinct that the proposed
definitions for what constitutes a globular cluster now often center the
existence of MPs. Whereas, people have have often tried to categorized objects
as GCs through relations between half-light radius, density, and surface
brightness profile, in fact many objects which are generally thought of as GCs
don't cleanly fit into these cuts \citep{Peebles1968, Brown1991, Brown1995, Bekki2002}.
Consequently, \citet{Carretta2010} proposed a definition of GC based on
observed chemical inhomogeneities in their stellar populations. The modern
understanding of GCs then is not simply one of a dense cluster of stars which
may have chemical inhomogeneities and multiple populations; rather, it is one
where those chemical inhomogeneities and multiple populations themselves are
the defining element of a GC.

All Milky Way globular clusters older than 2 Gyr studied in detail show
populations enriched in He, N, and Na while also being deplete in O and C
\citep{Piotto2015,Bastian2018}. These light element abundance patterns also are
not strongly correlated with variations in heavy element abundance, resulting
in spectroscopically uniform Fe abundances between populations. Further,
high-resolution spectral studies reveal anti-correlations between N-C
abundances, Na-O abundances, and potentially Al-Mg \citep{Sneden1992,
Gratton2012}. Typical stellar fusion reactions can deplete core oxygen;
however, the observed abundances of Na, Al, and Mg cannot be explained by the
likes of the CNO cycle \citep{Prantzos2007}.

Formation channels for these multiple populations remain a point of debate
among astronomers. Most proposed formation channels consist of some older,
more massive, population of stars polluting the pristine cluster media before a
second population forms, now enriched in heavier elements which they themselves could
not have generated \citep[for a detailed review see ][]{Gratton2012}. The four
primary candidates for these polluters are asymptotic giant branch stars
\citep[AGBs,][]{Ventura2001,DErcole2010}, fast rotating massive stars
\citep[FRMSs,][]{Decressin2007}, super massive stars
\citep[SMSs,][]{Denissenkov2014}, and massive interacting binaries
\citep[MIBs,][]{deMink2009, Bastian2018}. 

Hot hydrogen burning (proton capture), material transport to the surface, and
material ejection into the intra-cluster media are features of each of these
models and consequently they can all be made to {\it qualitatively} agree with
the observed elemental abundances. However, none of the standard models can
currently account for all specific abundances \citep{Gratton2012}. AGB and FRMS
models are the most promising; however, both models have difficulty reproducing
severe O depletion \citep{Ventura2009,Decressin2007}. Moreover, AGB and FRMS
models require significant mass loss ($\sim 90\%$) between cluster formation
and the current epoch --- implying that a significant fraction of halo stars
formed in GCs \citep{Renzini2008,DErcole2008,Bastian2015}.

In addition to the light-element anti-correlations observed it is also known
that younger populations are significantly enhanced in Helium
\citep{Piotto2007, Piotto2015, Latour2019}. Depending on the cluster, Helium
mass fractions as high as $Y=0.4$ have been inferred \citep[e.g][]{Milone2015}.
However, due to the relatively high and tight temperature range of partial
ionization for He it cannot be observed in globular clusters; consequently, the
evidence for enhanced He in GCs originates from comparison of theoretical
stellar isochrones to the observed color-magnitude-diagrams of globular
clusters. Therefore, a careful handling of chemistry is essential when modeling
with the aim of discriminating between MPs; yet, only a very limited number of
GCs have yet been studied with chemically self-consistent (structure and
atmosphere) isochrones \citep[e.g.][NGC 6752]{Dotter2015}. 

\chapter{Modeling Globular Clusters}
Due to the importance of globular clusters accross domains of astronomy 
there are multiple ways to model them. Some of these approaches, for work
which is primarily focused on teh dynamics of clusters and galxies, treat
globular clusters as dynamical systems and study aspects such as evaporation rates
and dynamical cooling timescales. 

An analytic framework for understanding cluster evaporation was proposed by
both \citet{Amb38} and \citet{Spi40}, who suggested an expression for a
dimensionless evaporation rate $\xi$.
\begin{equation}
    \xi \equiv -\frac{t_{rh}}{M}\frac{dM}{dt}
\end{equation}
where $M$ is the total cluster mass and the relaxation time $t_{rh}$ is defined as:
\begin{equation}
    t_{rh} = \frac{0.14 N}{\ln\Lambda}\sqrt{\frac{r^{3}_{hm}}{GM}}
\end{equation}
Where $r_{hm}$ is the half mass radius of the cluster. Additionally, $\Lambda
\equiv 0.4N$ and N is the total number of stars in the cluster. Due to the
complexity of cluster mass evaporation, the initial attempts to analytically
constrain the value of $\xi$ have been superseded by numerical calculations. As
a result, it has since been shown that both external tidal fields \citep{Mad17,
Bau03} and stellar evolution \citep{Bau03} play important roles in determining
a cluster's mass loss rate.

Understanding the evaporation and expansion history of GCs can shed light on
both the galactic potentials in which the cluster lives, and where the cluster
formed in the galactic potential \citep{Ren17}. \citet{Wil03} and \citet{Gon05}
presented early evidence via numeric studies that the primordial binary
fraction of a cluster may influence a cluster's core radius expansion rate.
Their results show that a direct proportionality between the binary fraction
and core radius expansion rates may exist. Further, \citet{Lan07} provided
empirical evidence, through a study of blue straggler stars (BSS) in M5, that
collisional binaries may increase evaporation rates. However, there has yet to
be established a firm quantitative relationship between the primordial binary
fraction of a cluster and its evolutionary rates


\section{Problems Modeling Globular Clusters}
There are some outstanding problems in models of Globular Clusters, primarily
these fall into two categories. One, formation channels do not currently capture
the observed abundance variations between multiple populations in clusters. This
is espcecially pronounced when considering the timescale involved between early 
and later population formation. 



\chapter{Multiple Populations in NGC 2808}
\section{INTRODUCTION}
Over the last half of the 19th and first decade of the 20th centuries Lane,
Ritter, and Emden codified the earliest mathematical model of stellar
structure, the polytrope (Equation \ref{eqn:polytrope}), in \textit{Gaskugeln}
(Gas Balls) \citep{Emden1907}.

\begin{align}\label{eqn:polytrope}
	\frac{d}{d\xi}\left(\xi^{2}\frac{d\theta}{d\xi}\right) = -\xi^{2}\theta^{n}
\end{align}

Where $\xi$ and $\theta$ are dimensionless parameterizations of radius and
temperature respectively, and $n$ is known as the polytropic index. Despite this
early work, it wasn't until the late 1930s and early 1940s that the full set of
equations needed to describe the structure of a steady state,
radially-symmetric, star (known as the equations of stellar structure) began to
take shape as proton-proton chains and the Carbon-Nitrogen-Oxygen cycle were,
for the first time, seriously considered as energy generation mechanisms
\citep{Cowling1966}. Since then, and especially with the proliferation of
computers in astronomy, the equations of stellar structure have proven
themselves an incredibly predictive set of models.  

There are currently many stellar structure codes \citep[e.g.][]{Dotter2008,
Kovetz2009, Paxton2011} which integrate the equations of stellar structure ---
in addition to equations of state and lattices of nuclear reaction rates ---
over time to track the evolution of an individual star. The Dartmouth Stellar
Evolution Program (DSEP) \citep{Chaboyer2001, Bjork2006, Dotter2008} is one
such, well tested, stellar evolution program.

Here we propose to model low-mass stars in both the local solar neighborhood
and in globular clusters using DSEP. This work will primarily extend our
understanding of stellar physics in two areas: the effects of chemically
self-consistency on stellar models and time evolution of the core-convective
instabilities which ultimatly are belived to result in the observed paucity of
stars at a Gaia G magnitude of $\sim$10. [NEED CITATIONS IN THIS PARAGRAPH]

Low mass stars form an important component of the stellar population, with
stars less than [MASS HERE] making up more than 70\% of stars in the galaxy
[CITE]. Moreover, due to their long lives, low-mass stars provide essential
constraints on ages of various stellar populations [CITE]. In globular
clusters, where all stars are coeval to one of a limited number of populations,
low mass stars provide the vast majority of constraints when fitting ischrones [CITE].
Additionally, stars around the fully-convective transition mass show
age-dependent core-convective instabilities [CITE].

\subsection{Globular Clusters}

Globular clusters in the local universe are primarly composed of old and
consequently low-mass stars. For decades, prevaling thought had it globular
clusters were composed of a single stellar population born from a preisten
interstellar medium. This was supported by visibly tight main sequences and
clear main sequence turn offs in optical CMDs \citep[Figure
\ref{fig:M3CMD}][]{Sandage1953}. These early studies either did not handel or
had very large photometric uncertanties and therefore they were unable to
discriminate beteween CMD features with small separations,

\begin{figure}
	\centering
	\includegraphics[width=0.75\textwidth]{src/Figures/Gould53.png}
	\caption{$m_{pg}$ - $m_{pv}$ color-magnitude diagram for the globular cluster M3.}
	\label{fig:M3CMD}
\end{figure}

[SOMETHING ABOUT EARLY SPECTROSCOPIC INDICATIONS OF MPs]

With the presicion photometric measurements, degenerecies between noise and
intrinsic scatter were broken and it became clear that globular clusters are
almost universally composed of multiple stellar populations (MPs). [GRAB SOME
TEXT FROM THE NGC 2808 SECTION FOR HERE].

\subsection{Local Solar Neighborhood}
\citet{Jao2018} discovered a novel feature in the Gaia Gp-Rp color-magnitude
diagram. Around $M_{G}=10$ there is an approximatly 17\% decreas in stellar
density of the volume complete sample of stars \citeauthor{Jao2018} considered.
Subsequently, this has become known as either the Jao Gap, or Gaia M dwarf Gap.
Section \ref{} will go into more detail regarding the physics belived to
underpin this feature; however, in brief convective instabilities in the core
are belived to form for stars straddeling the fully convective transition mass.
These instabilities result in stars preferentially falling to either side of
the gap location.

Stellar modeling has been sucsessful in reproducing the Jap Gap and, with these
models, we have begun to constrain parameters which constrain gap location. For
example, it is now well documented that a stars metallicity can affect the gap
color by up to [HOW MUCH DID GREG FIND/CHECK FOR OTHER PAPERS ON THIS]. 

Initial testting which we have done using DSEP along with work by [PAPER] also
indicated the Jao Gap's color sensitivity to age. We observe that as models age
the Jao Gap moves [DIRECTION OF MOVMENT IN MAG AND COLOR SPACE].

The OPAL opacity tables in particular are very widely used by current
generation stellar evolution programs (in addition to current generation
stellar model and isochrone grids). However, they are no longer the most up
date elemental opacities. Moreover, the generation mechanism for these tables,
a webform, is no longer reliably online.  Consequently, it makes sense to
transition to more modern opacity tables with a more stable generation
mechanism.

Here we will present work transitioning DSEP from OPAL opacities to opacities
based on measurements from Los Alamos national Labs T-1 group
\citep[OPLIB][]{Colgan2016}. Moreover, we will present two projects which are
in large part reliant on these updated opacities. For the first project we
investigate the affects of chemically self consistent modeling of multiple
populations within the globular cluster NGC 2808, and for the second project we
present the effects of the OPLIB opacities on the location of the recently
discovered Gaia M-dwarf gap.

This paper is organized as follows. In Section \ref{sec:opac} we outline some
basic information about OPLIB opacities, how we query them, and how we modify
them to work with DSEP. In Section \ref{sec:2808} we discuss scientific
background of the first project along with the current work done towards its
goal. Finally, in Section \ref{sec:Jao} we present our findings on the effects
of OPLIB opacities on the location of the Gaia M-dwarf gap.






\input{chapters/ngc2808/subsections/observations}
\section{Modeling}\label{sec:modeling}
One of the most pressing questions related to this work is whether or not the
increased star-to-star variability in the activity metric and the Jao Gap,
which are coincident in magnitude, are driven by the same underlying mechanism.
The challenge when addressing this question arises from current computational
limitations. Specifically, the kinds of three dimensional
magneto-hydrodynamical simulations --- which would be needed to derive the
effects of convective kissing instabilities on the magnetic field of the star
--- are infeasible to run over gigayear timescales while maintaining thermal
timescale resolutions needed to resolve periodic mixing events.

In order to address this and answer the specific question of \textit{could
kissing instabilities result in increased star-to-star variability of the
magnetic field}, we adopt a very simple toy model. Kissing instabilities result
in transient radiative zone separating the core of a star (convective) from its
envelope (convective). When this radiative zone breaks down two important
things happen: one, the entire star becomes mechanically coupled, and two,
convective currents can now move over the entire radius of the star.
\citet{Jao2023} propose that this mechanical coupling may allow the stars core
to act as an angular momentum sink thus accelerating a stars spin down and
resulting in anomalously low H$\alpha$ emission. 

Regardless of the exact mechanism by which the magnetic field may be effected,
it it reasonable to expect that both the mechanical coupling and the change to
the scale of convective currents will have some effect on the stars magnetic
field. On a microscopic scale both of these will change how packets of charge
within a star move and may serve to disrupt a stable dynamo. Therefore, in the
model we present here we make only one primary assumption: \textit{every mixing
event may modify the stars magnetic field by some amount}. Within our model
this assumption manifests as a random linear perturbation applied to some base
magnetic field at every mixing event. The strength of this perturbation is 
sampled from a normal distribution with some standard deviation, $\sigma_{B}$.

Synthetic stars are sampled from a grid of stellar models evolved using the
Dartmouth Stellar Evolution Program (DSEP). Each stellar model was evolved
using a high temporal resolution (timesteps no larger than 10,000 years
{\color{red} Check this}) and typical numerical tolerances of one part in
$10^5$. Each model was based on a GS98 \citep{Grevesse1998} solar
composition with a mass range from 0.3 M$_{\odot}$ to 0.4 M$_{\odot}$. Finally,
models adopt OPLIB high temperature radiative opacities, Ferguson 2004 low
temperature radiative opacities, and include both atomic diffusion and
gravitational settling. A Kippenhan-Iben diagram showing the structural
evolution of a model within the gap is shown in Figure \ref{fig:kippenhan}.

\begin{figure*}
  \centering
  \includegraphics[width=0.9\textwidth]{figures/jaoMagActivity/Kippenhan_clamped.pdf}
  \caption{Kippenhan-Iben diagram for a 0.345 solar mass star. Note the
  periodic mixing events (where the plotted curves peak).}
  \label{fig:kippenhan}
\end{figure*}

Each synthetic star is assigned some base magnetic activity ($B_{0} \sim
\mathcal{N}(1, \sigma_{B})$) and then the number of mixing events before some age $t$
are counted based on local maxima in the core temperature. The toy magnetic
activity at age $t$ for the model is given in Equation \ref{eqn:activity}. An
example of the magnetic evolution resulting from this model is given in Figure
\ref{fig:simpleB}. Fundamentally, this model presents magnetic
activity variation due to mixing events as a random walk and therefore results will
increasingly divergence over time.

\begin{align}\label{eqn:activity}
  B(t) = B_{0} + \sum_{i}B_{i} \sim \mathcal{N}(1, \sigma_{B}) 
\end{align}

\begin{figure}
  \centering
  \includegraphics[width=0.45\textwidth]{figures/jaoMagActivity/simpleBEvolution.pdf}
  \caption{Example of the toy model presented here resulting in increased
  divergence between stars magnetic fields. The shaded region represents the
  maximum spread in the two point correlation function at each age.}
  \label{fig:simpleB}
\end{figure}

Applying the same analysis to these models as was done to the observations as
described in Section {\color{red} X.X} we find that this simple model results
in a qualitatively similar trend in the standard deviation vs. Magnitude graph
(Figure \ref{fig:model}). In order to reproduce the approximately 50 percent
change to the spread of the activity metric observed in the combined dataset in
section \ref{sec:results} a distribution with a standard deviation of 0.1 is required when sampling the change in the magnetic activity metric at each mixing event. This corresponds to 68\% of mixing events modifying the activity strength by 10 percent or less. The interpretation here is important, what
this qualitative similarity demonstrates is that it may be reasonable to expect
kissing instabilities to result in the observed increased star-to-star
variation. Importantly, we are not able to claim that kissing instabilities
\textit{do} lead to these increased variations, only that they reasonably
could. Further modeling, observational, and theoretical efforts will be needed
to more definitively answer this question.

\begin{figure}
  \centering
  \includegraphics[width=0.45\textwidth]{figures/jaoMagActivity/SpreadModel.pdf}
  \caption{Toy model results showing a qualitatively similar discontinuity in the star-to-star magnetic activity variability.}
  \label{fig:model}
\end{figure}

\subsection{Limitations}
The model presented in this paper is very limited and it is important to keep
those limitations in mind when interpreting the results presented here. Some of
the main challenges which should be leveled at this model are the assumption
that the magnetic field will be altered by some small random perturbation at
every mixing event. This assumption was informed by the large number of free
parameters available to a physical star during the establishment of a large scale 
magnetic field and the associated likely stochastic nature of that process.
However, it is similarly believable that the magnetic field will tend to alter in
a uniform manner at each mixing event. For example, since differential rotation
is generally proportional to the temperature gradient within a star and activity is
strongly coupled to differential rotation then it may be that as the radiative zone reforms over thermal timescales the homogenization of angular momentum throughout the star results in overall lower amounts of differential rotation each after mixing event than would otherwise be present.

Moreover, this model does not consider how other degenerate sources of magnetic evolution such as stellar spin down, relaxation, or coronal heating may effect star-to-star variability. These could conceivably lead to a similar increase in star-to-star variability which is coincident with the Jao Gap magnitude as the switch from fully to partially convective may effect efficiency of these process.

Additionally, there are challenges with this toy model that originate from the stellar evolutionary model. Observations of the Jao Gap show that the feature is not perpendicular to the magnitude axis; rather, it is inversely proportional to the color. No models of the Jao Gap published at the time of writing capture this color dependency and \textit{what causes this color dependency} remains one of the most pressing questions relating to the underlying physics. This non captured physics is one potential explanation for why the magnitude where our model predicts the increase in variability is not in agreement with where the variability jump exists in the data.

Finally, we have not considered detailed descriptions of the dynamos of stars. The magnetohydrodynamical modeling which would be required to model the evolution of the magnetic field of these stars at thermal timescale resolutions over gigayears is currently beyond the ability of practical computing. Therefore future work should focus on limited modeling which may inform the evolution of the magnetic field directly around the time of a mixing event.

\section{Chemical Consistency}\label{sec:const}
There are three primary areas in which must the stellar models must be made chemically
consistent: the atmospheric boundary conditions, the opacities, and interior
abundances. The interior abundances are relatively easily handled by adjusting
parameters within our stellar evolutionary code. However, the other two areas are
more complicated to bring into consistency. Atmospheric boundary conditions and
opacities must both be calculated with a consistent set of chemical abundances
outside of the stellar evolution code. For evolution we use the Dartmouth Stellar Evolution Program (DSEP) \citep{Dotter2008}, a well tested 1D stellar evolution code which has a particular focus on modelling low mass stars ($\le 2$ M$_{\odot}$)

\subsection{Atmospheric Boundary Conditions}\label{sec:atm}
Certain assumptions, primarily that the radiation field is at equilibirum and radiative transport is diffusive \citep{Salaris2005}, made in stellar structure
codes, such as DSEP, are valid when the optical depth of a star is small.
However, in the atmospheres of stars, the number density of particles drops low
enough and the optical depth consequently becomes large enough that these
assumptions break down, and separate, more physically motivated, plasma modeling code is required.
Generally structure code will use tabulated atmospheric boundary conditions
generated by these specialized codes ATLAS9 \citep{Kurucz1993}, PHEONIX \citep{Husser2013}, MARCS \citep{Gustafsson2008}, and MPS-ATLAS \citep{Kostogryz2023}. Often, as the boundary conditions are both expensive to compute
and not the speciality of stellar structure researchers, the boundary
conditions are not updated as as light-element interior abundance varies. 

One key element when chemically consistently modeling NGC 2808 modeling is the
incorporation of new atmospheric models with the same elemental abundances as
the structure code. We use atmospheres generated from the \texttt{MARCS} grid
of model atmospheres \citep{Plez2008}. \texttt{MARCS} provides one-dimensional,
hydrostatic, plane-parallel and spherical LTE atmospheric models
\citep{Gustafsson2008}. Model atmospheres are made to match the
spectroscopically measured elemental abundances of populations A and E.
Moreover, for each populations, atmospheres with  various helium mass fractions
are generated. These range from Y=0.24 to Y=0.36 in steps of 0.02. All
atmospehric models are computed to an optical depth of $\tau = 100$ where their
temperature and pressures serves as boundary conditions for the strudcure code.
A comparison of the pressure and temperature throughout the atmospheres of the
two populations with helium abundances representative of literature values is
shown
in Figure \ref{fig:AEAtmComp}.

\begin{figure}
	\centering
	\includegraphics[width=0.85\textwidth]{figures/ngc2808/notebookFigures/AtmosphereComparison.pdf}
	\label{fig:AEAtmComp}
	\caption{Comparison of the MARCS model atmospheres generated for the two
	extreme populations of NGC 2808. These lines shows population A and E with
	the same Helium abundance; though, we fit a grid of models over various
	helumn abundances. Dashed lines show the temperature of the boundary
	condition while sold lines show the pressure.}
\end{figure}


\subsection{Opacities}\label{sec:opac}
In addition to the atmospheric boundary conditions, both the high and low
temperature opacities used by DSEP must be made chemically consistent. Here we
use OPLIB high temperature opacity tables \citep{Colgan2016} retrieved using
the TOPS web-interface. Low temperature opacity tables are retrieved from the
Aesopus 2.0 web-interface \citep{Marigo2009, Marigo2022}. Ideally, these
opacities would be the same used in the atmospheric models. However, the
opacities used in the MARCS models are not publicly available. As such, we use
the opacities provided by the TOPS and Aesopus 2.0 web-interfaces.

\section{fidanka}\label{sec:fidanka}
When fitting isochrones to the data we have four main criteria for any method

\begin{itemize}
	\item The method must be robust enough to work along the entire main sequence, turn off, and much of the subgiant and red giant branchs.
	\item Any method should consider photometric uncertainty in the fitting process.
	\item The method should be model independent, weighting any n number of populations equally.
	\item The method should be automated and require minimal intervention from the user.
\end{itemize}


We do not believe that any currently available software is a match for
our use case. Therefore, we elect to develop our own software suite, \fidanka.
\fidanka is a python package designed to automate much of the process of
measuring fiducial lines in CMDs, adhering to the four criteria we lay out
above. Primary features of \fidanka may be separated into three
categories: fiducial line measurement, stellar population synthesise, and
isochrone optimization/fitting. Additionally, there are utility functions which
are detailed in the \fidanka documentation.

\subsection{Fiducial Line Measurement}
\fidanka takes a iterative approach to measuring fiducial lines, the first step
of which is to make a ``guess'' as to the fiducial line. . This initial guess
is calculated by splitting the CMD into magnitude bins, with uniform numbers of
stars per bin (so that bins are cover a small magnitude range over densely
populated regions of the CMD while covering a much larger magnitude range in
sparsely populated regions of the CMD, such as the RGB). A unimodal Gaussian
distribution is then fit to the color distribution of each bin, and the
resulting mean color is used as the initial fiducial line guess. This rough
fiducial line will approximately trace the area of highest density. The initial
guess will be used to verticalze the CMD so that further algorithms can work in
1-D magnitude bins without worrying about weighting issues caused by varying
projections of the evolutionary sequence onto the magnitude axis.
Verticalization is preformed taking the difference between the guess fiducial
line and the color of each star in the CMD.

If \fidanka were to simply apply the same algorithm to the verticalized CMD
then the resulting fiducial line would likely be a re-extraction of the initial
fiducial line guess. To avoid this, we take a more robust, number density based
approach, which considers the distribution of stars in both color and magnitude
space simultaneously. For each star in the CMD we first using a
\texttt{introselect} partitioning algorithm to select the 50 nearest stars in F814W vs. F275W-F814W space.
To account for the case where the star is at an extreme edge of the CMD, those
50 stars include the star itself (such that we really select 49 stars + 1). We
use \texttt{qhull}\footnote{https://www.qhull.com}\citep{Barber1996, } to
calculate the convex hull of those 50 points. The number density at each star
then is defined as $50/A_{hull}$, where $A_{hull}$ is the area of the convex
hull. Because we use a fixed number of points per star, and a partitioning
algorithm as opposed to a sorting algorithm, this method scales like
$\mathcal{O}(n)$, where n is the number of stars in the CMD. This method also
intrinsically weights the density of of each star equally as the counting
statistics per bin are uniform. We are left with a CMD where each star
has a defined number density (Figure \ref{fig:densityMapDemo}).

\begin{figure*}
	\centering
	\includegraphics[width=0.9\textwidth]{figures/ngc2808/notebookFigures/DensityMapDemo.png}
	\label{fig:densityMapDemo}
	\caption{Density map demo showing density estimate over different parts of
	the evolutionary sequence. The left panel shows the density map over the
	entire evolutionary sequence, while the middle panel shows the density map
	over the main sequence and the right most panel shows the density map over
	the RGB. Figures in the top row are the raw CMD, while figures in the
	bottom row are colored by the density map.}
\end{figure*}

\fidanka can now exploit this density map to fit a better fiducial line to the
data, as the density map is far more robust to outliers. There are multiple
algorithms we implement to fit the fiducial line to the color-density profile
in each magnitude bin (Figure \ref{fig:densityBinsDemo}); they are explained in more detail in the \fidanka
documentation. However, of most relevance here is the Bayesian Gaussian Mixture
Modeling (BGMM) method. BGMM is a clustering algorithm which, for some fixed
number of n-dimensional Gaussian distributions, $K$, determines the mean, covariance, and
mixing probability (somewhat analogous to amplitude) of each $k^{th}$
distribution, such that the local lower bound of the evidence of each star
belonging strongly to a single distribution is maximized. 

\begin{figure}
	\centering
	\includegraphics[width=0.85\textwidth]{figures/ngc2808/notebookFigures/DensityBinsDemo.png}
	\label{fig:densityBinsDemo}
	\caption{CMD where points are colored by density. Lines show the
	density-color profile in each magnitude bin. In this figure adaptive
	binning targeted 1000 stars per bin}
\end{figure}

Maximization is preformed using the Dirichlet process, which is a
non-parametric Bayesian method of determining the number of Gaussian distributions, $K$,
which best fit the data \citep{Ferguson1973, scikit-learn}. Use of the Dirichlet process
allows for dynamic variation in the number of inferred populations from
magnitude bin to magnitude bin. Specifically, populations are clearly visually
separated from the lower main sequence through the turn off; however, at the
turn off and throughout much of the subgiant branch, the two visible
populations overlap due to their extremely similar ages \citep[i.e.][]{Jordan2002}. The Dirichlet process allows for the BGMM method to infer a single
population in these regions, while inferring two populations in regions where
they are clearly separated. More generally, the use of the Dirichlet process
removes the need for a prior on the exact number of populations to fit. Rather,
the user specifies a upper bound on the number of populations within the
cluster. An example bin (F814W = 20.6) is shown in Figure \ref{fig:BGMMDist}.

\begin{figure*}
	\centering
	\includegraphics[width=0.9\textwidth]{figures/ngc2808/BGMMMixingBin.pdf}
	\caption{Example of BGMM fit to a magnitude bin. The grey line shows the
	underlying color-density profile, while the black dashed-line shows the
	joint distribution of each BGMM component. The solid black lines show the
	two selected components.}
	\label{fig:BGMMDist}
\end{figure*}

\fidanka's BGMM method first breaks down the verticalized CMD into magnitude
bins with uniform numbers of stars per bin (here we adopt 250). Any stars left
over are placed into the final bin. For each bin a BGMM model with a maximum of
5 populations is fit to the color density profile. The number of populations is
then inferred from the weighting parameter (the mixing probability) of each
population. If the weighting parameter of any $k^{th}$ components less than
{\color{blue}0.05}, then that component is considered to be spurious and
removed. Additionally, if the number of populations in the bin above and the
bin below are the same, then the number of populations in the current bin is
forced to be the same as the number of populations in the bin above. Finally,
the initial guess fiducial line is added back to the BGMM inferred line. Figure
\ref{fig:vertFit} shows the resulting fiducial line(s) in each magnitude bin
for both a verticalized CMD and a non verticalized CMD.

\begin{figure}
	\centering
	\includegraphics[width=0.85\textwidth]{figures/ngc2808/vertFit.png}
	\caption{CMD where points are colored by density. Line trace the infered 
	fiducial line(s) in each magnitude bin.}
	\label{fig:vertFit}
\end{figure}

This method of fiducial line extraction effectively discriminated between
multiple populations long the main sequence and RGB of a cluster, while
simultaneously allowing for the presence of a single population along the MSTO
and subgiant branch. 

We can adapt this density map based BGMM method to consider photometric
uncertainties by adopting a simple Monte Carlo approach. Instead of measuring
the fiducial line(s) a single time, \fidanka can measure the fiducial line(s)
many times, resampling the data with replacement each time. For each resampling
\fidanka adds a random offset to each filter based on the photometric
uncertainties of each star. From these $n$ measurments the mean fiducial line
for each sequence can be identified along with upper and lower bound confidence
intervals in each magnitude bin.

\subsection{Stellar Population Synthesis}
In addition to measuring fiducial lines, \fidanka also includes a stellar
population synthesise module. This module is used to generate synthetic CMDs
from a given set of isochrones. This is of primary importance for binary
population modelling. The module is also used to generate synthetic CMDs for
the purpose of testing the fiducial line extraction algorithms against priors.

\fidanka uses MIST formatted isochrones \citep{Dotter2016} as input along
with distance modulus, B-V color excess, binary mass fraction, and bolometric
corrections. An arbitrarily large number of isochrones may be used to define an
arbitrary number of populations. Synthetic stars are samples from each
isochrone based on a definable probability (for example it is believed that
$\sim90\%$ of stars in globular clusters are younger population
\citep[e.g.][]{Suntzeff1996, Carretta2013}). Based on the metallicity, $\mu$, and E(B-V) of each
isochrone, bolometric corrections are taken from bolometric correction tables.
Where bolometric correction tables do not include exact metallicities or
extinctions a linear interpolation is preformed between the two bounding
values. 

\subsection{Isochrone Optimization}
The optimization routines in \fidanka will find the best fit distance modulus,
B-V color excess, and binary number fraction for a given set of isochrones. If
a single isochrone is provided then the optimization is done by minimizing the
$\chi^2$ of the perpendicular distances between an isochrone and a fiducial
line. If multiple isochrones are provided then those isochrones are first used
to run stellar population synthesis and generate a synthetic CMD. The
optimization is then done by minimizing the $\chi^2$ of both the perpendicular
distances between and widths of the observed fiducial line and the fiducial
line of the synthetic CMD.


\subsection{Fidanka Testing}
In order to validate fidanka we have run an series of injection recovery tests
using \fidanka's population synthesis routines to build various synthetic
populations and \fidanka's fiducial measurement routines to recover these
populations. Each population was generated using the initial mass function
given in \citep{Milone2012} for the redmost population ($\alpha=-1.2$).
Further, every population was given a binary population fraction of 10\%,
distance uniformly sampled between 5000pc and 15000pc, and a B-V color excess
uniformly sampled between 0 and 0.1. Finally, each synthetic population was
generated using a fixed age  uniformlly sampled between 7 Gyr and 14 Gyr. An
example synthetic population along with its associated best fit isochrone are
shown in Figure \ref{fig:ValidationBestFit}.

\begin{figure}
  \centering
  \includegraphics[width=0.85\textwidth]{figures/ngc2808/ExtractedIsoFit.pdf}
  \caption{Synthetic population generated by fidanka at 10000pc with E(B-V) = 0, and an age of 12 Gyr along with the best fitting isochrone. The best fit paremeters are derived to be $mu=15.13$, E(B-V)=0.001, and an age of 12.33 Gyr.}
  \label{fig:ValidationBestFit}
\end{figure}

For each trial we use \fidanka to measure the fiducial line and then optimize that fiducial line against the originating isochrone to esimate distance modulus, age, and color B-V excess. Figure \ref{fig:validationDist} is built from 1000 runs of these trials and show the mean and width of the percent error distributions for $\mu$, $E(B-V)$, and age. In general \fidanka is able to recover distance modulii effectively with age and E(B-V) reovery falling in line with other literature that does not cosider the CMD outside of the main sequence, main sequence turn off, sub giant, and red giant branches; specifically, it should be noted that \fidanka is not setup to model the horizontal branch.

\begin{figure}
  \centering
  \includegraphics[width=0.85\textwidth]{figures/ngc2808/DistributionOfErrors.pdf}
  \caption{Percent Error distribution for each of the three deriver parameters. Note that these values will be sensitive to the magnitude uncertainties of the photometry. Here we made use of the ACS artificial star tests to estimate the uncertanties. {\color{blue}Note that currently this is built with 100 runs, these take a long time so currently re running with 1000 runs.}}
  \label{fig:validationDist}
\end{figure}

\section{Isochrone Fitting}\label{sec:isoFit}
We fit pairs of isochrones to the HUGS data for NGC 2808 using \fidanka, as
described in \S \ref{sec:fidanka}. Two isochrones, one for Population A and one
for Population E are fit simultaneously. These isochrones are constrained to
have distance modulus, $\mu$, and color excess, E(B-V) which agree to within
0.5\% and an ages which agree to within 1\%. Moreover, we constrain the mixing
length, $\alpha_{ML}$, for any two isochrones in a set to be within 0.5 of one
and other. For every isochrone in the set of combination of which fulfilling
these constraints $\mu$, $E(B-V)$, Age$_{A}$, and $Age_{B}$ are optimized to
reduce the $\chi^{2}$ distance ($\chi^{2} = \sum\sqrt{\Delta \text{color}^{2} +
\Delta \text{mag} ^{2}}$) between the fiducial lines and the isochrones.
Because we fit fiducial lines directly, we do not need to consider the binary
population fraction, $f_{bin}$, as a free parameter.

The best fit isochrones are shown in Figure \ref{fig:BestFitResults} and
optimized parameters for these are presented in Table \ref{tab:BestFitResults}.
The initial guess for the age of these populations was locked to 12 Gyr
and the initial Extinction was locked to 0.5 mag. The initial guess for the
distance modulus was determined at run time using a dynamic time warping
algorithm to best align the morphologies of the fiducial line with the target
isochrone. This algorithm is explained in more detail in the \fidanka
documentation under the function called \texttt{guess\_mu}. We find helium mass
fractions that are consistent with those identified in past literature
\citep[e.g.][]{Milone2015}. Note that our helium mass fraction grid has a
spacing of 0.03 between grid points and we are therefore unable to resolve
between certain proposed helium mass fractions for the younger sequence (for
example between 0.37 and 0.39). We also note that the best fit mixing
length parameter which we derive for populations A and E do not agree within
their uncertainties. This is not surprising as the much high mean molecular mass
of population E --- when compared to population A, due to population E's larger
helium mass fraction --- will result in a steeper adiabatic temperature
gradient.

\begin{figure*}
  \centering
  \includegraphics[width=0.9\textwidth]{figures/ngc2808/BestFitResults.pdf}
  \caption{Best fit isochrone results for NGC 2808. The best fit population A
  and E models are shown as black lines. The following 50 best fit models are
  presented as gray lines. The solid black line is fit to population A, while
  the dashed black line is fit to population E.}
  \label{fig:BestFitResults}
\end{figure*}

\begin{table*}
  \centering
  \begin{tabular}{c | c c c c c c}
    \hline
    Population & Age & Distance Modulus & Extinction & Y & $\alpha_{ML}$ & $\chi^{2}_{\nu}$\\
    & [Gyr] & & [mag] & & &\\
    \hline
    \hline
    A & 12.996$^{+0.87}_{-0.64}$ & 15.021 & 0.54 & 0.24 & 2.050 & 0.021\\
    E & 13.061$^{+0.86}_{-0.69}$ & 15.007 & 0.537 & 0.39 & 1.600 & 0.033 \\
    \hline
  \end{tabular}
  \caption{Best fit parameters derived from fitting isochrones to the fiducual
  lines derived from the NCG 2808 photometry. The one sigma uncertainty
  reported on population age were determined from the 16th and 84th percentiles
  of the distribution of best fit isochrones ages.}
  \label{tab:BestFitResults}
\end{table*}


Past literature \citep[e.g. ][]{Milone2015, Milone2018} have found helium mass
fraction variation from the low red-most to blue-most populations of $\sim 0.12$.
Here we find a helium mass fraction variation of 0.15 which, given the spacing
of the helium grid we use {\em is consistent with these past results}.

\subsection{The Number of Populartions in NGC 2808}
In order to estimate the number of populations which ideally fit the NGC 2808
F275W-F814W photometry without over-fitting the data we make use of silhouette
analysis \citep[][and in a similar manner to how \citet{Valle2022} perform
their analysis of spectroscopic data]{ROUSSEEUW198753}. We find the average
silhouette score for all tagged clusters identified using BGMM in all magnitude
bins over the CMD using the standard python module \texttt{sklearn}. Figure
\ref{fig:clusterAn} shows the silhouette analysis results and that two
populations fit the photometry most ideally. This is in line with what our BGMM
model predicts for the majority of the CMD.

\begin{figure}
  \centering
  \includegraphics[width=0.85\textwidth]{figures/ngc2808/ClusterAnalysis.pdf}
  \caption{Silhouette analysis for NGC 2808 F275W-F814W photometry. The
  Silhouette scores are an average of score for each magnitude bin. Positive
  scores indicate that the clustering algorithm produced well distinguished
  clusters while negative scores indicate clusters which are not well
  distinguished.}
  \label{fig:clusterAn}
\end{figure}

While we make use a purely CMD based approach in this work, other literature
has made use of Chromosome Maps. These consist of implicitly verticalized
pseudo colors. In the chromosome map for NGC 2808 there may be evidence for
more than two populations; however, the process of transforming magnitude
measurements into chromosome space results in dramatically increased
uncertainties for each star. We find a mean fractional uncertainty for
chromosome parameters of $\approx1$ when starting with magnitude measurements
having a mean best-case (i.e. uncertainty assumed to only be due to Poisson
statistics) fractional uncertainty of $\approx 0.0005$ (Figure
\ref{fig:chromMapUn}). Because of how \fidanka operates, i.e. resampling a
probability distribution for each star in order to identify clusters, we are
unable to make statistically meaningful statements from the chromosome map


\begin{figure}
  \centering
  \includegraphics[width=0.85\textwidth]{figures/ngc2808/ChromosomeMapFractionalErrorDist.pdf}
  \caption{Fractional uncertainty distribution of the chromosome map parameter
  space for targets in NGC 2808. Note that fractional uncertainties of the
  magnitudes which went into the production of this chromosome map were on the
  order of 0.0005 (the blue vertical line in both plots marks this). Further,
  we assumed that there was no uncertainty on the placement of the red and blue
  fiducial lines. If there were uncertainty on those placements then the mean
  of this distribution would be higher.}
  \label{fig:chromMapUn}
\end{figure}

\subsection{ACS-HUGS Photometric Zero Point Offset}
The Hubble legacy archive photometry used in this work is calibrated to the
Vega magnitude system. However, we have found that the photometry has a
systematic offset of $\sim0.026$ magnitudes in the F814W band when
compared to the same stars in the ACS survey (Figure \ref{fig:offset}). The
exact cause of this offset is unknown, but it is likely due to a difference in
the photometric zero point between the two surveys. A full correction of this
offset would require a careful re-reduction of the HUGS photometry, which is
beyond the scope of this work. We instead recognize a 0.02 inherent uncertainty
in the inferred magnitude of any fit when comparing to the ACS survey. This
uncertainty is small when compared to the uncertainty in the
distance modulus and should not affect the conclusion of this
paper. 

\begin{figure*}
  \centering
  \includegraphics[width=0.90\textwidth]{figures/ngc2808/photometricOffset.pdf}
  \caption{(left) CMD showing the photometric offset between the ACS and HUGS
  data for NGC 2808. CMDs have been randomly subsampled and colored by point
  density for clarity. (right) Mean difference between the color of the HUGS
  and ACS fiducual lines at the same magnitude. Note that the ACS data is
  systematically bluer than the HUGS data.}
  \label{fig:offset}
\end{figure*}

The oberved photometric offset between ACS and HUGS reductions introduces a
systematic uncertainty when comparing parameters derived from isochrone fits
to ACS data vs those fit to HUGS data. Specifically, this offset introduces a
$\sim 2 Gyr$ uncertainty when comparing ages between ACS and HUGS. Moreover,
for two isochrone of the same age, only separated by helium mass fraction, a
shift of the main sequence turn off of is also expected. Figure \ref{fig:HeMO}
shows this shift. Note a change in the helium mass fraction of a model by 0.03
results in an approximate 0.08 magnitude shift to the main sequence turn off
location. This means that the mean 0.026 magnitude offset we find in between
ACS and HUGS data corresponds to an additional approximate 0.01 uncertainty
in the derived helium mass fraction when comparing between these two data sets. 

\begin{figure}
  \centering
  \includegraphics[width=0.85\textwidth]{figures/ngc2808/HeliumMeanOffset.pdf}
  \caption{Main sequence turn off magnitude offset from a gauge helium mass
  fraction (Y=0.30 chosen). All main sequence turn off locations are measured
  at 12.3 Gyr}
  \label{fig:HeMO}
\end{figure}

\section{Results}\label{sec:results}
Using \fidanka we fit pairs of Population A + E isochrones to the HUGS data for
NGC 2808. Each pair of isochrones is allowed to vary in distance modulus,
reddening, relative helium mass fraction (A/E), and age. Any population pairs which vary by more than 1\% in distance modulus or B-V color excess are rejected. The $\chi^{2}$
distribution for the isochrone pairs is shown in Figure {\color{red}[FIGURE]}.
The best fit isochrones are shown in Figure \ref{fig:BestFitResults} and optimized
parameters for these are presented in Table {\color{red}[TABLE]}.

{\color{red} Need to make the chi2 dist plot still. Have all these values but need to figure out best way to visualize it}

\begin{figure}
  \centering
  \includegraphics[width=0.45\textwidth]{figures/ngc2808/BestFitResults.pdf}
  \label{fig:BestFitResults}
  \caption{Best fit isochrone results for NGC 2808.}
\end{figure}

\begin{table*}
  \centering
  \begin{tabular}{c | c c c c c c}
    \hline
    population & age & distance modulus & extinction & Y & $\alpha_{ML}$ & $\chi^{2}_{\nu}$\\
    & [Gyr] & & mag & & &\\
    \hline
    \hline
    A & 12.3 & 14.91 & 0.54 & 0.24 & 1.901 & 0.014\\
    E & 14.3 & 14.96 & 0.54 & 0.39 & 1.750 & 0.017 \\
    \hline
  \end{tabular}
  \label{tab:BestFitResults}
  \caption{Best fit parameters derived from fitting isochrones to the fiducual lines derived from the NCG 2808 photometry.}
\end{table*}

{\color{blue} Currently are still seeing a discontinutiy in the isochrobne below the MSTO. This must be addressed before submission.}

\subsection{The Number of Populartions in NGC 2808}
\fidanka provides a somewhat straigtforward way to estimate the number of populations expected in a given magnitude bin given the observations. See Section \ref{sec:fidanka} for specific implimentaiton details. Here we preform an analysis of the number of populations seen in the NGC 2808 F814W-F274W vs F814W color-magnitude diagram. We find that for the majority of the main sequence and red giant branches BGMM prefers two populations; wherease, near the main sequnce turn off and on the majority of the subgiant branches BGMM prefers a single population model.

{\color{red}[FIGURE SHOWING BGMM population probability]}


\section{Conclusion}\label{sec:2808conclusion}
Here we have preformed the first chemically self-consistent modeling of the
Milky Way Globular Cluster NGC 2808. We find that, updated atmospheric boundary
conditions and opacity tables do not have a significant effect on the inferred
helium abundances of multiple populations. Specifically, we find that
population  has a helium mass fraction of 0.24, while population E has a helium
mass fraction of 0.39. Additionally, we find that the ages of these two populations 
agree within uncertainties. We only find evidence for two distinct stellar
populations, which is in agreement with recent work studying the number
of populations in NGC 2808 spectroscopic data.

We introduce a new software suite for globular cluster science,
\fidanka, which has been released under a permissive open source license.
\fidanka aims to provide a statistically robust set of tools for estimating the
parameters of multiple populations within globular clusters.


\chapter{Modeling NGC 6752}


\chapter{Gap Sensitivity to Opacity Source}
\section{INTRODUCTION}
Over the last half of the 19th and first decade of the 20th centuries Lane,
Ritter, and Emden codified the earliest mathematical model of stellar
structure, the polytrope (Equation \ref{eqn:polytrope}), in \textit{Gaskugeln}
(Gas Balls) \citep{Emden1907}.

\begin{align}\label{eqn:polytrope}
	\frac{d}{d\xi}\left(\xi^{2}\frac{d\theta}{d\xi}\right) = -\xi^{2}\theta^{n}
\end{align}

Where $\xi$ and $\theta$ are dimensionless parameterizations of radius and
temperature respectively, and $n$ is known as the polytropic index. Despite this
early work, it wasn't until the late 1930s and early 1940s that the full set of
equations needed to describe the structure of a steady state,
radially-symmetric, star (known as the equations of stellar structure) began to
take shape as proton-proton chains and the Carbon-Nitrogen-Oxygen cycle were,
for the first time, seriously considered as energy generation mechanisms
\citep{Cowling1966}. Since then, and especially with the proliferation of
computers in astronomy, the equations of stellar structure have proven
themselves an incredibly predictive set of models.  

There are currently many stellar structure codes \citep[e.g.][]{Dotter2008,
Kovetz2009, Paxton2011} which integrate the equations of stellar structure ---
in addition to equations of state and lattices of nuclear reaction rates ---
over time to track the evolution of an individual star. The Dartmouth Stellar
Evolution Program (DSEP) \citep{Chaboyer2001, Bjork2006, Dotter2008} is one
such, well tested, stellar evolution program.

Here we propose to model low-mass stars in both the local solar neighborhood
and in globular clusters using DSEP. This work will primarily extend our
understanding of stellar physics in two areas: the effects of chemically
self-consistency on stellar models and time evolution of the core-convective
instabilities which ultimatly are belived to result in the observed paucity of
stars at a Gaia G magnitude of $\sim$10. [NEED CITATIONS IN THIS PARAGRAPH]

Low mass stars form an important component of the stellar population, with
stars less than [MASS HERE] making up more than 70\% of stars in the galaxy
[CITE]. Moreover, due to their long lives, low-mass stars provide essential
constraints on ages of various stellar populations [CITE]. In globular
clusters, where all stars are coeval to one of a limited number of populations,
low mass stars provide the vast majority of constraints when fitting ischrones [CITE].
Additionally, stars around the fully-convective transition mass show
age-dependent core-convective instabilities [CITE].

\subsection{Globular Clusters}

Globular clusters in the local universe are primarly composed of old and
consequently low-mass stars. For decades, prevaling thought had it globular
clusters were composed of a single stellar population born from a preisten
interstellar medium. This was supported by visibly tight main sequences and
clear main sequence turn offs in optical CMDs \citep[Figure
\ref{fig:M3CMD}][]{Sandage1953}. These early studies either did not handel or
had very large photometric uncertanties and therefore they were unable to
discriminate beteween CMD features with small separations,

\begin{figure}
	\centering
	\includegraphics[width=0.75\textwidth]{src/Figures/Gould53.png}
	\caption{$m_{pg}$ - $m_{pv}$ color-magnitude diagram for the globular cluster M3.}
	\label{fig:M3CMD}
\end{figure}

[SOMETHING ABOUT EARLY SPECTROSCOPIC INDICATIONS OF MPs]

With the presicion photometric measurements, degenerecies between noise and
intrinsic scatter were broken and it became clear that globular clusters are
almost universally composed of multiple stellar populations (MPs). [GRAB SOME
TEXT FROM THE NGC 2808 SECTION FOR HERE].

\subsection{Local Solar Neighborhood}
\citet{Jao2018} discovered a novel feature in the Gaia Gp-Rp color-magnitude
diagram. Around $M_{G}=10$ there is an approximatly 17\% decreas in stellar
density of the volume complete sample of stars \citeauthor{Jao2018} considered.
Subsequently, this has become known as either the Jao Gap, or Gaia M dwarf Gap.
Section \ref{} will go into more detail regarding the physics belived to
underpin this feature; however, in brief convective instabilities in the core
are belived to form for stars straddeling the fully convective transition mass.
These instabilities result in stars preferentially falling to either side of
the gap location.

Stellar modeling has been sucsessful in reproducing the Jap Gap and, with these
models, we have begun to constrain parameters which constrain gap location. For
example, it is now well documented that a stars metallicity can affect the gap
color by up to [HOW MUCH DID GREG FIND/CHECK FOR OTHER PAPERS ON THIS]. 

Initial testting which we have done using DSEP along with work by [PAPER] also
indicated the Jao Gap's color sensitivity to age. We observe that as models age
the Jao Gap moves [DIRECTION OF MOVMENT IN MAG AND COLOR SPACE].

The OPAL opacity tables in particular are very widely used by current
generation stellar evolution programs (in addition to current generation
stellar model and isochrone grids). However, they are no longer the most up
date elemental opacities. Moreover, the generation mechanism for these tables,
a webform, is no longer reliably online.  Consequently, it makes sense to
transition to more modern opacity tables with a more stable generation
mechanism.

Here we will present work transitioning DSEP from OPAL opacities to opacities
based on measurements from Los Alamos national Labs T-1 group
\citep[OPLIB][]{Colgan2016}. Moreover, we will present two projects which are
in large part reliant on these updated opacities. For the first project we
investigate the affects of chemically self consistent modeling of multiple
populations within the globular cluster NGC 2808, and for the second project we
present the effects of the OPLIB opacities on the location of the recently
discovered Gaia M-dwarf gap.

This paper is organized as follows. In Section \ref{sec:opac} we outline some
basic information about OPLIB opacities, how we query them, and how we modify
them to work with DSEP. In Section \ref{sec:2808} we discuss scientific
background of the first project along with the current work done towards its
goal. Finally, in Section \ref{sec:Jao} we present our findings on the effects
of OPLIB opacities on the location of the Gaia M-dwarf gap.






\section{The Underlying Physics of the Gap}\label{sec:JaoGap}
A theoretical explanation for the Jao Gap (Figure \ref{fig:JaoGap}) comes from
\citet{van2012}, who propose that in a star directly above the transition mass,
due to asymmetric production and destruction of $^{3}$He during the
proton-proton I chain (ppI), periodic luminosity variations can be induced.
This process is known as convective-kissing instability. Very shortly after the
zero-age main sequence such a star will briefly develop a radiative core;
however, as the core temperature exceeds $7\times 10^{6}$ K, enough energy will
be produced by the ppI chain that the core once again becomes convective. At
this point the star exists with both a convective core and envelope, in
addition to a thin, radiative layer separating the two. Subsequently,
asymmetries in ppI affect the evolution of the star's convective core.

While kissing instability has been the most widely adopted model to
explain the existence of the Jao Gap, slightly different mechanisms have also
been proposed. \citet{MacDonald2018} make use of a fully implicit stellar
evolution suite which treats convective mixing as a diffusive property.
\citeauthor{MacDonald2018} treat convective mixing this way in order to account
for a core deuterium concentration gradient proposed by \citet{Baraffe1997}.
Under this treatment the instability results only in a single mixing event ---
as opposed to periodic mixing events. Single mixing events may be more in line with
observations (see section \ref{sec:results} for more details on how periodic
mixing can effect a synthetic population) where there is only well documented
evidence of a single gap. However, recent work by \citet{Jao2021} which
identify an second under density of stars below the canonical gap, does leave
the door open for the periodic mixing events.

\begin{figure}
	\centering
	\includegraphics[width=0.85\textwidth]{figures/jaoOpacity/JaoGapEDR3.png}
	\caption{The Jao Gap (circled) seen in the Gaia Catalogue of Nearby Stars \citep{GaiaCollaboration2021}.}
	\label{fig:JaoGap}
\end{figure}

The proton-proton I chain constitutes three reactions 
\begin{enumerate} 
	\item $p + p \longrightarrow d + e^{+} + \nu_{e}$
	\item $p + d \longrightarrow \ ^{3}\text{He} + \gamma$
	\item $^{3}\text{He} + ^{3}\text{He} \longrightarrow \ ^{3}\text{He} + 2p$ 
\end{enumerate} 
Initially, reaction 3 of ppI consumes $^{3}$He at a slower rate than it is
produced by reaction 2 and as a result, the core $^{3}$He abundance and
consequently the rate of reaction 3, increases with time. The core convective
zone expands as more of the star becomes unstable to convection. This expansion
continues until the core connects with the convective envelope. At this point
convective mixing can transport material throughout the entire star and the
high concentration of $^{3}$He rapidly diffuses outward, away from the core,
decreasing energy generation as reaction 3 slows down. Ultimately, this leads
to the convective region around the core pulling back away from the convective
envelope, leaving in place the radiative transition zone, at which point
$^{3}$He concentrations grow in the core until it once again expands to meet
the envelope.  These periodic mixing events will continue until $^{3}$He
concentrations throughout the star reach an equilibrium ultimately resulting in
a fully convective star. Figure \ref{fig:Kippenhan1} traces the evolution of a
characteristic star within the Jao Gap's mass range.

\begin{figure*}
	\centering
	\includegraphics[width=0.95\textwidth]{figures/jaoOpacity/Kippenhan.pdf}
	\caption{Diagram for a characteristic stellar model of 0.35625 $M_{\odot}$
	which is within the Jao Gap's mass range. The black shaded regions denote
	whether, at a particular model age, a radial shell within the model is
	radiative (with white meaning convective). The lines trace the models core
	temperature, core $^{3}$He mass fraction, fractional luminosity wrt. the
	zero age main sequence and fractional radius wrt. the zero age main
	sequence.}
	\label{fig:Kippenhan1}
\end{figure*}


\subsection{Efforts to Model the Gap}
Since the identification of the Gap, stellar modeling has been
conducted to better constrain its location, effects, and exact cause.
Both \citet{Mansfield2021} and \citet{Feiden2021} identify that the Gap's mass
location is correlated with model metallicity --- the mass-luminosity
discontinuity in lower metallicity models being at a commensurately lower mass.
\citet{Feiden2021} suggests this dependence is due to the steep relation of
the radiative temperature gradient, $\nabla_{rad}$, on temperature and, in turn,
on stellar mass.

\begin{align}\label{eqn:radGrad}
	\nabla_{rad} \propto \frac{L\kappa}{T^{4}}
\end{align}

As metallicity decreases so does opacity, which, by Equation \ref{eqn:radGrad},
dramatically lowers the temperature at which radiation will dominate energy
transport \citep{Chabrier1997}. Since main sequence stars are virialized the
core temperature is proportional to the core density and total mass. Therefore,
if the core temperature where convective-kissing instability is expected
decreases with metallicity, so too will the mass of stars which experience such
instabilities.

% \begin{align}\label{eqn:TMRelation}
% 	T_{c} \propto \rho_{c}M^{2}
% \end{align}

The strong opacity dependence of the Jao Gap begs the question: what is
the effect of different opacity calculations on Gap properties?
As we can see above, changing opacity should affect the Gap's location in the
mass-luminosity relation and therefore in a color-magnitude diagram. Moreover,
current models of the Gap have yet to locate it precisely in the CMD
\citep{Feiden2021} with an approximate 0.16 G-magnitude difference between the
observed and modeled Gaps. Opacity provides one, as yet unexplored, parameter
which has the potential to resolve these discrepancies.

\section{Updated Opacities}\label{sec:opac}
% Radiative opacity is fundamental to stellar structure, it determines how much
% incident radiation is absorbed or scattered. Moreover, when a media is in
% thermodynamic equilibrium with the radiation field, that is when the temperature
% of the media and that of the radiation field is the same, the opacity may be
% used via Kirchhoff's law to find the emissivity of a material
% \citep{Huebner2014}. Local Thermodynamic Equilibrium (LTE) is a common state to
% find within a star and therefore stellar models have long relied on opacities
% calculated in LTE.
Multiple groups have released high-temperature opacities including, the Opacity
Project \citep[OP][]{Seaton1994}, Laurence Livermore National Labs OPAL opacity
tables \citep{Iglesias1996}, and Los Alamos National Labs OPLIB opacity tables
\citep{Colgan2016}. OPAL high-temperature radiative opacity tables in
particular are very widely used by current generation isochrone grids
\citep[e.g. Dartmouth, MIST, \& StarEvol, ][]{Dotter2008,Choi2016,Amard2019}.
OPLIB opacity tables \citep{Colgan2016} are not widely used but include the
most up-to-date plasma modeling.

While the overall effect on the CMD of using OPLIB compared to OPAL tables is
small, the strong theoretical opacity dependence of the Jao Gap raises the
potential for these small effects to measurably shift the Gap's location. We
update DSEP to use high temperature opacity tables based on measurements from
Los Alamos national Labs T-1 group \citep[OPLIB,][]{Colgan2016}. The OPLIB
tables are created with ATOMIC \citep{Magee2004,Hakel2006,Fontes2016}, a modern
LTE and non-LTE opacity and plasma modeling code. These updated tables were
initially created in order to incorporate the most up to date plasma
physics at the time \citep{Bahcall2005}. 

OPLIB tables include monochromatic Rosseland mean opacities --- composed from
bound-bound, bound-free, free-free, and scattering opacities --- for elements
hydrogen through zinc over temperatures 0.5eV to 100 keV (5802 K -- 1.16$\times
10 ^{9}$ K) and for mass densities from approximately $10^{-8}$ g cm$^{-3}$ up
to approximately $10^{4}$ g cm$^{-3}$ (though the exact mass density range
varies as a function of temperature). 

DSEP ramps the \citet{Ferguson2005} low temperature opacities to high
temperature opacities tables between $10^{4.3}$ K and $10^{4.5}$ K; therefore,
only differences between high-temperature opacity sources above $10^{4.3}$ K
can effect model evolution. When comparing OPAL and OPLIB opacity tables
(Figure \ref{fig:opacComp}) we find OPLIB opacities are systematically lower
than OPAL opacities for temperatures above $10^{5}$ K. Between $10^{4.3}$ and
$10^{5} K$ OPLIB opacities are larger than OPAL opacities. These generally
lower opacities will decrease the radiative temperature gradient throughout
much of the radius of a model.

\begin{figure}
	\centering
	\includegraphics[width=0.85\textwidth]{figures/jaoOpacity/OpacityComparision.pdf}
	\caption{Rosseland mean opacity with the GS98 solar composition for both
	OPAL opacities and OPLIB opacities (top). Residuals between OPLIB opacities
	and OPAL opacities (bottom). These opacities are plotted at $\log _{10}(R)
	= -0.5$, $X=0.7$, and $Z=0.02$. $\log _{10}(R)=-0.5$ approximates
	much of the interior a 0.35 M$_{\odot}$ model. Note how the OPLIB
	opacities are systematically lower than the OPAL opacities for temperatures
	above $10^{5.2}$ K.}
	\label{fig:opacComp}
\end{figure}

\subsection{Table Querying and Conversion}
The high-temperature opacity tables used by DSEP and most other stellar
evolution programs give Rosseland-mean opacity, $\kappa_{R}$, along three
dimensions: temperature, a density proxy $R$ (Equation \ref{eqn:R}; $T_{6} =
T\times10^{-6}$, $\rho$ is the mass density), and composition. 

\begin{align} \label{eqn:R}
	R = \frac{\rho}{T_{6}^{3}}
\end{align}

OPLIB tables may be queried from a web
interface\footnote{https://aphysics2.lanl.gov/apps/}; however, OPLIB opacities
are parametrized using mass-density and temperature instead of $R$ and
temperature. It is most efficient for us to convert these tables to the OPAL
format instead of modifying DSEP to use the OPLIB format directly. In order to
generate many tables easily and quickly we develop a web scraper
\citep[\texttt{pyTOPSScrape},][]{Boudreaux22pyTOPS} which can automatically retrieve
all the tables needed to build an opacity table in the OPAL format.
\texttt{pyTOPSScrape}\footnote{https://github.com/tboudreaux/pytopsscrape} has
been released under the permissive \texttt{MIT} license with the consent of the
Los Alamos T-1 group. For a detailed discussion of how the web scraper works
and how OPLIB tables are transformed into a format DSEP can use see Appendices
\ref{apx:pytopsscrape} \& \ref{apx:interp}.

\subsection{Solar Calibrated Stellar Models}\label{sec:SCSM}
% In order to further validate the OPLIB high-temperature opacities we first visually
% compare a set of opacity vs. temperature curves from OPLIB at a constant $R$
% and \citet{Grevesse1998} composition (GS98) to the same curve from OPAL. A
% characteristic opacity vs temperature curve is shown in Figure
% \ref{fig:OpacCompare}, $\log _{10}(R) = -1.5$ is chosen as for much of the
% radius of a main sequence star $\log _{10}(R)$ is around that value. The
% largest variation in $\kappa_{R}$ from OPAL to OPLIB at $\log _{10}(R)=-1.5$ is
% on the order of a few percent. This is inline with expectations of OPLIB and OPAL
% being in relatively close agreement \citep{Colgan2016}.


In order to validate the OPLIB opacities, we generate a solar calibrated
stellar model (SCSM) using these new tables. We first manually calibrate the
surface Z/X abundance to within one part in 100 of the solar value \citep[][Z/X=0.23]{Grevesse1998}.
Subsequently, we allow both the convective mixing length parameter,
$\alpha_{ML}$, and the initial Hydrogen mass fraction, $X$, to vary
simultaneously, minimizing the difference, to within one part in $10^{5}$,
between resultant models' final radius and luminosity to those of the sun.
Finally, we confirm that the model's surface Z/X abundance is still within one
part in 100 of the solar value.

\begin{figure}
	\centering
	\includegraphics[width=0.85\textwidth]{figures/jaoOpacity/HRDiagramOPALvsOPLIB_SCCM.pdf}
	\caption{HR Diagram for the two SCSMs, OPAL and OPLIB. OPLIB is shown as a red
	dashed line.}
	\label{fig:OPLIBOPALHR}
\end{figure}

Solar calibrated stellar models evolved using GS98 OPAL and OPLIB opacity
tables (Figure \ref{fig:OPLIBOPALHR}) differ $\sim 0.5\%$ in the SCSM hydrogen
mass fractions and $\sim 1.5\%$ in the SCSM convective mixing length parameters
(Table \ref{tab:SCSMResults}). While the two evolutionary tracks are very
similar, note that the OPLIB SCSM's luminosity is systematically lower past the
solar age. While at the solar age the OPLIB SCSM luminosity is effectively the
same as the OPAL SCSM. This luminosity difference between OPAL and OPLIB based
models is not inconsistent with expectations given the more shallow radiative
temperature gradient resulting from the lower OPLIB opacities

\begin{table}
	\centering
	\begin{tabular}{l c c}
		\hline
		Model & $X$ & $\alpha_{ML}$ \\
		\hline
		\hline
		OPAL & 0.7066 & 1.9333 \\
		OPLIB & 0.7107 & 1.9629
	\end{tabular}
	\caption{Optimized parameters for SCSMs evolved using OPAL and OPLIB high
	temperature opacity tables.}
	\label{tab:SCSMResults}
\end{table}


\section{\texttt{pyTOPSScrape}}\label{apx:pytopsscrape}
\texttt{pyTOPSScrape} provides an easy to use command line and python interface
for the OPLIB opacity tables accessed through the TOPS web form. Extensive
documentation of both the command line and programmatic interfaces is linked
in the version controlled repository. However, here we provide a brief,
illustrative, example of potential use.

Assuming \texttt{pyTOPSScrape} has been installed and given some working
directory which contains a file describing a base composition (``comp.dat'')
and another file containing a list of rescalings of that base composition
(``rescalings.dat'') (both of these file formats are described in detail in the
documentation), one can query OPLIB opacity tables and convert them to a form
mimicking that of type 1 OPAL high temperature opacity tables using the
following shell command.

\begin{verbatim}
	$ generateTOPStables comp.dat rescalings.dat -d ./TOPSCache -o out.opac -j 20
\end{verbatim}

\noindent For further examples of pyTOPSScrape please visit the repository.

\section{Interpolating $\rho \rightarrow $ R}\label{apx:interp}
OPLIB parameterizes $\kappa_{R}$ as a function of mass density, temperature in keV,
and composition. Type 1 OPAL high temperature opacity tables, which DSEP and
many other stellar evolution programs use, instead parameterizes opacity as a function
of temperature in Kelvin, $R$ (Equation \ref{eqn:Req}), and composition. The
conversion from temperature in keV to Kelvin is trivial (Equation
\ref{eqn:K2Kev}).
\begin{align}\label{eqn:Req}
	R = \frac{\rho}{T_{6}^{3}}
\end{align}
\begin{align}\label{eqn:K2Kev}
	T_{K} = T_{keV} * 11604525.0061657
\end{align}
However, the conversion from mass density to $R$ is more involved. Because $R$
is coupled with both mass density and temperature there there is no way to
directly convert tabulated values of opacity reported in the OPLIB tables to
their equivalents in $R$ space. The TOPS webform does allow for a
density range to be specified at a specific temperature, which allows for R
values to be directly specified. However, issuing a query to the TOPS webform
for not just every composition in a Type 1 OPAL high temperature opacity table
but also every temperature for every composition will increase the number of
calls to the webform by a factor of 70. Therefore, instead of directly
specifying R through the density range we choose to query tables over a
broad temperature and density range and then rotate these tables,
interpolating $\kappa_{R}(\rho,T_{eff}) \rightarrow \kappa_{R}(R,T_{eff})$. 


To perform this rotation we use the \texttt{interp2d} function within
\texttt{scipy}'s \texttt{interpolate} \citep{2020SciPy-NMeth} module to
construct a cubic bivariate B-spline \citep{Dierckx1981} interpolating function
$s$, with a smoothing factor of 0, representing the surface $\kappa_{R}(\rho,
T_{eff})$. For each $R^{i}$ and $T^{j}_{eff}$ reported in type 1 OPAL tables,
we evaluate Equation \ref{eqn:Req} to find $\rho^{ij} =
\rho(T^{j}_{eff},R^{i})$.  Opacities in $T_{eff}$, $R$ space are then inferred
as $\kappa^{ij}_{R}(R^{i},T^{j}_{eff}) = s(\rho^{ij}, T^{j}_{eff})$. 

As first-order validation of this interpolation scheme we can perform a similar
interpolation in the opposite direction, rotating the tables back to
$\kappa_{R}(\rho, T_{eff})$ and then comparing the initial, ``raw'', opacities
to those which have gone through the interpolations process. Figure
\ref{fig:fracdiff} shows the fractional difference between the raw opacities
and a set which have gone through this double interpolation. The red line
denotes $\log(R)=-0.79$ where models near the Jao Gap mass range will
tend to sit for much of their radius. Along the $\log(R)=-0.79$ line the mean
fractional difference is $\langle \delta \rangle = 0.005$ with an uncertainty of
$\sigma_{\langle\delta\rangle} = 0.013$. One point of note is that, because the
initial rotation into $\log(R)$ space also reduces the domain of the opacity
function, interpolation-edge effects which we avoid initially by extending the
domain past what type 1 OPAL tables include cannot be avoided when
interpolating back into $\rho$ space. 

\begin{figure}
	\centering
	\includegraphics[width=0.85\textwidth]{figures/jaoOpacity/FractionalDifference.pdf}
	\caption{Log Fractional Difference between opacities in $\kappa_{R}(\rho,
	T_{eff})$ space directly queried from the OPLIB web-form and those which
	have been interpolated into $\log(R)$ space and back. Note that, due to the
	temperature grid of type 1 OPAL tables not aligning perfectly which the temperature
	grid OPLIB uses there may be edge effects where the interpolation is poorly
	constrained. The red line corresponds to $\log(R) = -0.79$ where much of a
	stellar model's radius exists.}
	\label{fig:fracdiff}
\end{figure}


\section{Modeling}\label{sec:modeling}
In order to model the Jao Gap we evolve two extremely finely sampled mass grids
of models. One of these grids uses the OPAL high-temperature opacity tables
while the other uses the OPLIB tables (Figure \ref{fig:PunchIn}). Each grid
evolves a model every 0.00025 $M_{\odot}$ from 0.2 to 0.4 $M_{\odot}$ and every
0.005 $M_{\odot}$ from 0.4 to 0.8 $M_{\odot}$. All models in both grids use a
GS98 solar composition, the (1, 101, 0) \texttt{FreeEOS} (version
{\color{red}2.7}) configuration, and 1000 year old pre-main sequence polytropic
models, with polytropic index 1.5, as their initial conditions. We
include gravitational settling in our models where elements are grouped
together. Finally, we set a maximum allowed timestep of 50 million years to
assure that we fully resolve the build of of core $^{3}$He in gap stars.

Despite the alternative view of convection provided by
\citet{MacDonald2018} discussed in Section \ref{sec:JaoGap}, given that the
mixing timescales in these low mass stars are so short \citep[between $10^{7}$s
and $10^{8}$s per][Figure 2 \& Equation 39, which present the
averaged velocity over the convection zone]{Jermyn2022} instantaneous mixing is a valid
approximation. Moreover, one principal motivation for a diffusive model of
convective mixing has been to account for a deuterium concentration gradient
which \citet{Chabrier1997} identify will develop when the deuterium lifetime
against proton capture is significantly shorter than the mixing timescale.
However, the treatment of energy generation used by DSEP \citep{Bahcall2001}
avoides this issue by computing both the equilibrium deuterium abundance and
luminosity of each shell individually, implicitly accounting for the overall
luminosity discrepancy identified by \citeauthor{Chabrier1997}.

Because in this work we are just interested in the location shift of the Gap as
the opacity source varies, we do not model variations in composition.
\citet{Mansfield2021,Jao2020,Feiden2021} all look at the effect composition has
on Jao Gap location. They find that as population metallicity increases so too
does the mass range and consequently the magnitude of the Gap. From an extremely
low metallicity population (Z=0.001) to a population with a more solar like
metallicity this shift in mass range can be up to 0.05 M$_{\odot}$
\citep{Mansfield2021}.

\begin{figure}
	\centering
	\includegraphics[width=0.75\textwidth]{figures/jaoOpacity/OPALPunchIn.pdf}
	\includegraphics[width=0.75\textwidth]{figures/jaoOpacity/OPLIBPunchIn.pdf}
	\caption{Mass-luminosity relation at 7 Gyrs for models evolved using OPAL opacity
	tables (top) and those evolved using OPLIB opacity tables (bottom). Note
	the lower mass range of the OPLIB Gap.}
	\label{fig:PunchIn}
		
\end{figure}

\subsection{Population Synthesis}
In order to compare the Gap to observations we use in house population
synthesis code. We empirically calibrate the relation between G, BP, and RP
magnitudes and their uncertainties along with the parallax/G magnitude
uncertainty relation using the Gaia Catalog of Nearby Stars
\citep[GCNS,][]{GaiaCollaboration2021} and Equations \ref{eqn:plxCalib} \&
\ref{eqn:MagCalib}. $M_{g}$ is the Gaia G magnitude while $M_{i}$ is the
magnitude in the i$^\text{th}$ band, G, BP, or RP. The coefficients $a$, $b$,
and $c$ determined using a non-linear least squares fitting routine. Equation
\ref{eqn:plxCalib} then models the relation between G magnitude and parallax
uncertainty while Equation \ref{eqn:MagCalib} models the relation between each
magnitude and its uncertainty.

\begin{align}\label{eqn:plxCalib}
	\sigma_{plx}(M_{g}) = ae^{bM_{g}}+c
\end{align}
\begin{align}\label{eqn:MagCalib}
	\sigma_{i}(M_{i}) = ae^{M_{i}-b}+c
\end{align}

\noindent The full series of steps in our population synthesis code
are:

\begin{figure}
	\centering
	\includegraphics[width=0.85\textwidth]{figures/jaoOpacity/pdist.pdf}
	\caption{Probability distribution sampled when assigning true parallaxes to
	synthetic stars. This distribution is built from the GCNS and includes all
	stars with BP-RP colors between 2.3 and 2.9, the same color range
	of the Jao Gap.}
	\label{fig:pdist}
\end{figure}

\begin{enumerate}
	\item Sample from a \citet{Sollima2019} ($0.25 M_{\odot} < M < 1 M_{\odot}$,
		$\alpha=-1.34\pm0.07$) IMF to determine synthetic star mass.
	\item Find the closest model above and below the synthetic star, lineally
		interpolate these models' $T_{eff}$, $\log(g)$, and $\log(L)$ to those
		at the synthetic star mass.
	\item Convert synthetic star $g$, $T_{eff}$, and $Log(L)$ to Gaia G, BP,
		and RP magnitudes using the Gaia (E)DR3 bolometric corrections
		\citep{Creevey2022} along with code obtained thorough personal
		communication with Aaron Dotter \citep{Choi2016}.
	\item Sample from the GCNS parallax distribution (Figure \ref{fig:pdist}),
		limited to stars within the BP-RP color range of 2.3 -- 2.9, to assign
		synthetic star a ``true'' parallax.
	\item Use the true parallax to find an apparent magnitude for each filter.
	\item Evaluate the empirical calibration given in Equation
		\ref{eqn:plxCalib} to find an associated parallax uncertainty. Then
		sample from a normal distribution with a standard deviation equal to
		that uncertainty to adjust the true parallax resulting in an
		``observed'' parallax.
	\item Use the ``observed'' parallax and the apparent magnitude to find an
		``observed'' magnitude.
	\item Fit the empirical calibration given in Equation \ref{eqn:MagCalib} to
		the GCNS and evaluate it to give a magnitude uncertainty scale in each
		band.
	\item Adjust each magnitude by an amount sampled from a normal
		distribution with a standard deviation of the magnitude uncertainty
		scale found in the previous step.
\end{enumerate}

This method then incorporates both photometric and astrometric uncertainties
into our population synthesis. An example 7 Gyr old synthetic populations
using OPAL and OPLIB opacities are presented in Figure
\ref{fig:PopSynthCompareBasic}.

\begin{figure*}
	\centering
	\includegraphics[width=0.85\textwidth]{figures/jaoOpacity/OPALOPLIB_popsynth_compare.pdf}
	\caption{Population synthesis results for models evolved with OPAL (left)
	and models evolved with OPLIB (right). A Gaussian kernel-density estimate
	has been overlaid to better highlight the density variations.}
	\label{fig:PopSynthCompareBasic}
\end{figure*}

\subsection{Mixing Length Dependence}
In order to test the sensitivity of Gap properties to mixing length parameter we
evolve three separate sets OPLIB of models. The first uses a GS98
solar calibrated mixing length parameter, the second uses a mixing length parameter of
1.5, and the third uses a mixing length parameter of 1.0.

We find a clear inverse correlation between mixing length parameter used and
the magnitude of the Jao Gap Figures \ref{fig:MixingLengthCMD} \&
\ref{fig:MixingLengthScaling} ($\mu_{G} \propto -1.5\alpha_{ML}$, where
$\mu_{G}$ is the mean magnitude of the Gap). This is somewhat surprising given
the long established view that the mixing length parameter is of little
relevance in fully convective stars \citep{Baraffe1997}. We find an approximate
0.3 magnitude shift in both the color and magnitude comparing a solar
calibrated mixing length parameter to a mixing length parameter of 1.5, despite only a 16K
difference in effective temperature at 7Gyr between two 0.3 solar mass models.
The slight temperature differences between these models are
attributable to the steeper adiabatic temperature gradients just below the
atmosphere in the solar calibrated mixing length model compared to the
$\alpha_{ML} = 1.5$ model ($\nabla_{ad,solar} - \nabla_{ad,1.5} \approx 0.05$).
Despite this relatively small temperature variance, the large magnitude
difference is expected due to the extreme sensitivity of the bolometric
corrections on effective temperature at these low temperatures. The
mixing length parameter then provides a free parameter which may be used to shift the gap
location in order to better match observations without having a major impact on
the effective temperature of models. Moreover, recent work indicates that using
a solar calibrated mixing length parameter is not appropriate for all stars
\citep[e.g.][]{Trampedach2014, Joyce2018}.

\begin{figure}
	\centering
	\includegraphics[width=0.85\textwidth]{figures/jaoOpacity/./alphaMLComparisionCMD.pdf}
	\caption{CMD showing OPLIB populations (from left to right) A, B, and C.}
	\label{fig:MixingLengthCMD}
\end{figure}

\begin{figure}
	\centering
	\includegraphics[width=0.85\textwidth]{figures/jaoOpacity/./MixingLengthScaling.pdf}
	\caption{Location of the two identified paucities of stars in OPLIB synthetic
	populations as a function of the mixing length used.}
	\label{fig:MixingLengthScaling}
\end{figure}

Given the variability of gap location with mixing length parameter, it is
possible that a better fit to the gap location may be achieved through
adjustment of the convective mixing length parameter. However, calibrations of
the mixing length parameter for stars other than the sun have focused on stars
with effective temperature at or above that of the sun and there are no current
calibrations of the mixing length parameter for M dwarfs. Moreover, there are
additional uncertainties when comparing the predicted gap location to the
measured gap location, such as those in the conversion from effective
temperature, surface gravity, and luminosity to color, which must be considered
if the mixing length parameter is to be used as a gap location free parameter. Given the
dangers of freely adjustable parameters and the lack of an a priori expectation
for what the convective mixing parameter should be for the population of M
Dwarfs in the Gaia DR2 and EDR3 CMD any attempt to use the Jao Gap magnitude to
calibrate a mixing length parameter value must be done with caution, and take into
account the other uncertainties in the stellar models which could affect the
Jao Gap magnitude.


\section{Rotation--Activity Relation}\label{sec:results}
We show our rotation-activity relation in Figures
\ref{fig:RpHKvsRossbySelf} \& \ref{fig:RpHKvsRossbyDef}. Note that
errors are shown in both figures; however, they render smaller than the data
point size. Ca II H\&K is also known to be time variable
\citep[e.g.][]{Baroch2020,Perdelwitz2021}, which is not captured in our
single-epoch data. There is one target cut off by the domain of this graph,
2MASS J10252645+0512391. This target has a measured vsini of $59.5\pm2.1$ km
s$^{-1}$ \citep{Kesseli2018} and is therefore quite rotationally broadened, which
is known to affect $R'_{HK}$ measurements \citep[figure 8]{Schroder2009}. The
data used to generate this figure is given in Table \ref{tab:finalData}. Table
\ref{tab:finalData} includes uncertainties, the R'$_{HK}$ measurements for
stars which did not have photometrically derived rotational periods in MEarth,
and data for 2MASS J10252645+0512391

We find a rotation activity relationship qualitatively similar to that
presented in \citet{Def17}. Our rotation activity relationship exhibits both
the expected saturated and unsaturated regimes --- the flat region at $Ro <
Ro_{s}$ and the sloped region at $Ro \geq Ro_{s}$ respectively. We fit the
rotation activity relation given in Equation \ref{eqn:fitEqn} to our data using
Markov Chain Monte Carlo (MCMC), implemented in \texttt{pymc}
\citep{Salvatier2016}. 

  \begin{equation}\label{eqn:fitEqn}
      \log(R'_{HK}) = \begin{cases}
          \log(R_{s}) & Ro < Ro_{s} \\
          k\log(Ro) + \log(R_{s}) - k\log(Ro_{s}) & Ro \geq Ro_{s}
      \end{cases}
  \end{equation}

\noindent $Ro_{s}$ is the Rossby number cutoff between the saturated and
unsaturated regime. $R_{s}$ is the maximum, saturated, value of $R'_{HK}$ and
$k$ is the index of the power law when $Ro \geq Ro_{s}$. Due to the
issues measuring $R'_{HK}$ for high vsini targets discussed above, we exclude
2MASS J10252645+0512391 from this fit. All logarithms are base ten unless
another base is explicitly given.
\begin{table}[ht]
  \small
    \centering
    \setlength{\tabcolsep}{4pt}
    \begin{tabular}{lcccccccc}
\hline
2MASS ID & Mass & $Ro$ & $\log(R'_{HK})$ & $\log(R'_{HK})_{err}$ & $V_{mag}$ & $V-K$ & prot & $r_{prot}$\\
 & $\mathrm{M_{\odot}}$ &  &  &  & $\mathrm{mag}$ & $\mathrm{mag}$ & $\mathrm{d}$ &   \\
\hline
\hline
06000351+0242236 & 0.24 & 0.020 & -4.5475 & 0.0021 & 11.31 & 5.268 & 1.809 & 2016ApJ...821...93N  \\
02125458+0000167 & 0.27 & 0.048 & -4.6345 & 0.0014 & 13.58 & 5.412 & 4.732 & 2016ApJ...821...93N  \\
01124752+0154395 & 0.28 & 0.026 & -4.4729 & 0.0017 & 14.009 & 5.240 & 2.346 & 2016ApJ...821...93N  \\
10252645+0512391 & 0.11 & 0.000 & -4.9707 & 0.0380 & 18.11 & 7.322 & 0.102 & 2016ApJ...821...93N  \\
05015746-0656459 & 0.17 & 0.873 & -5.0049 & 0.0028 & 12.2 & 5.464 & 88.500 & 2012AcA....62...67K  \\
06022261-2019447 & 0.23 & 1.307 & -5.6980 & 0.0192 & 13.26 & 4.886 & 95.000 & This Work  \\
06105288-4324178 & 0.30 & 0.705 & -5.2507 & 0.0139 & 12.28 & 4.968 & 53.736 & 2018AJ....156..217N  \\
09442373-7358382 & 0.24 & 0.542 & -5.6026 & 0.0147 & 15.17 & 5.795 & 66.447 & 2018AJ....156..217N  \\
14211512-0107199 & 0.24 & 1.160 & -5.5846 & 0.0125 & 13.12 & 5.027 & 91.426 & 2018AJ....156..217N  \\
14294291-6240465 & 0.12 & 0.394 & -5.0053 & 0.0014 & 11.13 & 6.746 & 83.500 & 1998AJ....116..429B  \\
16352464-2718533 & 0.23 & 1.423 & -5.5959 & 0.0108 & 14.18 & 5.182 & 122.656 & 2018AJ....156..217N  \\
16570570-0420559 & 0.24 & 0.014 & -4.3071 & 0.0014 & 12.25 & 5.130 & 1.212 & 2012AcA....62...67K  \\
02004725-1021209 & 0.34 & 0.188 & -4.7907 & 0.0026 & 14.118 & 5.026 & 14.793 & 2018AJ....156..217N  \\
18494929-2350101 & 0.18 & 0.034 & -4.5243 & 0.0015 & 10.5 & 5.130 & 2.869 & 2007AcA....57..149K  \\
20035892-0807472 & 0.33 & 0.946 & -5.6530 & 0.0077 & 13.54 & 5.254 & 84.991 & 2018AJ....156..217N  \\
21390081-2409280 & 0.21 & 1.152 & -6.1949 & 0.0190 & 13.45 & 5.091 & 94.254 & 2018AJ....156..217N  \\
23071524-2307533 & 0.30 & 0.720 & -5.2780 & 0.0077 & 13.587 & 4.849 & 51.204 & 2018AJ....156..217N  \\
00094508-4201396 & 0.30 & 0.009 & -4.3392 & 0.0018 & 13.62 & 5.397 & 0.859 & 2018AJ....156..217N  \\
00310412-7201061 & 0.31 & 0.906 & -5.3879 & 0.0074 & 13.69 & 5.245 & 80.969 & 2018AJ....156..217N  \\
01040695-6522272 & 0.17 & 0.006 & -4.4889 & 0.0024 & 13.98 & 5.448 & 0.624 & 2018AJ....156..217N  \\
02014384-1017295 & 0.19 & 0.034 & -4.5400 & 0.0022 & 14.473 & 5.284 & 3.152 & 2018AJ....156..217N  \\
03100305-2341308 & 0.40 & 0.028 & -4.2336 & 0.0017 & 13.502 & 4.935 & 2.083 & 2018AJ....156..217N  \\
03205178-6351524 & 0.33 & 1.029 & -5.6288 & 0.0096 & 13.433 & 5.238 & 91.622 & 2018AJ....156..217N  \\
07401183-4257406 & 0.15 & 0.002 & -4.3365 & 0.0022 & 13.81 & 6.042 & 0.307 & 2018AJ....156..217N  \\
08184619-4806172 & 0.37 & 0.021 & -4.2834 & 0.0025 & 14.37 & 5.019 & 1.653 & 2018AJ....156..217N  \\
08443891-4805218 & 0.20 & 1.348 & -5.6682 & 0.0067 & 13.932 & 5.370 & 129.513 & 2018AJ....156..217N  \\
09342791-2643267 & 0.19 & 0.007 & -4.3415 & 0.0025 & 13.992 & 5.373 & 0.694 & 2018AJ....156..217N  \\
09524176-1536137 & 0.26 & 1.342 & -5.6319 & 0.0110 & 13.43 & 4.923 & 99.662 & 2018AJ....156..217N  \\
11075025-3421003 & 0.25 & 0.068 & -4.2250 & 0.0032 & 15.04 & 5.633 & 7.611 & 2018AJ....156..217N  \\
11575352-2349007 & 0.39 & 0.031 & -4.2952 & 0.0026 & 14.77 & 5.415 & 3.067 & 2018AJ....156..217N  \\
12102834-1310234 & 0.36 & 0.435 & -4.6892 & 0.0029 & 13.83 & 5.418 & 42.985 & 2018AJ....156..217N  \\
12440075-1110302 & 0.18 & 0.020 & -4.4053 & 0.0033 & 14.22 & 5.546 & 2.099 & 2018AJ....156..217N  \\
13442092-2618350 & 0.35 & 2.032 & -5.9634 & 0.0253 & 13.253 & 4.968 & 154.885 & 2018AJ....156..217N  \\
14253413-1148515 & 0.51 & 0.301 & -4.7641 & 0.0030 & 13.512 & 5.121 & 25.012 & 2018AJ....156..217N  \\
14340491-1824106 & 0.38 & 0.271 & -4.6093 & 0.0038 & 14.346 & 5.638 & 30.396 & 2018AJ....156..217N  \\
15154371-0725208 & 0.38 & 0.050 & -4.6214 & 0.0023 & 12.93 & 5.224 & 4.379 & 2018AJ....156..217N  \\
15290145-0612461 & 0.46 & 0.095 & -4.2015 & 0.0017 & 14.011 & 5.230 & 8.434 & 2018AJ....156..217N  \\
16204186-2005139 & 0.45 & 0.031 & -4.3900 & 0.0035 & 13.68 & 5.261 & 2.814 & 2018AJ....156..217N  \\
16475517-6509116 & 0.17 & 0.889 & -4.8744 & 0.0045 & 13.98 & 5.101 & 73.142 & 2018AJ....156..217N  \\
20091824-0113377 & 0.15 & 0.010 & -4.3772 & 0.0023 & 14.47 & 5.958 & 1.374 & 2018AJ....156..217N  \\
20273733-5452592 & 0.35 & 1.520 & -5.9982 & 0.0181 & 13.18 & 5.259 & 136.924 & 2018AJ....156..217N  \\
20444800-1453208 & 0.49 & 0.073 & -4.4912 & 0.0023 & 14.445 & 5.305 & 6.715 & 2018AJ....156..217N  \\
15404341-5101357 & 0.10 & 0.318 & -5.0062 & 0.0081 & 15.26 & 7.317 & 93.702 & 2018AJ....156..217N  \\
22480446-2422075 & 0.20 & 0.005 & -4.4123 & 0.0016 & 12.59 & 5.384 & 0.466 & 2013AJ....146..154M  \\
06393742-2101333 & 0.26 & 0.952 & -5.2524 & 0.0069 & 12.77 & 5.120 & 79.152 & 2018AJ....156..217N  \\
04130560+1514520 & 0.30 & 0.019 & -4.4775 & 0.0088 & 15.881 & 5.437 & 1.881 & 2016ApJ...818..46M  \\
02411510-0432177 & 0.20 & 0.004 & -4.4272 & 0.0016 & 13.79 & 5.544 & 0.400 & 2020ApJ...905..107M  \\
  11381671-7721484 & 0.12 & 0.958 & \textbf{-5.5015} & 0.0369 & 14.78 & 6.259 & 153.506 & This Work  \\
  12384914-3822527 & 0.15 & 2.527 & \textbf{-6.0690} & 0.0156 & 12.75 & 5.364 & 241.913 & This Work  \\
  13464102-5830117 & 0.48 & 1.340 & \textbf{-5.6977} & 0.0146 &  &  & 65.017 & This Work  \\
  15165576-0037116 & 0.31 & 0.157 & \textbf{-4.0704} & 0.0024 & 14.469 & 5.364 & 15.028 & This Work  \\
  19204795-4533283 & 0.18 & 1.706 & \textbf{-5.8392} & 0.0091 & 12.25 & 5.405 & 167.225 & This Work  \\
  21362532-4401005 & 0.20 & 1.886 & \textbf{-5.8978} & 0.0168 & 14.14 & 5.610 & 207.983 & This Work  \\
\hline
\end{tabular}


% \begin{tabular}{lcccccc}
% \hline
% 	2MASS ID &  Mass &    $Ro$ &  $\log(R'_{HK})$ & $\log(R'_{HK})_{err}$ & $V_{mag}$ &   $V-K$ \\
% \hline
% \hline
% 06000351+0242236 &  0.237 &  0.020 &     -4.548 &          0.002 &  11.310 &  5.268 \\
% 02125458+0000167 &  0.268 &  0.048 &     -4.635 &          0.001 &  13.580 &  5.412 \\
% 01124752+0154395 &  0.278 &  0.026 &     -4.473 &          0.001 &  14.009 &  5.240 \\
% 10252645+0512391 &  0.111 &  0.000 &     -4.971 &          0.007 &  18.110 &  7.322 \\
% 05015746-0656459 &  0.168 &  0.873 &     -5.005 &          0.003 &  12.200 &  5.464 \\
% 06022261-2019447 &  0.234 &  1.307 &     -5.698 &          0.012 &  13.260 &  4.886 \\
% 06105288-4324178 &  0.295 &  0.705 &     -5.251 &          0.008 &  12.280 &  4.968 \\
% 09442373-7358382 &  0.240 &  0.542 &     -5.603 &          0.006 &  15.170 &  5.795 \\
% 14211512-0107199 &  0.238 &  1.160 &     -5.585 &          0.008 &  13.120 &  5.027 \\
% 14294291-6240465 &  0.119 &  0.394 &     -5.005 &          0.001 &  11.130 &  6.746 \\
% 16352464-2718533 &  0.228 &  1.423 &     -5.596 &          0.006 &  14.180 &  5.182 \\
% 16570570-0420559 &  0.242 &  0.014 &     -4.307 &          0.001 &  12.250 &  5.130 \\
% 02004725-1021209 &  0.343 &  0.188 &     -4.791 &          0.002 &  14.118 &  5.026 \\
% 18494929-2350101 &  0.175 &  0.034 &     -4.524 &          0.001 &  10.500 &  5.130 \\
% 20035892-0807472 &  0.328 &  0.946 &     -5.653 &          0.007 &  13.540 &  5.254 \\
% 21390081-2409280 &  0.209 &  1.152 &     -6.195 &          0.015 &  13.450 &  5.091 \\
% 23071524-2307533 &  0.303 &  0.720 &     -5.278 &          0.006 &  13.587 &  4.849 \\
% 00094508-4201396 &  0.304 &  0.009 &     -4.339 &          0.001 &  13.620 &  5.397 \\
% 00310412-7201061 &  0.311 &  0.906 &     -5.388 &          0.006 &  13.690 &  5.245 \\
% 01040695-6522272 &  0.171 &  0.006 &     -4.489 &          0.002 &  13.980 &  5.448 \\
% 02014384-1017295 &  0.193 &  0.034 &     -4.540 &          0.002 &  14.473 &  5.284 \\
% 03100305-2341308 &  0.395 &  0.028 &     -4.234 &          0.001 &  13.502 &  4.935 \\
% 03205178-6351524 &  0.330 &  1.029 &     -5.629 &          0.007 &  13.433 &  5.238 \\
% 07401183-4257406 &  0.154 &  0.002 &     -4.337 &          0.001 &  13.810 &  6.042 \\
% 08184619-4806172 &  0.370 &  0.021 &     -4.283 &          0.001 &  14.370 &  5.019 \\
% 08443891-4805218 &  0.202 &  1.348 &     -5.668 &          0.004 &  13.932 &  5.370 \\
% 09342791-2643267 &  0.192 &  0.007 &     -4.341 &          0.001 &  13.992 &  5.373 \\
% 09524176-1536137 &  0.264 &  1.342 &     -5.632 &          0.007 &  13.430 &  4.923 \\
% 11075025-3421003 &  0.255 &  0.068 &     -4.225 &          0.001 &  15.040 &  5.633 \\
% 11575352-2349007 &  0.393 &  0.031 &     -4.295 &          0.001 &  14.770 &  5.415 \\
% 12102834-1310234 &  0.355 &  0.435 &     -4.689 &          0.002 &  13.830 &  5.418 \\
% 12440075-1110302 &  0.184 &  0.020 &     -4.405 &          0.002 &  14.220 &  5.546 \\
% 13442092-2618350 &  0.348 &  2.032 &     -5.963 &          0.014 &  13.253 &  4.968 \\
% 14253413-1148515 &  0.505 &  0.301 &     -4.764 &          0.002 &  13.512 &  5.121 \\
% 14340491-1824106 &  0.377 &  0.271 &     -4.609 &          0.002 &  14.346 &  5.638 \\
% 15154371-0725208 &  0.378 &  0.050 &     -4.621 &          0.002 &  12.930 &  5.224 \\
% 15290145-0612461 &  0.455 &  0.095 &     -4.201 &          0.001 &  14.011 &  5.230 \\
% 16204186-2005139 &  0.453 &  0.031 &     -4.390 &          0.002 &  13.680 &  5.261 \\
% 16475517-6509116 &  0.170 &  0.889 &     -4.874 &          0.003 &  13.980 &  5.101 \\
% 20091824-0113377 &  0.147 &  0.010 &     -4.377 &          0.001 &  14.470 &  5.958 \\
% 20273733-5452592 &  0.350 &  1.520 &     -5.998 &          0.012 &  13.180 &  5.259 \\
% 20444800-1453208 &  0.485 &  0.073 &     -4.491 &          0.002 &  14.445 &  5.305 \\
% 15404341-5101357 &  0.098 &  0.318 &     -5.006 &          0.003 &  15.260 &  7.317 \\
% 22480446-2422075 &  0.198 &  0.005 &     -4.412 &          0.001 &  12.590 &  5.384 \\
% 06393742-2101333 &  0.258 &  0.952 &     -5.252 &          0.004 &  12.770 &  5.120 \\
% 04130560+1514520 &  0.298 &  0.019 &     -4.477 &          0.004 &  15.881 &  5.437 \\
% 02411510-0432177 &  0.197 &  0.004 &     -4.427 &          0.001 &  13.790 &  5.544 \\
% \hline
% \end{tabular}

% \begin{tabular}{lccccc}
% \hline
% 2MASS ID &  Mass &    $Ro$ &  $\log(R'_{HK})$ &   $V_{mag}$ &   V-K \\
% \hline
% \hline
% J00094508-4201396 &  0.30 &  0.01 &      -4.33 &  13.62 &  5.40 \\
% J00310412-7201061 &  0.31 &  0.91 &      -5.36 &  13.69 &  5.24 \\
% J01040695-6522272 &  0.17 &  0.01 &      -4.47 &  13.98 &  5.45 \\
% J01124752+0154395 &  0.28 &  0.03 &      -4.45 &  14.01 &  5.24 \\
% J02014384-1017295 &  0.19 &  0.03 &      -4.53 &  14.47 &  5.28 \\
% J02125458+0000167 &  0.27 &  0.05 &      -4.63 &  13.58 &  5.41 \\
% J03100305-2341308 &  0.40 &  0.03 &      -4.21 &  13.50 &  4.94 \\
% J03205178-6351524 &  0.33 &  1.03 &      -5.60 &  13.43 &  5.24 \\
% J05015746-0656459 &  0.17 &  0.87 &      -4.98 &  12.20 &  5.46 \\
% J06000351+0242236 &  0.24 &  0.02 &      -4.53 &  11.31 &  5.27 \\
% J06105288-4324178 &  0.30 &  0.71 &      -5.21 &  12.28 &  4.97 \\
% J06105288-4324178 &  0.30 &  0.71 &      -5.21 &  12.28 &  4.97 \\
% J06393742-2101333 &  0.26 &  0.95 &      -5.21 &  12.77 &  5.12 \\
% J06393742-2101333 &  0.26 &  0.95 &      -5.21 &  12.77 &  5.12 \\
% J07401183-4257406 &  0.15 &  0.00 &      -4.28 &  13.81 &  6.04 \\
% J08184619-4806172 &  0.37 &  0.02 &      -4.22 &  14.37 &  5.02 \\
% J08443891-4805218 &  0.20 &  1.35 &      -5.59 &  13.93 &  5.37 \\
% J09342791-2643267 &  0.19 &  0.01 &      -4.31 &  13.99 &  5.37 \\
% J09524176-1536137 &  0.26 &  1.34 &      -5.48 &  13.43 &  4.92 \\
% J11075025-3421003 &  0.25 &  0.07 &      -4.21 &  15.04 &  5.63 \\
% J11575352-2349007 &  0.39 &  0.03 &      -4.28 &  14.77 &  5.41 \\
% J12102834-1310234 &  0.36 &  0.44 &      -4.60 &  13.83 &  5.42 \\
% J12440075-1110302 &  0.18 &  0.02 &      -4.35 &  14.22 &  5.55 \\
% J13442092-2618350 &  0.35 &  2.03 &      -5.74 &  13.25 &  4.97 \\
% J14211512-0107199 &  0.24 &  1.16 &      -5.43 &  13.12 &  5.03 \\
% J14253413-1148515 &  0.51 &  0.30 &      -4.75 &  13.51 &  5.12 \\
% J14294291-6240465 &  0.12 &  0.39 &      -5.00 &  11.13 &  6.75 \\
% J14340491-1824106 &  0.38 &  0.27 &      -4.56 &  14.35 &  5.64 \\
% J15154371-0725208 &  0.38 &  0.05 &      -4.58 &  12.93 &  5.22 \\
% J15290145-0612461 &  0.46 &  0.10 &      -4.44 &  14.01 &  5.23 \\
% J16204186-2005139 &  0.45 &  0.03 &      -4.32 &  13.68 &  5.26 \\
% J16204186-2005139 &  0.45 &  0.03 &      -4.32 &  13.68 &  5.26 \\
% J16352464-2718533 &  0.23 &  1.42 &      -5.46 &  14.18 &  5.18 \\
% J16360563+0848491 &  0.22 &  0.07 &      -3.93 &  13.81 &  5.30 \\
% J16400599+0042188 &  0.18 &  0.00 &      -4.35 &  13.70 &  5.49 \\
% J16570570-0420559 &  0.24 &  0.01 &      -4.28 &  12.25 &  5.13 \\
% J16570570-0420559 &  0.24 &  0.01 &      -4.28 &  12.25 &  5.13 \\
% J18494929-2350101 &  0.18 &  0.03 &      -4.52 &  10.50 &  5.13 \\
% J20035892-0807472 &  0.33 &  0.95 &      -5.65 &  13.54 &  5.25 \\
% J20091824-0113377 &  0.15 &  0.01 &      -4.37 &  14.47 &  5.96 \\
% J20444800-1453208 &  0.49 &  0.07 &      -4.46 &  14.44 &  5.30 \\
% J21390081-2409280 &  0.21 &  1.15 &      -6.16 &  13.45 &  5.09 \\
% J22480446-2422075 &  0.20 &  0.00 &      -4.39 &  12.59 &  5.38 \\
% J22480446-2422075 &  0.20 &  0.00 &      -4.39 &  12.59 &  5.38 \\
% J23071524-2307533 &  0.30 &  0.72 &      -5.28 &  13.59 &  4.85 \\
% J23071524-2307533 &  0.30 &  0.72 &      -5.28 &  13.59 &  4.85 \\
% J23532520-7056410 &  0.26 &  0.01 &      -4.31 &  13.01 &  5.23 \\
% \hline
% \end{tabular}

  \caption{Calculated Rossby Numbers and $R'_{HK}$ values. All circular data
  points in Figures \ref{fig:RpHKvsRossbySelf} \& \ref{fig:RpHKvsRossbyDef} are
  present in this table. Masses are taken from the MEarth database. A machine
  readable version of this table is available. Rows where the activity metric
  is in bold face were estimates derived from our model fit not empirical
  measurements.}
    \label{tab:finalData}
\end{table}
\begin{figure*}
    \centering
    \includegraphics[width=0.9\textwidth]{figures/magActivity/RpHKvsR0_MC_justThisPaper.pdf}
	\caption{Rotation activity relation from this work. The color axis gives
	each stars mass. The dashed line is the best fit to our data set.}
    \label{fig:RpHKvsRossbySelf}
\end{figure*}
\begin{figure*}
    \centering
    \includegraphics[width=0.9\textwidth]{figures/magActivity/RpHKvsR0_MC.pdf}
	\caption{Rotation activity relation for both our work and \citet{Def17}.
	The dotted line is the best fit to the re-derived rotation-activity
	relation from \citet{Def17}.  Note that targets from \citet{Def17} are
	systematically higher than targets presented here as a consequence of the
	range in mass probed by the samples.}
    \label{fig:RpHKvsRossbyDef}
\end{figure*}
\begin{figure*}
    \centering
    \includegraphics[width=0.9\textwidth]{figures/magActivity/RpHKvsR0_MC_fits.pdf}
	\caption{Derived rotation-activity curves from this work, \citet{Def17} and
	\citet{Mamajek2008}. Note both that \citet{Mamajek2008} focuses their work
	on earlier spectral classes and fits the rotation activity relation in
	linear space.}
    \label{fig:RpHKvsRossbyFits}
\end{figure*}

We find best fit parameters with one $\sigma$ errors:
\begin{itemize}
    \item $k = -1.347\pm 0.203$
    \item $Ro_{s} =  0.155\pm0.045$
    \item $\log(R_{s}) = -4.436\pm0.048$

\end{itemize}
A comparison of the rotation activity derived in this work to those
from both \citet{Def17} and \citet{Mamajek2008} is presented in Figure
\ref{fig:RpHKvsRossbyFits}. For the 6 targets which do not have measured
rotational periods we include an estimate of $Ro$ and $p_{rot}$ in the machine
readable version of Table \ref{tab:finalData}. The convective overturn
timescale for one of these 6 targets (2MASS J13464102-5830117) can not be
inferred via Equation \ref{eqn:convectiveOverturn} as it lacks a V-K color
measurement. Instead, we infer $\tau_{c}$ via \citet{Wri18} Equation 6 (this
paper Equation \ref{eqn:ConvectiveOverturnTimeMass}) using mass. Similar to our
manner of inferring $\tau_{c}$ via color, when inferring $\tau_c$ via mass, we
adopt the larger of the two antisymmetric errors from \citet{Wri18}.

\begin{equation}\label{eqn:ConvectiveOverturnTimeMass}
	\log_{10}(\tau_{c}) = 2.33\pm0.06 - 1.5\pm0.21\left(M/M_{\odot}\right) + 0.31\pm0.17\left(M/M_{\odot}\right)^{2}
\end{equation}

Note that $R'_{HK}$ for one of six of these targets (2MASS
J15165576-0037116) is consistent to within 1$\sigma$ of the saturated value;
therefore, the reported $Ro$ for this target should only be taken as an upper
bound. The remaining five targets have measured $R'_{HK}$ values consistent
with the unsaturated regime. Estimated periods are consistent with previous
constraints. Of the six stars, two were listed as non-detections in
\citet{Newton2018}, and the remaining four as uncertain (possible) detections.
Of the four classed as uncertain, 2MASS 12384914-3822527 and 2MASS
19204795-4533283 have candidate periods $>100$ days and non-detections of
H-alpha emission \citep{Hawley96}. These two stars and the two non-detections
have Ca II H\&K activity levels suggesting very long periods. 2MASS
13464102-5830117 has a candidate period of 45 days, and 2MASS 15165576-0037116
of 0.8 days, both consistent with their higher levels of Ca II H\&K emission.

As a test of the proposed weak correlation between activity and rotation in the
``saturated'' regime seen in some works \citep{Mamajek2008,
Reiners2014, Leh20, Med20} --- though not in others \citep{Wri11, Nunez2015,
Newton2017} ---   we fit a second model whose power law index is allowed to
vary at $Ro < Ro_{s}$. We find a saturated regime power law index of
$-0.052\pm0.117$, consistent with 0 to within 1$\sigma$. Moreover,
all other parameter for this model are consistent to within one $\sigma$ of the
nominal  parameters for the model where the index is constrained to 0 below
$Ro=Ro_{s}$. We can constrain the slope in the saturated
regime to be between -0.363 and 0.259 at the $3\sigma$ confidence level.
Ultimately, we adopt the most standard activity interpretation, a
fully-saturated regime at $Ro < Ro_{s}$. 

We investigate whether our lack of detection of a slope for $Ro <
Ro_{s}$ is due to the limited number of observations in that region when
compared to other works \citep[e.g.][93 targets $Ro < Ro_{s}$]{Med20} through
injection and recovery tests. We inject, fake, rotation-activity measurements
into the saturated regime with an a priori slope of -0.13 --- the same as in
\citeauthor{Med20}. These fake data are given a standard deviation equal to the
standard deviation of our residuals ($12\%$). We perform the same MCMC model
fitting to this new data set as was done with the original data set multiple
times, each with progressively more injected data, until we can detect the
injected slope to the three sigma confidence level. Ultimately, we need more
than 65 data points --- 43 more than we observed in the saturated regime --- to
consistently recover this slope. Therefore, given the spread of our data we
cannot detect slopes on the order of what has previously been reported in the
literature.

We observe a gap in rotational period over a comparable range to the
one presented in \citet{Newton2016} Figure 2. Namely, that M-dwarfs are
preferentially observed as either fast or slow rotators, with a seeming lack of
stars existing at mid rotational periods. This period gap manifests in the
Rossby Number and can be seen in Figure \ref{fig:RpHKvsRossbyDef} as a lack of
our targets near to the knee-point in the fit. This period gap likely
corresponds to that seen by \citet{Browning2010}, who found a paucity of M
dwarfs at intermediate activity levels in Ca II H\&K and note the similarity to
the Vaughn-Preston gap established in higher mass stars \citep{vaughan1980}.
\citet{Magaudda2020} also identify a double-gap in x-ray activity for stars in
the unsaturated regime; it is not clear that the gap we see is related. As a
consequence of this period gap, there exists a degeneracy in our data
between moving the knee-point and allowing the activity level to vary in the
saturated regime.  In the following, we adopt the model of a fully saturated
regime.

We wish to compare our best fit parameters to those derived in \citet{Def17};
however, the authors of that paper do not fit the knee-point of the
rotation-activity relation. They select the canonical value for the rotational
period separating the saturated regime from the unsaturated regime ($P_{rot,s}
= 10$ days) and use a fixed convective overturn timescale ($\tau_{c} = 70$
days). To make our comparison more meaningful we use the $P_{rot}$ and $V-K$
colors presented in \citet{Def17} to re-derive $Ro$ values using $\tau_{c}$
\citep{Wri18}. Doing this for all targets presented in \citet{Def17} Table 3
and fitting the same piecewise power law as before, we find best fit parameters
of $Ro_{s} = 0.17\pm0.04$, $\log(R_{s}) = -4.140\pm0.067$, and
$k=-1.43\pm0.21$. Compared to the best fit parameters for our data, $Ro_{s}$
and the unsaturated regime's index, $k$, are consistent to within one sigma,
while the saturated value, $R_{s}$, differs. 

The mass ranges of our respective samples explain the differences in saturation
values between our work and that of \citet{Def17}. Our work focuses on
mid-to-late M-dwarfs and includes no stars above a mass of $0.5$ M$_{\odot}$
(Figure \ref{fig:massDistribution}).  The strength of Ca II H\&K
emission is known to decrease as stellar mass decreases \citep{Schrijver1987,
Rauscher2006, Hou17}. As \citet{Rauscher2006} note, this is the opposite as the
trend seen in H-alpha; the latter primarily reflects the increasing length of
time that lower M dwarfs remain active and rapidly rotating \citep{West2015,
Newton2016}.

A mass dependence can be seen in Figure 10 in \citet{Def17}, consistent
with expectations from the literature. If we clip the data from \citet{Def17}
Table 3 to the same mass range as our data-set ($M_{*} < 0.5M_{\odot}$) and fit
the same function as above, we find that all best fit parameters are consistent
to within one sigma between the two data-sets. 

\begin{figure}
    \centering
    \includegraphics[width=0.85\textwidth]{figures/magActivity/B2020vsAD2016_Masses.pdf}
	\caption{Distribution of masses between our sample and the sample presented
	in \citet{Def17}. Note how the two studies have approximately the same
	sample sizes; however, our sample is more tightly concentrated at lower
	masses \textbackslash later spectral classes.}
    \label{fig:massDistribution}
\end{figure}

We also compare our best fit $Ro_{s}$ to both those derived in
\citet{Newton2017} using $H_{\alpha}$ as an activity measure and those derived
in \citep{ Wri18, Magaudda2020} using $L_{X}/L_{bol}$ as an activity measure.
Works using $L_{X}/L_{bol}$ identify a similar, yet not consistent to within
one sigma result for $Ro_{s}$; while, the value of $k$ we find here is
consistent between all four works. Therefore, we find similar results not only
to other work using the same activity tracer, but also a power-law slope that
is consistent with work using different tracers. 





\chapter{Jao Gap connection to Magnatism}
\section{INTRODUCTION}
Over the last half of the 19th and first decade of the 20th centuries Lane,
Ritter, and Emden codified the earliest mathematical model of stellar
structure, the polytrope (Equation \ref{eqn:polytrope}), in \textit{Gaskugeln}
(Gas Balls) \citep{Emden1907}.

\begin{align}\label{eqn:polytrope}
	\frac{d}{d\xi}\left(\xi^{2}\frac{d\theta}{d\xi}\right) = -\xi^{2}\theta^{n}
\end{align}

Where $\xi$ and $\theta$ are dimensionless parameterizations of radius and
temperature respectively, and $n$ is known as the polytropic index. Despite this
early work, it wasn't until the late 1930s and early 1940s that the full set of
equations needed to describe the structure of a steady state,
radially-symmetric, star (known as the equations of stellar structure) began to
take shape as proton-proton chains and the Carbon-Nitrogen-Oxygen cycle were,
for the first time, seriously considered as energy generation mechanisms
\citep{Cowling1966}. Since then, and especially with the proliferation of
computers in astronomy, the equations of stellar structure have proven
themselves an incredibly predictive set of models.  

There are currently many stellar structure codes \citep[e.g.][]{Dotter2008,
Kovetz2009, Paxton2011} which integrate the equations of stellar structure ---
in addition to equations of state and lattices of nuclear reaction rates ---
over time to track the evolution of an individual star. The Dartmouth Stellar
Evolution Program (DSEP) \citep{Chaboyer2001, Bjork2006, Dotter2008} is one
such, well tested, stellar evolution program.

Here we propose to model low-mass stars in both the local solar neighborhood
and in globular clusters using DSEP. This work will primarily extend our
understanding of stellar physics in two areas: the effects of chemically
self-consistency on stellar models and time evolution of the core-convective
instabilities which ultimatly are belived to result in the observed paucity of
stars at a Gaia G magnitude of $\sim$10. [NEED CITATIONS IN THIS PARAGRAPH]

Low mass stars form an important component of the stellar population, with
stars less than [MASS HERE] making up more than 70\% of stars in the galaxy
[CITE]. Moreover, due to their long lives, low-mass stars provide essential
constraints on ages of various stellar populations [CITE]. In globular
clusters, where all stars are coeval to one of a limited number of populations,
low mass stars provide the vast majority of constraints when fitting ischrones [CITE].
Additionally, stars around the fully-convective transition mass show
age-dependent core-convective instabilities [CITE].

\subsection{Globular Clusters}

Globular clusters in the local universe are primarly composed of old and
consequently low-mass stars. For decades, prevaling thought had it globular
clusters were composed of a single stellar population born from a preisten
interstellar medium. This was supported by visibly tight main sequences and
clear main sequence turn offs in optical CMDs \citep[Figure
\ref{fig:M3CMD}][]{Sandage1953}. These early studies either did not handel or
had very large photometric uncertanties and therefore they were unable to
discriminate beteween CMD features with small separations,

\begin{figure}
	\centering
	\includegraphics[width=0.75\textwidth]{src/Figures/Gould53.png}
	\caption{$m_{pg}$ - $m_{pv}$ color-magnitude diagram for the globular cluster M3.}
	\label{fig:M3CMD}
\end{figure}

[SOMETHING ABOUT EARLY SPECTROSCOPIC INDICATIONS OF MPs]

With the presicion photometric measurements, degenerecies between noise and
intrinsic scatter were broken and it became clear that globular clusters are
almost universally composed of multiple stellar populations (MPs). [GRAB SOME
TEXT FROM THE NGC 2808 SECTION FOR HERE].

\subsection{Local Solar Neighborhood}
\citet{Jao2018} discovered a novel feature in the Gaia Gp-Rp color-magnitude
diagram. Around $M_{G}=10$ there is an approximatly 17\% decreas in stellar
density of the volume complete sample of stars \citeauthor{Jao2018} considered.
Subsequently, this has become known as either the Jao Gap, or Gaia M dwarf Gap.
Section \ref{} will go into more detail regarding the physics belived to
underpin this feature; however, in brief convective instabilities in the core
are belived to form for stars straddeling the fully convective transition mass.
These instabilities result in stars preferentially falling to either side of
the gap location.

Stellar modeling has been sucsessful in reproducing the Jap Gap and, with these
models, we have begun to constrain parameters which constrain gap location. For
example, it is now well documented that a stars metallicity can affect the gap
color by up to [HOW MUCH DID GREG FIND/CHECK FOR OTHER PAPERS ON THIS]. 

Initial testting which we have done using DSEP along with work by [PAPER] also
indicated the Jao Gap's color sensitivity to age. We observe that as models age
the Jao Gap moves [DIRECTION OF MOVMENT IN MAG AND COLOR SPACE].

The OPAL opacity tables in particular are very widely used by current
generation stellar evolution programs (in addition to current generation
stellar model and isochrone grids). However, they are no longer the most up
date elemental opacities. Moreover, the generation mechanism for these tables,
a webform, is no longer reliably online.  Consequently, it makes sense to
transition to more modern opacity tables with a more stable generation
mechanism.

Here we will present work transitioning DSEP from OPAL opacities to opacities
based on measurements from Los Alamos national Labs T-1 group
\citep[OPLIB][]{Colgan2016}. Moreover, we will present two projects which are
in large part reliant on these updated opacities. For the first project we
investigate the affects of chemically self consistent modeling of multiple
populations within the globular cluster NGC 2808, and for the second project we
present the effects of the OPLIB opacities on the location of the recently
discovered Gaia M-dwarf gap.

This paper is organized as follows. In Section \ref{sec:opac} we outline some
basic information about OPLIB opacities, how we query them, and how we modify
them to work with DSEP. In Section \ref{sec:2808} we discuss scientific
background of the first project along with the current work done towards its
goal. Finally, in Section \ref{sec:Jao} we present our findings on the effects
of OPLIB opacities on the location of the Gaia M-dwarf gap.






\section{Results}\label{sec:results}
Using \fidanka we fit pairs of Population A + E isochrones to the HUGS data for
NGC 2808. Each pair of isochrones is allowed to vary in distance modulus,
reddening, relative helium mass fraction (A/E), and age. Any population pairs which vary by more than 1\% in distance modulus or B-V color excess are rejected. The $\chi^{2}$
distribution for the isochrone pairs is shown in Figure {\color{red}[FIGURE]}.
The best fit isochrones are shown in Figure \ref{fig:BestFitResults} and optimized
parameters for these are presented in Table {\color{red}[TABLE]}.

{\color{red} Need to make the chi2 dist plot still. Have all these values but need to figure out best way to visualize it}

\begin{figure}
  \centering
  \includegraphics[width=0.45\textwidth]{figures/ngc2808/BestFitResults.pdf}
  \label{fig:BestFitResults}
  \caption{Best fit isochrone results for NGC 2808.}
\end{figure}

\begin{table*}
  \centering
  \begin{tabular}{c | c c c c c c}
    \hline
    population & age & distance modulus & extinction & Y & $\alpha_{ML}$ & $\chi^{2}_{\nu}$\\
    & [Gyr] & & mag & & &\\
    \hline
    \hline
    A & 12.3 & 14.91 & 0.54 & 0.24 & 1.901 & 0.014\\
    E & 14.3 & 14.96 & 0.54 & 0.39 & 1.750 & 0.017 \\
    \hline
  \end{tabular}
  \label{tab:BestFitResults}
  \caption{Best fit parameters derived from fitting isochrones to the fiducual lines derived from the NCG 2808 photometry.}
\end{table*}

{\color{blue} Currently are still seeing a discontinutiy in the isochrobne below the MSTO. This must be addressed before submission.}

\subsection{The Number of Populartions in NGC 2808}
\fidanka provides a somewhat straigtforward way to estimate the number of populations expected in a given magnitude bin given the observations. See Section \ref{sec:fidanka} for specific implimentaiton details. Here we preform an analysis of the number of populations seen in the NGC 2808 F814W-F274W vs F814W color-magnitude diagram. We find that for the majority of the main sequence and red giant branches BGMM prefers two populations; wherease, near the main sequnce turn off and on the majority of the subgiant branches BGMM prefers a single population model.

{\color{red}[FIGURE SHOWING BGMM population probability]}


\section{Modeling}\label{sec:modeling}
One of the most pressing questions related to this work is whether or not the
increased star-to-star variability in the activity metric and the Jao Gap,
which are coincident in magnitude, are driven by the same underlying mechanism.
The challenge when addressing this question arises from current computational
limitations. Specifically, the kinds of three dimensional
magneto-hydrodynamical simulations --- which would be needed to derive the
effects of convective kissing instabilities on the magnetic field of the star
--- are infeasible to run over gigayear timescales while maintaining thermal
timescale resolutions needed to resolve periodic mixing events.

In order to address this and answer the specific question of \textit{could
kissing instabilities result in increased star-to-star variability of the
magnetic field}, we adopt a very simple toy model. Kissing instabilities result
in transient radiative zone separating the core of a star (convective) from its
envelope (convective). When this radiative zone breaks down two important
things happen: one, the entire star becomes mechanically coupled, and two,
convective currents can now move over the entire radius of the star.
\citet{Jao2023} propose that this mechanical coupling may allow the stars core
to act as an angular momentum sink thus accelerating a stars spin down and
resulting in anomalously low H$\alpha$ emission. 

Regardless of the exact mechanism by which the magnetic field may be effected,
it it reasonable to expect that both the mechanical coupling and the change to
the scale of convective currents will have some effect on the stars magnetic
field. On a microscopic scale both of these will change how packets of charge
within a star move and may serve to disrupt a stable dynamo. Therefore, in the
model we present here we make only one primary assumption: \textit{every mixing
event may modify the stars magnetic field by some amount}. Within our model
this assumption manifests as a random linear perturbation applied to some base
magnetic field at every mixing event. The strength of this perturbation is 
sampled from a normal distribution with some standard deviation, $\sigma_{B}$.

Synthetic stars are sampled from a grid of stellar models evolved using the
Dartmouth Stellar Evolution Program (DSEP). Each stellar model was evolved
using a high temporal resolution (timesteps no larger than 10,000 years
{\color{red} Check this}) and typical numerical tolerances of one part in
$10^5$. Each model was based on a GS98 \citep{Grevesse1998} solar
composition with a mass range from 0.3 M$_{\odot}$ to 0.4 M$_{\odot}$. Finally,
models adopt OPLIB high temperature radiative opacities, Ferguson 2004 low
temperature radiative opacities, and include both atomic diffusion and
gravitational settling. A Kippenhan-Iben diagram showing the structural
evolution of a model within the gap is shown in Figure \ref{fig:kippenhan}.

\begin{figure*}
  \centering
  \includegraphics[width=0.9\textwidth]{figures/jaoMagActivity/Kippenhan_clamped.pdf}
  \caption{Kippenhan-Iben diagram for a 0.345 solar mass star. Note the
  periodic mixing events (where the plotted curves peak).}
  \label{fig:kippenhan}
\end{figure*}

Each synthetic star is assigned some base magnetic activity ($B_{0} \sim
\mathcal{N}(1, \sigma_{B})$) and then the number of mixing events before some age $t$
are counted based on local maxima in the core temperature. The toy magnetic
activity at age $t$ for the model is given in Equation \ref{eqn:activity}. An
example of the magnetic evolution resulting from this model is given in Figure
\ref{fig:simpleB}. Fundamentally, this model presents magnetic
activity variation due to mixing events as a random walk and therefore results will
increasingly divergence over time.

\begin{align}\label{eqn:activity}
  B(t) = B_{0} + \sum_{i}B_{i} \sim \mathcal{N}(1, \sigma_{B}) 
\end{align}

\begin{figure}
  \centering
  \includegraphics[width=0.45\textwidth]{figures/jaoMagActivity/simpleBEvolution.pdf}
  \caption{Example of the toy model presented here resulting in increased
  divergence between stars magnetic fields. The shaded region represents the
  maximum spread in the two point correlation function at each age.}
  \label{fig:simpleB}
\end{figure}

Applying the same analysis to these models as was done to the observations as
described in Section {\color{red} X.X} we find that this simple model results
in a qualitatively similar trend in the standard deviation vs. Magnitude graph
(Figure \ref{fig:model}). In order to reproduce the approximately 50 percent
change to the spread of the activity metric observed in the combined dataset in
section \ref{sec:results} a distribution with a standard deviation of 0.1 is required when sampling the change in the magnetic activity metric at each mixing event. This corresponds to 68\% of mixing events modifying the activity strength by 10 percent or less. The interpretation here is important, what
this qualitative similarity demonstrates is that it may be reasonable to expect
kissing instabilities to result in the observed increased star-to-star
variation. Importantly, we are not able to claim that kissing instabilities
\textit{do} lead to these increased variations, only that they reasonably
could. Further modeling, observational, and theoretical efforts will be needed
to more definitively answer this question.

\begin{figure}
  \centering
  \includegraphics[width=0.45\textwidth]{figures/jaoMagActivity/SpreadModel.pdf}
  \caption{Toy model results showing a qualitatively similar discontinuity in the star-to-star magnetic activity variability.}
  \label{fig:model}
\end{figure}

\subsection{Limitations}
The model presented in this paper is very limited and it is important to keep
those limitations in mind when interpreting the results presented here. Some of
the main challenges which should be leveled at this model are the assumption
that the magnetic field will be altered by some small random perturbation at
every mixing event. This assumption was informed by the large number of free
parameters available to a physical star during the establishment of a large scale 
magnetic field and the associated likely stochastic nature of that process.
However, it is similarly believable that the magnetic field will tend to alter in
a uniform manner at each mixing event. For example, since differential rotation
is generally proportional to the temperature gradient within a star and activity is
strongly coupled to differential rotation then it may be that as the radiative zone reforms over thermal timescales the homogenization of angular momentum throughout the star results in overall lower amounts of differential rotation each after mixing event than would otherwise be present.

Moreover, this model does not consider how other degenerate sources of magnetic evolution such as stellar spin down, relaxation, or coronal heating may effect star-to-star variability. These could conceivably lead to a similar increase in star-to-star variability which is coincident with the Jao Gap magnitude as the switch from fully to partially convective may effect efficiency of these process.

Additionally, there are challenges with this toy model that originate from the stellar evolutionary model. Observations of the Jao Gap show that the feature is not perpendicular to the magnitude axis; rather, it is inversely proportional to the color. No models of the Jao Gap published at the time of writing capture this color dependency and \textit{what causes this color dependency} remains one of the most pressing questions relating to the underlying physics. This non captured physics is one potential explanation for why the magnitude where our model predicts the increase in variability is not in agreement with where the variability jump exists in the data.

Finally, we have not considered detailed descriptions of the dynamos of stars. The magnetohydrodynamical modeling which would be required to model the evolution of the magnetic field of these stars at thermal timescale resolutions over gigayears is currently beyond the ability of practical computing. Therefore future work should focus on limited modeling which may inform the evolution of the magnetic field directly around the time of a mixing event.

\section{Conclusion}\label{sec:2808conclusion}
Here we have preformed the first chemically self-consistent modeling of the
Milky Way Globular Cluster NGC 2808. We find that, updated atmospheric boundary
conditions and opacity tables do not have a significant effect on the inferred
helium abundances of multiple populations. Specifically, we find that
population  has a helium mass fraction of 0.24, while population E has a helium
mass fraction of 0.39. Additionally, we find that the ages of these two populations 
agree within uncertainties. We only find evidence for two distinct stellar
populations, which is in agreement with recent work studying the number
of populations in NGC 2808 spectroscopic data.

We introduce a new software suite for globular cluster science,
\fidanka, which has been released under a permissive open source license.
\fidanka aims to provide a statistically robust set of tools for estimating the
parameters of multiple populations within globular clusters.



\part{Individual Stars}
\chapter{Magnetic Fields In M Dwarfs}
\section{Magnetic Activity in M dwarfs} \label{sec:magActivity-intro}
M-dwarfs are the most numerous stars in our galaxy; however, spun-up M-dwarfs
are more magnetically active when compared to larger and hotter stars
\citep{Haw91, Del98, Sch14}. The increase in activity may accelerate the
stripping of an orbiting planet's atmosphere \citep[e.g.][]{Owe16}, and may
dramatically impact habitability \citep{Shi16}. Therefore, it is essential to
understand the magnetic activity of M-dwarfs in order to constrain the
potential habitability and history of the planets that orbit them.
Additionally, rotation and activity may impact the detectability of hosted
planets \citep[e.g.][]{Rob14, Newton2016, Van16}.

Robust theories explaining the origin of solar-like magnetic fields exist and
have proven extensible to other regions of the main sequence \citep{Cha14}. The
classical $\alpha\Omega$ dynamo relies on differential rotation between layers
of a star to stretch a seed poloidal field into a toroidal field \citep{Par55,
Cam17}. Magnetic buoyancy causes the toroidal field to rise through the star.
During this rise, turbulent helical stretching converts the toroidal field back
into a poloidal field \citep{Par55}. Seed fields may originate from the
stochastic movement of charged particles within a star's atmospheres.

In non-fully convective stars the initial conversion of the toroidal field to a
poloidal field is believed to take place at the interface layer between the
radiative and convective regions of a star --- the tachocline \citep{Noy84,
Tom96, Dik99}. The tachocline has two key properties that allow it to play an
important role in solar type magnetic dynamos: 1), there are high shear
stresses, which have been confirmed by astroseismology \citep{Tho96}, and 2),
the density stratification between the radiative and convective zones serves to
``hold'' the newly generated toroidal fields at the tachocline for an extended
time. Over this time, the fields build in strength significantly more than they
would otherwise \citep{Par75}. This theory does not trivially extend to
mid-late M-dwarfs, as they are believed to be fully convective and consequently
do not contain a tachocline \citep{Cha97}. Moreover, fully convective M-dwarfs
are not generally expected to exhibit internal differential rotation
\citep[e.g.][]{barnes2004differential, barnes2005dependence}, though, some
models do produce it \citep{Yad13}.

Currently, there is no single accepted process that serves to build and
maintain fully convective M-dwarf magnetic fields in the same way that the
$\alpha$ and $\Omega$ processes are presently accepted in solar magnetic dynamo
theory. Three-dimensional magneto anelastic hydrodynamical simulations have
demonstrated that local fields generated by convective currents can self
organize into large scale dipolar fields. 
%These fields are similar to smaller scale structure which have been observed through Zeeman-Doppler imaging \citep{Brown2011, Yad15}
These models indicate that for a fully convective star to sustain a magnetic
field it must have a high degree of density stratification --- density
contrasts greater than 20 at the tachocline --- and a sufficiently large magnetic Reynolds
number\footnote{The Reynolds Number is the ratio of magnetic induction to
magnetic diffusion; consequently, a plasma with a larger magnetic Reynolds
number will  sustain a magnetic field for a longer time than a plasma with a
smaller magnetic Reynolds number.}.

An empirical relation between the rotation rate and the level of magnetic
activity has been demonstrated in late-type stars \citep{Skumanich1972, Pal81}. This is
believed to be a result of faster rotating stars exhibiting excess non-thermal
emission from the upper chromosphere or corona when compared to their slower
rotating counterparts. This excess emission is due to magnetic heating of the
upper atmosphere, driven by the underlying stellar dynamo.
The faster a star rotates, up to some saturation threshold, the more such emission is expected. However,
the dynamo process is not dependent solely on rotation; rather, it depends on
whether the contribution from the rotational period ($P_{rot}$) or convective
motion --- parameterized by the convective overturn time scale ($\tau_{c}$) ---
dominates the motion of a charge packet within a star. Therefore, the Rossby
Number ($Ro = P_{rot}/\tau_{c}$) is often used in place of the rotational
period as it accounts for both.

The rotation-activity relation was first discovered using the ratio of X-ray
luminosity to bolometric luminosity ($L_{X}/L_{bol}$) \citep{Pal81} and was
later demonstrated to be a more general phenomenon, observable through other
activity tracers, such as Ca II H\&K emission \citep{Vilhu1984}. This relation has
a number of important structural elements. \citet{Noy84} showed that magnetic
activity as a function of Rossby Number is well modeled as a piecewise power
law relation including a saturated and non-saturated regime. In the saturated
regime, magnetic activity is invariant to changes in Rossby Number; in the
non-saturated regime, activity decreases as Rossby Number increases. The
transition between the saturated and non-saturated regions occurs at $Ro \sim
0.1$ \citep[e.g.][]{Wri11}. Recent evidence may suggest that, instead of an
unsaturated region where activity is fully invariant to rotational period,
activity is more weakly, but still positively, correlated with rotation rate
\citep{Mamajek2008, Reiners2014, Leh20, Magaudda2020}. 

Previous studies of the Ca II H\&K rotation-activity relation
\citep[e.g.][]{Vau81, Sua15, Def17, Hou17} have focused on on spectral ranges
which both extend much earlier than M-dwarfs and which do not fully probe late
M-dwarfs. Other studies have relied on $v\sin(i)$ measurements
\citep[e.g.][]{Browning2010, Hou17}, which are not sensitive to the long
rotation periods reached by slowly rotating, inactive mid-to-late type M dwarfs
\citep[70-150 days:][]{Newton2016}. Therefore, these studies can present only
coarse constraints on the rotation activity relation in the fully convective
regime. The sample we present in this paper is focused on mid-to-late type M
dwarfs, with photometrically measured rotational periods, while maintaining of
order the same number of targets as previous studies.  Consequently, we provide
much finer constraints on the rotation-activity relation in this regime. 

One example of an application of the rotation-activity relation is as a means
of approximating stellar ages. Because as stars spin down, they move along
the rotation-activity relation \citep{Soderblom1991}. To calibrate this
relation, however, one needs a priori knowledge of a star's age and therefore
stars need to be in clusters where population statistics may be used to
accurately measure ages. This has proved doable for FGK type stars; however,
as is often the case, M-dwarfs pose some unique challenges. 

Firstly, the sample of all clusters in which we can observe M-dwarfs is
extremely small due to M-dwarfs' low luminosities. Secondly, open clusters
preferentially contain stars younger than the characteristic time it takes an
M-dwarf to spin down out of the saturated regime of the rotation-activity
relation \citep{West2009, Newton2016, Giacobbe2020}. Therefore, even in the
small set of clusters with measured ages and that contain observable
mid-to-late M-dwarfs, the unsaturated regimes in the rotation-activity
relation is not present. Currently, their has not been a successful
demonstration of using the M-dwarf rotation-activity relation to measure
ages.

We present a high resolution spectroscopic study of 53 mid-late M-dwarfs. We
measure Ca II H\&K strengths, quantified through the $R'_{HK}$ metric, which is
a bolometric flux normalized version of the Mount Wilson S-index. These
activity tracers are then used in concert with photometrically determined
rotational periods, compiled by \citet{Newton2017}, to generate a
rotation--activity relation for our sample. This paper is organized as follows:
Section \ref{sec:Observations} provides an overview of the observations and
data reduction, Section \ref{sec:Analysis} details the analysis of our data,
and Section \ref{sec:results} presents our results and how they fit within the
literature. 

\section{Observations \& Data Reduction}\label{sec:Observations}
We initially selected a sample of 55 mid-late M-dwarfs from targets of
the MEarth survey \citep{Ber12} to observe. Targets were selected based on high
proper motions and availability of a previously measured photometric rotation
period, or an expectation of a measurement based on data available from
MEarth-South at the time. These rotational periods were derived photometrically
\citep[e.g.][]{Newton2016,Man16,Med20}. For star 2MASS J06022261-2019447, which
was categorized as an ``uncertain detection'' from MEarth photometry by
\citet{Newton2018}, including new data from MEarth DR10 we find a period of 95
days. This value was determined following similar methodology to \citet{Irw11}
and \citet{Newton2016,Newton2018}, and is close to the reported candidate
period of 116 days.  References for all periods are provided in the machine
readable version of Table \ref{tab:finalData}.   

High resolution spectra were collected from March to October 2017 using the
Magellan Inamori Kyocera Echelle (MIKE) spectrograh on the 6.5 meter Magellan 2
telescope at the Las Campanas Observatory in Chile. MIKE is a high resolution
double echelle spectrograph with blue and red arms. Respectively, these cover
wavelengths from 3350 - 5000 \AA\ and 4900-9500 \AA\ \citep{Ber03}. We
collected data using a 0.75x5.00" slit resulting in a resolving power of 32700.
Each science target was observed an average of four times with mean integration
times per observation ranging from 53.3 to 1500 seconds. Ca II H\&K
lines were observed over a wide range of signa-to-noise ratios, from $\sim 5$ up
to $\sim 240$ with mean and median values of 68 and 61 respectively.

We use the \texttt{CarPy} pipeline \citep{Kel00, Kel03} to reduce our blue arm
spectra. \texttt{CarPy}'s data products are wavelength calibrated, blaze
corrected, and background subtracted spectra comprising 36 orders. We shift all
resultant target spectra into the rest frame by cross correlating against a
velocity template spectrum. For the velocity template we use an observation of
Proxima Centari in our sample. This spectrum's velocity is both barycentrically
corrected, using astropy's \texttt{SkyCoord} module \citep{Ast18}, and
corrected for Proxima Centari's measured radial velocity, -22.4 km s$^{-1}$
\citep{Tor06}. Each echelle order of every other target observation is cross
correlated against the corresponding order in the template spectra using
\texttt{specutils} \texttt{template\_correlate} function \citep{Nic21}.
Velocity offsets for each order are inferred from a Gaussian fit to the
correlation vs. velocity lag function. For each target, we apply a three sigma
clip to list of echelle order velocities, visually verifying this clip removed
low S/N orders. We take the mean of the sigma-clipped velocities Finally, each
wavelength bin is shifted according to its measured velocity.

Ultimately, two targets (2MASS J16570570-0420559 and 2MASS J04102815-5336078)
had S/N ratios around the Ca II H\&K lines which were too low to be of use,
reducing the number of R'$_{HK}$ measurement we can make from 55 to 53. 

\section{Analysis}\label{sec:Analysis}
Since the early 1960s, the Calcium Fraunhaufer lines have been used as
chromospheric activity tracers \citep{Wil63}. Ca II H\&K lines are observed as
a combination of a broad absorption feature originating in the upper
photosphere along with a narrow emission feature from non-thermal heating of
the upper chromosphere \citep{Catalano1983}. Specifically, the ratio between
emission in the Ca II H\&K lines and flux contributed from the photosphere is
used to define an activity metric known as the S-index \citep{Wil68}.   The
S-index increases with increasing magnetic activity. The S-index is defined as 

\begin{equation}\label{eqn:SIndex}
    S = \alpha \frac{f_{H} + f_{K}}{f_{V} + f_{R}}    
\end{equation}

\noindent where $f_{H}$  and $f_{K}$ are the integrated flux over triangular
passbands with a full width at half maximum of $1.09\text{ \AA}$ centered at
$3968.47\text{ \AA}$ and $3933.66\text{ \AA}$, respectively. The values of
$f_{V}$ and $f_{R}$ are integrated, top hat, broadband regions. They
approximate the continuum (Figure \ref{fig:SindexBandpass}) and are centered at
3901 \AA \ and 4001 \AA \ respectively, with widths of 20 \AA \ each. Finally,
$\alpha$ is a scaling factor with $\alpha = 2.4$.

\begin{figure*}[ht!]
    \centering
    \includegraphics[width=0.9\textwidth]{figures/magActivity/SIndexBandpass.pdf}
	\caption{Spectrum of 2MASS J06105288-4324178 overplotted with the S index
	bandpasses. (top) V band and Ca II K emission line. (bottom) Ca II H
	emission line and R band. Note that the rectangular and triangular
	regions denote both the wavelength range of the band and the relative
	weight assigned to each wavelength in the band while integrating. }
    \label{fig:SindexBandpass}
\end{figure*}

Following the procedure outlined in \citet{Lov11} we use the mean flux per
wavelength interval, $\tilde{f_{i}}$, as opposed to the integrated flux over
each passband when computing the S-index. This means that for each passband,
$i$, with a blue most wavelength $\lambda_{b,i}$ and a red most wavelength
$\lambda_{r,i}$, $\tilde{f}_{i}$ is the summation of the product of flux ($f$)
and weight ($w_{i})$ over the passband.

\begin{equation}\label{eqn:meanFlux}
    \tilde{f}_{i} = \frac{\sum_{l = \lambda_{b,i}}^{\lambda_{r,i}}f(l)w_{i}(l)}{\lambda_{r,i}-\lambda_{b,i}}    
\end{equation}
\noindent where $w_{i}$ represents the triangular passband for $f_{H}$ \& $f_{K}$ and the tophat for $f_{V}$ \& $f_{R}$.

Additionally, the spectrograph used at Mount Wilson during the development of
the S-index exposed the H \& K lines for eight times longer than the continuum
of the spectra. Therefore, for a modern instrument that exposes the entire
sensor simultaneously, there will be 8 times less flux in the Ca II H\&K
passbands than the continuum passbands than for historical observations. This
additional flux is accounted for by defining a new constant $\alpha_{H}$,
defined as:

\begin{equation}
    \alpha_{H} = 8\alpha\left(\frac{1.09\text{ \AA}}{20\text{ \AA}}\right)
\end{equation}
Therefore, S-indices are calculated here not based on the historical definition
given in Equation \ref{eqn:SIndex}; rather, the slightly modified version:

\begin{equation}\label{eqn:finalSIndex}
    S = \alpha_{H}\frac{\tilde{f}_{H} + \tilde{f}_{K}}{\tilde{f}_{V} + \tilde{f}_{R}}
\end{equation}

The S-index may be used to make meaningful comparisons between stars of similar
spectral class; however, it does not account for variations in photospheric
flux and is therefore inadequate for making comparisons between stars of
different spectral classes. The $R'_{HK}$ index \citep{Middelkoop1982} is a
transformation of the S-index intended to remove the contribution of the
photosphere. 

$R'_{HK}$ introduces a bolometric correction factor, $C_{cf}$, developed by
\citet{Middelkoop1982} and later improved upon by \citet{Rutten1984}.
Calibrations of $C_{cf}$ have focused on FGK-type stars using broad band color
indices, predominately B-V. However, these FGK-type solutions do not extend to
later type stars easily as many mid-late M-dwarfs lack B-V photometry.
Consequently, $C_{cf}$ based on B-V colors were never calibrated for M-dwarfs
as many M-dwarfs lack B and V photometry. \citet{SuarezMascareno2016} provided
the first $C_{cf}$ calibrations for M-dwarfs using the more appropriate color
index of $V-K$. The calibration was later extended by \citet{Def17}, which we
adopt here. 

Generally $R'_{HK}$ is defined as

\begin{equation}\label{eqn:RpHKDef}
    R'_{HK} = K\sigma^{-1}10^{-14}C_{cf}(S-S_{phot})
\end{equation}
where K is a factor to scale surface fluxes of arbitrary units into physical
units; the current best value for K is taken from \citet{Hal07},
$K=1.07\times10^{6}\text{erg cm$^{-2}$ s$^{-1}$}$. $S_{phot}$ is the
photospheric contribution to the S-index; in the spectra this manifests as the
broad absorption feature wherein the narrow Ca II H\&K emission resides.
$\sigma$ is the Stephan-Boltzmann constant. If we define 

\begin{equation}
    R_{phot}\equiv K\sigma^{-1}10^{-14}C_{cf}S_{phot}
\end{equation}
then we may write $R'_{HK}$ as 

\begin{equation}\label{eqn:RpHKFinal}
    R'_{HK} = K\sigma^{-1}10^{-14}C_{cf}S - R_{phot}.
\end{equation}

We use the color calibrated coefficients for $\log_{10}(C_{cf})$ and
$\log_{10}(R_{phot})$ presented in Table 1 of \citet{Def17}.

We estimate the uncertainty of $R'_{HK}$ as the standard deviation of a
distribution of $R'_{HK}$ measurements from 5000 Monte Carlo tests. For each
science target we offset the flux value at each wavelength bin by an amount
sampled from a normal distribution. The standard deviation of this normal
distribution is equal to the estimated error at each wavelength bin. These
errors are calculated at reduction time by the pipeline. \textbf{The R$'_{HK}$
uncertainty varies drastically with signal-to-noise; targets with
signal-to-noise ratios $\sim 5$ have typical uncertainties of a few percent
whereas targets with signal-to-noise ratios $\sim 100$ have typical
uncertainties of a few tenth of a percent.}

\subsection{Rotation and Rossby Number}
The goal of this work is to constrain the rotation activity relation;
therefore, in addition to the measured $R'_{HK}$ value, we also need the
rotation of the star. As mentioned, one of the selection criteria for targets
was that their rotation periods were already measured; however, ultimately
\textbf{6} of the 53 targets with acceptable S/N did not have well constrained
rotational periods. We therefore only use the remaining 47 targets to fit the
rotation-activity relation. 

% In order to make the most meaningful comparison possible we transform rotation
% period into Rossby Number . This transformation was done using the convective
% overturn timescale, $\tau_{c}$, such that the Rossby Number, $Ro =
% \frac{P_{rot}}{\tau_{c}}$ . To first order $\tau_{c}$ can be approximated as
% $70$ days for fully-convective M-dwarfs \citep{Pizzolato2000}. However,
% \citet{Wri18} Equation (5) presents an empirically calibrated expression for
% $\tau_{c}$. This calibration is derived by fitting the convective overturn
% timescale as a function of color index, in order to minimize the horizontal
% offset between stars of different mass in the rotation-activity relationship.
% The calibration from \citet{Wri18} that we use to find convective overturn
% timescales and subsequently Rossby numbers is:

In order to make the most meaningful comparison possible we transform rotation
period into Rossby Number . This transformation was done using the convective
overturn timescale, $\tau_{c}$, such that the Rossby Number, $Ro =
P_{rot}/\tau_{c}$ . To first order $\tau_{c}$ can be approximated as $70$ days
for fully-convective M-dwarfs \citep{Pizzolato2000}. However, \citet{Wri18}
Equation (5) presents an empirically calibrated expression for $\tau_{c}$. This
calibration is derived by fitting the convective overturn timescale as a
function of color index, in order to minimize the horizontal offset between
stars of different mass in the rotation-activity relationship.  The calibration
from \citet{Wri18} that we use to find convective overturn timescales and
subsequently Rossby numbers is:

\begin{equation}\label{eqn:convectiveOverturn}
    \log_{10}(\tau_{c}) = (0.64\pm0.12)+(0.25\pm0.08)(V-K)
\end{equation}
We adopt symmetric errors for the parameters of Equation
\ref{eqn:convectiveOverturn} equal to the larger of the two anti-symmetric
errors presented in \citet{Wri18} Equation 5. 

\section{Rotation--Activity Relation}\label{sec:results}
We show our rotation-activity relation in Figures
\ref{fig:RpHKvsRossbySelf} \& \ref{fig:RpHKvsRossbyDef}. Note that
errors are shown in both figures; however, they render smaller than the data
point size. Ca II H\&K is also known to be time variable
\citep[e.g.][]{Baroch2020,Perdelwitz2021}, which is not captured in our
single-epoch data. There is one target cut off by the domain of this graph,
2MASS J10252645+0512391. This target has a measured vsini of $59.5\pm2.1$ km
s$^{-1}$ \citep{Kesseli2018} and is therefore quite rotationally broadened, which
is known to affect $R'_{HK}$ measurements \citep[figure 8]{Schroder2009}. The
data used to generate this figure is given in Table \ref{tab:finalData}. Table
\ref{tab:finalData} includes uncertainties, the R'$_{HK}$ measurements for
stars which did not have photometrically derived rotational periods in MEarth,
and data for 2MASS J10252645+0512391

We find a rotation activity relationship qualitatively similar to that
presented in \citet{Def17}. Our rotation activity relationship exhibits both
the expected saturated and unsaturated regimes --- the flat region at $Ro <
Ro_{s}$ and the sloped region at $Ro \geq Ro_{s}$ respectively. We fit the
rotation activity relation given in Equation \ref{eqn:fitEqn} to our data using
Markov Chain Monte Carlo (MCMC), implemented in \texttt{pymc}
\citep{Salvatier2016}. 

  \begin{equation}\label{eqn:fitEqn}
      \log(R'_{HK}) = \begin{cases}
          \log(R_{s}) & Ro < Ro_{s} \\
          k\log(Ro) + \log(R_{s}) - k\log(Ro_{s}) & Ro \geq Ro_{s}
      \end{cases}
  \end{equation}

\noindent $Ro_{s}$ is the Rossby number cutoff between the saturated and
unsaturated regime. $R_{s}$ is the maximum, saturated, value of $R'_{HK}$ and
$k$ is the index of the power law when $Ro \geq Ro_{s}$. Due to the
issues measuring $R'_{HK}$ for high vsini targets discussed above, we exclude
2MASS J10252645+0512391 from this fit. All logarithms are base ten unless
another base is explicitly given.
\begin{table}[ht]
  \small
    \centering
    \setlength{\tabcolsep}{4pt}
    \begin{tabular}{lcccccccc}
\hline
2MASS ID & Mass & $Ro$ & $\log(R'_{HK})$ & $\log(R'_{HK})_{err}$ & $V_{mag}$ & $V-K$ & prot & $r_{prot}$\\
 & $\mathrm{M_{\odot}}$ &  &  &  & $\mathrm{mag}$ & $\mathrm{mag}$ & $\mathrm{d}$ &   \\
\hline
\hline
06000351+0242236 & 0.24 & 0.020 & -4.5475 & 0.0021 & 11.31 & 5.268 & 1.809 & 2016ApJ...821...93N  \\
02125458+0000167 & 0.27 & 0.048 & -4.6345 & 0.0014 & 13.58 & 5.412 & 4.732 & 2016ApJ...821...93N  \\
01124752+0154395 & 0.28 & 0.026 & -4.4729 & 0.0017 & 14.009 & 5.240 & 2.346 & 2016ApJ...821...93N  \\
10252645+0512391 & 0.11 & 0.000 & -4.9707 & 0.0380 & 18.11 & 7.322 & 0.102 & 2016ApJ...821...93N  \\
05015746-0656459 & 0.17 & 0.873 & -5.0049 & 0.0028 & 12.2 & 5.464 & 88.500 & 2012AcA....62...67K  \\
06022261-2019447 & 0.23 & 1.307 & -5.6980 & 0.0192 & 13.26 & 4.886 & 95.000 & This Work  \\
06105288-4324178 & 0.30 & 0.705 & -5.2507 & 0.0139 & 12.28 & 4.968 & 53.736 & 2018AJ....156..217N  \\
09442373-7358382 & 0.24 & 0.542 & -5.6026 & 0.0147 & 15.17 & 5.795 & 66.447 & 2018AJ....156..217N  \\
14211512-0107199 & 0.24 & 1.160 & -5.5846 & 0.0125 & 13.12 & 5.027 & 91.426 & 2018AJ....156..217N  \\
14294291-6240465 & 0.12 & 0.394 & -5.0053 & 0.0014 & 11.13 & 6.746 & 83.500 & 1998AJ....116..429B  \\
16352464-2718533 & 0.23 & 1.423 & -5.5959 & 0.0108 & 14.18 & 5.182 & 122.656 & 2018AJ....156..217N  \\
16570570-0420559 & 0.24 & 0.014 & -4.3071 & 0.0014 & 12.25 & 5.130 & 1.212 & 2012AcA....62...67K  \\
02004725-1021209 & 0.34 & 0.188 & -4.7907 & 0.0026 & 14.118 & 5.026 & 14.793 & 2018AJ....156..217N  \\
18494929-2350101 & 0.18 & 0.034 & -4.5243 & 0.0015 & 10.5 & 5.130 & 2.869 & 2007AcA....57..149K  \\
20035892-0807472 & 0.33 & 0.946 & -5.6530 & 0.0077 & 13.54 & 5.254 & 84.991 & 2018AJ....156..217N  \\
21390081-2409280 & 0.21 & 1.152 & -6.1949 & 0.0190 & 13.45 & 5.091 & 94.254 & 2018AJ....156..217N  \\
23071524-2307533 & 0.30 & 0.720 & -5.2780 & 0.0077 & 13.587 & 4.849 & 51.204 & 2018AJ....156..217N  \\
00094508-4201396 & 0.30 & 0.009 & -4.3392 & 0.0018 & 13.62 & 5.397 & 0.859 & 2018AJ....156..217N  \\
00310412-7201061 & 0.31 & 0.906 & -5.3879 & 0.0074 & 13.69 & 5.245 & 80.969 & 2018AJ....156..217N  \\
01040695-6522272 & 0.17 & 0.006 & -4.4889 & 0.0024 & 13.98 & 5.448 & 0.624 & 2018AJ....156..217N  \\
02014384-1017295 & 0.19 & 0.034 & -4.5400 & 0.0022 & 14.473 & 5.284 & 3.152 & 2018AJ....156..217N  \\
03100305-2341308 & 0.40 & 0.028 & -4.2336 & 0.0017 & 13.502 & 4.935 & 2.083 & 2018AJ....156..217N  \\
03205178-6351524 & 0.33 & 1.029 & -5.6288 & 0.0096 & 13.433 & 5.238 & 91.622 & 2018AJ....156..217N  \\
07401183-4257406 & 0.15 & 0.002 & -4.3365 & 0.0022 & 13.81 & 6.042 & 0.307 & 2018AJ....156..217N  \\
08184619-4806172 & 0.37 & 0.021 & -4.2834 & 0.0025 & 14.37 & 5.019 & 1.653 & 2018AJ....156..217N  \\
08443891-4805218 & 0.20 & 1.348 & -5.6682 & 0.0067 & 13.932 & 5.370 & 129.513 & 2018AJ....156..217N  \\
09342791-2643267 & 0.19 & 0.007 & -4.3415 & 0.0025 & 13.992 & 5.373 & 0.694 & 2018AJ....156..217N  \\
09524176-1536137 & 0.26 & 1.342 & -5.6319 & 0.0110 & 13.43 & 4.923 & 99.662 & 2018AJ....156..217N  \\
11075025-3421003 & 0.25 & 0.068 & -4.2250 & 0.0032 & 15.04 & 5.633 & 7.611 & 2018AJ....156..217N  \\
11575352-2349007 & 0.39 & 0.031 & -4.2952 & 0.0026 & 14.77 & 5.415 & 3.067 & 2018AJ....156..217N  \\
12102834-1310234 & 0.36 & 0.435 & -4.6892 & 0.0029 & 13.83 & 5.418 & 42.985 & 2018AJ....156..217N  \\
12440075-1110302 & 0.18 & 0.020 & -4.4053 & 0.0033 & 14.22 & 5.546 & 2.099 & 2018AJ....156..217N  \\
13442092-2618350 & 0.35 & 2.032 & -5.9634 & 0.0253 & 13.253 & 4.968 & 154.885 & 2018AJ....156..217N  \\
14253413-1148515 & 0.51 & 0.301 & -4.7641 & 0.0030 & 13.512 & 5.121 & 25.012 & 2018AJ....156..217N  \\
14340491-1824106 & 0.38 & 0.271 & -4.6093 & 0.0038 & 14.346 & 5.638 & 30.396 & 2018AJ....156..217N  \\
15154371-0725208 & 0.38 & 0.050 & -4.6214 & 0.0023 & 12.93 & 5.224 & 4.379 & 2018AJ....156..217N  \\
15290145-0612461 & 0.46 & 0.095 & -4.2015 & 0.0017 & 14.011 & 5.230 & 8.434 & 2018AJ....156..217N  \\
16204186-2005139 & 0.45 & 0.031 & -4.3900 & 0.0035 & 13.68 & 5.261 & 2.814 & 2018AJ....156..217N  \\
16475517-6509116 & 0.17 & 0.889 & -4.8744 & 0.0045 & 13.98 & 5.101 & 73.142 & 2018AJ....156..217N  \\
20091824-0113377 & 0.15 & 0.010 & -4.3772 & 0.0023 & 14.47 & 5.958 & 1.374 & 2018AJ....156..217N  \\
20273733-5452592 & 0.35 & 1.520 & -5.9982 & 0.0181 & 13.18 & 5.259 & 136.924 & 2018AJ....156..217N  \\
20444800-1453208 & 0.49 & 0.073 & -4.4912 & 0.0023 & 14.445 & 5.305 & 6.715 & 2018AJ....156..217N  \\
15404341-5101357 & 0.10 & 0.318 & -5.0062 & 0.0081 & 15.26 & 7.317 & 93.702 & 2018AJ....156..217N  \\
22480446-2422075 & 0.20 & 0.005 & -4.4123 & 0.0016 & 12.59 & 5.384 & 0.466 & 2013AJ....146..154M  \\
06393742-2101333 & 0.26 & 0.952 & -5.2524 & 0.0069 & 12.77 & 5.120 & 79.152 & 2018AJ....156..217N  \\
04130560+1514520 & 0.30 & 0.019 & -4.4775 & 0.0088 & 15.881 & 5.437 & 1.881 & 2016ApJ...818..46M  \\
02411510-0432177 & 0.20 & 0.004 & -4.4272 & 0.0016 & 13.79 & 5.544 & 0.400 & 2020ApJ...905..107M  \\
  11381671-7721484 & 0.12 & 0.958 & \textbf{-5.5015} & 0.0369 & 14.78 & 6.259 & 153.506 & This Work  \\
  12384914-3822527 & 0.15 & 2.527 & \textbf{-6.0690} & 0.0156 & 12.75 & 5.364 & 241.913 & This Work  \\
  13464102-5830117 & 0.48 & 1.340 & \textbf{-5.6977} & 0.0146 &  &  & 65.017 & This Work  \\
  15165576-0037116 & 0.31 & 0.157 & \textbf{-4.0704} & 0.0024 & 14.469 & 5.364 & 15.028 & This Work  \\
  19204795-4533283 & 0.18 & 1.706 & \textbf{-5.8392} & 0.0091 & 12.25 & 5.405 & 167.225 & This Work  \\
  21362532-4401005 & 0.20 & 1.886 & \textbf{-5.8978} & 0.0168 & 14.14 & 5.610 & 207.983 & This Work  \\
\hline
\end{tabular}


% \begin{tabular}{lcccccc}
% \hline
% 	2MASS ID &  Mass &    $Ro$ &  $\log(R'_{HK})$ & $\log(R'_{HK})_{err}$ & $V_{mag}$ &   $V-K$ \\
% \hline
% \hline
% 06000351+0242236 &  0.237 &  0.020 &     -4.548 &          0.002 &  11.310 &  5.268 \\
% 02125458+0000167 &  0.268 &  0.048 &     -4.635 &          0.001 &  13.580 &  5.412 \\
% 01124752+0154395 &  0.278 &  0.026 &     -4.473 &          0.001 &  14.009 &  5.240 \\
% 10252645+0512391 &  0.111 &  0.000 &     -4.971 &          0.007 &  18.110 &  7.322 \\
% 05015746-0656459 &  0.168 &  0.873 &     -5.005 &          0.003 &  12.200 &  5.464 \\
% 06022261-2019447 &  0.234 &  1.307 &     -5.698 &          0.012 &  13.260 &  4.886 \\
% 06105288-4324178 &  0.295 &  0.705 &     -5.251 &          0.008 &  12.280 &  4.968 \\
% 09442373-7358382 &  0.240 &  0.542 &     -5.603 &          0.006 &  15.170 &  5.795 \\
% 14211512-0107199 &  0.238 &  1.160 &     -5.585 &          0.008 &  13.120 &  5.027 \\
% 14294291-6240465 &  0.119 &  0.394 &     -5.005 &          0.001 &  11.130 &  6.746 \\
% 16352464-2718533 &  0.228 &  1.423 &     -5.596 &          0.006 &  14.180 &  5.182 \\
% 16570570-0420559 &  0.242 &  0.014 &     -4.307 &          0.001 &  12.250 &  5.130 \\
% 02004725-1021209 &  0.343 &  0.188 &     -4.791 &          0.002 &  14.118 &  5.026 \\
% 18494929-2350101 &  0.175 &  0.034 &     -4.524 &          0.001 &  10.500 &  5.130 \\
% 20035892-0807472 &  0.328 &  0.946 &     -5.653 &          0.007 &  13.540 &  5.254 \\
% 21390081-2409280 &  0.209 &  1.152 &     -6.195 &          0.015 &  13.450 &  5.091 \\
% 23071524-2307533 &  0.303 &  0.720 &     -5.278 &          0.006 &  13.587 &  4.849 \\
% 00094508-4201396 &  0.304 &  0.009 &     -4.339 &          0.001 &  13.620 &  5.397 \\
% 00310412-7201061 &  0.311 &  0.906 &     -5.388 &          0.006 &  13.690 &  5.245 \\
% 01040695-6522272 &  0.171 &  0.006 &     -4.489 &          0.002 &  13.980 &  5.448 \\
% 02014384-1017295 &  0.193 &  0.034 &     -4.540 &          0.002 &  14.473 &  5.284 \\
% 03100305-2341308 &  0.395 &  0.028 &     -4.234 &          0.001 &  13.502 &  4.935 \\
% 03205178-6351524 &  0.330 &  1.029 &     -5.629 &          0.007 &  13.433 &  5.238 \\
% 07401183-4257406 &  0.154 &  0.002 &     -4.337 &          0.001 &  13.810 &  6.042 \\
% 08184619-4806172 &  0.370 &  0.021 &     -4.283 &          0.001 &  14.370 &  5.019 \\
% 08443891-4805218 &  0.202 &  1.348 &     -5.668 &          0.004 &  13.932 &  5.370 \\
% 09342791-2643267 &  0.192 &  0.007 &     -4.341 &          0.001 &  13.992 &  5.373 \\
% 09524176-1536137 &  0.264 &  1.342 &     -5.632 &          0.007 &  13.430 &  4.923 \\
% 11075025-3421003 &  0.255 &  0.068 &     -4.225 &          0.001 &  15.040 &  5.633 \\
% 11575352-2349007 &  0.393 &  0.031 &     -4.295 &          0.001 &  14.770 &  5.415 \\
% 12102834-1310234 &  0.355 &  0.435 &     -4.689 &          0.002 &  13.830 &  5.418 \\
% 12440075-1110302 &  0.184 &  0.020 &     -4.405 &          0.002 &  14.220 &  5.546 \\
% 13442092-2618350 &  0.348 &  2.032 &     -5.963 &          0.014 &  13.253 &  4.968 \\
% 14253413-1148515 &  0.505 &  0.301 &     -4.764 &          0.002 &  13.512 &  5.121 \\
% 14340491-1824106 &  0.377 &  0.271 &     -4.609 &          0.002 &  14.346 &  5.638 \\
% 15154371-0725208 &  0.378 &  0.050 &     -4.621 &          0.002 &  12.930 &  5.224 \\
% 15290145-0612461 &  0.455 &  0.095 &     -4.201 &          0.001 &  14.011 &  5.230 \\
% 16204186-2005139 &  0.453 &  0.031 &     -4.390 &          0.002 &  13.680 &  5.261 \\
% 16475517-6509116 &  0.170 &  0.889 &     -4.874 &          0.003 &  13.980 &  5.101 \\
% 20091824-0113377 &  0.147 &  0.010 &     -4.377 &          0.001 &  14.470 &  5.958 \\
% 20273733-5452592 &  0.350 &  1.520 &     -5.998 &          0.012 &  13.180 &  5.259 \\
% 20444800-1453208 &  0.485 &  0.073 &     -4.491 &          0.002 &  14.445 &  5.305 \\
% 15404341-5101357 &  0.098 &  0.318 &     -5.006 &          0.003 &  15.260 &  7.317 \\
% 22480446-2422075 &  0.198 &  0.005 &     -4.412 &          0.001 &  12.590 &  5.384 \\
% 06393742-2101333 &  0.258 &  0.952 &     -5.252 &          0.004 &  12.770 &  5.120 \\
% 04130560+1514520 &  0.298 &  0.019 &     -4.477 &          0.004 &  15.881 &  5.437 \\
% 02411510-0432177 &  0.197 &  0.004 &     -4.427 &          0.001 &  13.790 &  5.544 \\
% \hline
% \end{tabular}

% \begin{tabular}{lccccc}
% \hline
% 2MASS ID &  Mass &    $Ro$ &  $\log(R'_{HK})$ &   $V_{mag}$ &   V-K \\
% \hline
% \hline
% J00094508-4201396 &  0.30 &  0.01 &      -4.33 &  13.62 &  5.40 \\
% J00310412-7201061 &  0.31 &  0.91 &      -5.36 &  13.69 &  5.24 \\
% J01040695-6522272 &  0.17 &  0.01 &      -4.47 &  13.98 &  5.45 \\
% J01124752+0154395 &  0.28 &  0.03 &      -4.45 &  14.01 &  5.24 \\
% J02014384-1017295 &  0.19 &  0.03 &      -4.53 &  14.47 &  5.28 \\
% J02125458+0000167 &  0.27 &  0.05 &      -4.63 &  13.58 &  5.41 \\
% J03100305-2341308 &  0.40 &  0.03 &      -4.21 &  13.50 &  4.94 \\
% J03205178-6351524 &  0.33 &  1.03 &      -5.60 &  13.43 &  5.24 \\
% J05015746-0656459 &  0.17 &  0.87 &      -4.98 &  12.20 &  5.46 \\
% J06000351+0242236 &  0.24 &  0.02 &      -4.53 &  11.31 &  5.27 \\
% J06105288-4324178 &  0.30 &  0.71 &      -5.21 &  12.28 &  4.97 \\
% J06105288-4324178 &  0.30 &  0.71 &      -5.21 &  12.28 &  4.97 \\
% J06393742-2101333 &  0.26 &  0.95 &      -5.21 &  12.77 &  5.12 \\
% J06393742-2101333 &  0.26 &  0.95 &      -5.21 &  12.77 &  5.12 \\
% J07401183-4257406 &  0.15 &  0.00 &      -4.28 &  13.81 &  6.04 \\
% J08184619-4806172 &  0.37 &  0.02 &      -4.22 &  14.37 &  5.02 \\
% J08443891-4805218 &  0.20 &  1.35 &      -5.59 &  13.93 &  5.37 \\
% J09342791-2643267 &  0.19 &  0.01 &      -4.31 &  13.99 &  5.37 \\
% J09524176-1536137 &  0.26 &  1.34 &      -5.48 &  13.43 &  4.92 \\
% J11075025-3421003 &  0.25 &  0.07 &      -4.21 &  15.04 &  5.63 \\
% J11575352-2349007 &  0.39 &  0.03 &      -4.28 &  14.77 &  5.41 \\
% J12102834-1310234 &  0.36 &  0.44 &      -4.60 &  13.83 &  5.42 \\
% J12440075-1110302 &  0.18 &  0.02 &      -4.35 &  14.22 &  5.55 \\
% J13442092-2618350 &  0.35 &  2.03 &      -5.74 &  13.25 &  4.97 \\
% J14211512-0107199 &  0.24 &  1.16 &      -5.43 &  13.12 &  5.03 \\
% J14253413-1148515 &  0.51 &  0.30 &      -4.75 &  13.51 &  5.12 \\
% J14294291-6240465 &  0.12 &  0.39 &      -5.00 &  11.13 &  6.75 \\
% J14340491-1824106 &  0.38 &  0.27 &      -4.56 &  14.35 &  5.64 \\
% J15154371-0725208 &  0.38 &  0.05 &      -4.58 &  12.93 &  5.22 \\
% J15290145-0612461 &  0.46 &  0.10 &      -4.44 &  14.01 &  5.23 \\
% J16204186-2005139 &  0.45 &  0.03 &      -4.32 &  13.68 &  5.26 \\
% J16204186-2005139 &  0.45 &  0.03 &      -4.32 &  13.68 &  5.26 \\
% J16352464-2718533 &  0.23 &  1.42 &      -5.46 &  14.18 &  5.18 \\
% J16360563+0848491 &  0.22 &  0.07 &      -3.93 &  13.81 &  5.30 \\
% J16400599+0042188 &  0.18 &  0.00 &      -4.35 &  13.70 &  5.49 \\
% J16570570-0420559 &  0.24 &  0.01 &      -4.28 &  12.25 &  5.13 \\
% J16570570-0420559 &  0.24 &  0.01 &      -4.28 &  12.25 &  5.13 \\
% J18494929-2350101 &  0.18 &  0.03 &      -4.52 &  10.50 &  5.13 \\
% J20035892-0807472 &  0.33 &  0.95 &      -5.65 &  13.54 &  5.25 \\
% J20091824-0113377 &  0.15 &  0.01 &      -4.37 &  14.47 &  5.96 \\
% J20444800-1453208 &  0.49 &  0.07 &      -4.46 &  14.44 &  5.30 \\
% J21390081-2409280 &  0.21 &  1.15 &      -6.16 &  13.45 &  5.09 \\
% J22480446-2422075 &  0.20 &  0.00 &      -4.39 &  12.59 &  5.38 \\
% J22480446-2422075 &  0.20 &  0.00 &      -4.39 &  12.59 &  5.38 \\
% J23071524-2307533 &  0.30 &  0.72 &      -5.28 &  13.59 &  4.85 \\
% J23071524-2307533 &  0.30 &  0.72 &      -5.28 &  13.59 &  4.85 \\
% J23532520-7056410 &  0.26 &  0.01 &      -4.31 &  13.01 &  5.23 \\
% \hline
% \end{tabular}

  \caption{Calculated Rossby Numbers and $R'_{HK}$ values. All circular data
  points in Figures \ref{fig:RpHKvsRossbySelf} \& \ref{fig:RpHKvsRossbyDef} are
  present in this table. Masses are taken from the MEarth database. A machine
  readable version of this table is available. Rows where the activity metric
  is in bold face were estimates derived from our model fit not empirical
  measurements.}
    \label{tab:finalData}
\end{table}
\begin{figure*}
    \centering
    \includegraphics[width=0.9\textwidth]{figures/magActivity/RpHKvsR0_MC_justThisPaper.pdf}
	\caption{Rotation activity relation from this work. The color axis gives
	each stars mass. The dashed line is the best fit to our data set.}
    \label{fig:RpHKvsRossbySelf}
\end{figure*}
\begin{figure*}
    \centering
    \includegraphics[width=0.9\textwidth]{figures/magActivity/RpHKvsR0_MC.pdf}
	\caption{Rotation activity relation for both our work and \citet{Def17}.
	The dotted line is the best fit to the re-derived rotation-activity
	relation from \citet{Def17}.  Note that targets from \citet{Def17} are
	systematically higher than targets presented here as a consequence of the
	range in mass probed by the samples.}
    \label{fig:RpHKvsRossbyDef}
\end{figure*}
\begin{figure*}
    \centering
    \includegraphics[width=0.9\textwidth]{figures/magActivity/RpHKvsR0_MC_fits.pdf}
	\caption{Derived rotation-activity curves from this work, \citet{Def17} and
	\citet{Mamajek2008}. Note both that \citet{Mamajek2008} focuses their work
	on earlier spectral classes and fits the rotation activity relation in
	linear space.}
    \label{fig:RpHKvsRossbyFits}
\end{figure*}

We find best fit parameters with one $\sigma$ errors:
\begin{itemize}
    \item $k = -1.347\pm 0.203$
    \item $Ro_{s} =  0.155\pm0.045$
    \item $\log(R_{s}) = -4.436\pm0.048$

\end{itemize}
A comparison of the rotation activity derived in this work to those
from both \citet{Def17} and \citet{Mamajek2008} is presented in Figure
\ref{fig:RpHKvsRossbyFits}. For the 6 targets which do not have measured
rotational periods we include an estimate of $Ro$ and $p_{rot}$ in the machine
readable version of Table \ref{tab:finalData}. The convective overturn
timescale for one of these 6 targets (2MASS J13464102-5830117) can not be
inferred via Equation \ref{eqn:convectiveOverturn} as it lacks a V-K color
measurement. Instead, we infer $\tau_{c}$ via \citet{Wri18} Equation 6 (this
paper Equation \ref{eqn:ConvectiveOverturnTimeMass}) using mass. Similar to our
manner of inferring $\tau_{c}$ via color, when inferring $\tau_c$ via mass, we
adopt the larger of the two antisymmetric errors from \citet{Wri18}.

\begin{equation}\label{eqn:ConvectiveOverturnTimeMass}
	\log_{10}(\tau_{c}) = 2.33\pm0.06 - 1.5\pm0.21\left(M/M_{\odot}\right) + 0.31\pm0.17\left(M/M_{\odot}\right)^{2}
\end{equation}

Note that $R'_{HK}$ for one of six of these targets (2MASS
J15165576-0037116) is consistent to within 1$\sigma$ of the saturated value;
therefore, the reported $Ro$ for this target should only be taken as an upper
bound. The remaining five targets have measured $R'_{HK}$ values consistent
with the unsaturated regime. Estimated periods are consistent with previous
constraints. Of the six stars, two were listed as non-detections in
\citet{Newton2018}, and the remaining four as uncertain (possible) detections.
Of the four classed as uncertain, 2MASS 12384914-3822527 and 2MASS
19204795-4533283 have candidate periods $>100$ days and non-detections of
H-alpha emission \citep{Hawley96}. These two stars and the two non-detections
have Ca II H\&K activity levels suggesting very long periods. 2MASS
13464102-5830117 has a candidate period of 45 days, and 2MASS 15165576-0037116
of 0.8 days, both consistent with their higher levels of Ca II H\&K emission.

As a test of the proposed weak correlation between activity and rotation in the
``saturated'' regime seen in some works \citep{Mamajek2008,
Reiners2014, Leh20, Med20} --- though not in others \citep{Wri11, Nunez2015,
Newton2017} ---   we fit a second model whose power law index is allowed to
vary at $Ro < Ro_{s}$. We find a saturated regime power law index of
$-0.052\pm0.117$, consistent with 0 to within 1$\sigma$. Moreover,
all other parameter for this model are consistent to within one $\sigma$ of the
nominal  parameters for the model where the index is constrained to 0 below
$Ro=Ro_{s}$. We can constrain the slope in the saturated
regime to be between -0.363 and 0.259 at the $3\sigma$ confidence level.
Ultimately, we adopt the most standard activity interpretation, a
fully-saturated regime at $Ro < Ro_{s}$. 

We investigate whether our lack of detection of a slope for $Ro <
Ro_{s}$ is due to the limited number of observations in that region when
compared to other works \citep[e.g.][93 targets $Ro < Ro_{s}$]{Med20} through
injection and recovery tests. We inject, fake, rotation-activity measurements
into the saturated regime with an a priori slope of -0.13 --- the same as in
\citeauthor{Med20}. These fake data are given a standard deviation equal to the
standard deviation of our residuals ($12\%$). We perform the same MCMC model
fitting to this new data set as was done with the original data set multiple
times, each with progressively more injected data, until we can detect the
injected slope to the three sigma confidence level. Ultimately, we need more
than 65 data points --- 43 more than we observed in the saturated regime --- to
consistently recover this slope. Therefore, given the spread of our data we
cannot detect slopes on the order of what has previously been reported in the
literature.

We observe a gap in rotational period over a comparable range to the
one presented in \citet{Newton2016} Figure 2. Namely, that M-dwarfs are
preferentially observed as either fast or slow rotators, with a seeming lack of
stars existing at mid rotational periods. This period gap manifests in the
Rossby Number and can be seen in Figure \ref{fig:RpHKvsRossbyDef} as a lack of
our targets near to the knee-point in the fit. This period gap likely
corresponds to that seen by \citet{Browning2010}, who found a paucity of M
dwarfs at intermediate activity levels in Ca II H\&K and note the similarity to
the Vaughn-Preston gap established in higher mass stars \citep{vaughan1980}.
\citet{Magaudda2020} also identify a double-gap in x-ray activity for stars in
the unsaturated regime; it is not clear that the gap we see is related. As a
consequence of this period gap, there exists a degeneracy in our data
between moving the knee-point and allowing the activity level to vary in the
saturated regime.  In the following, we adopt the model of a fully saturated
regime.

We wish to compare our best fit parameters to those derived in \citet{Def17};
however, the authors of that paper do not fit the knee-point of the
rotation-activity relation. They select the canonical value for the rotational
period separating the saturated regime from the unsaturated regime ($P_{rot,s}
= 10$ days) and use a fixed convective overturn timescale ($\tau_{c} = 70$
days). To make our comparison more meaningful we use the $P_{rot}$ and $V-K$
colors presented in \citet{Def17} to re-derive $Ro$ values using $\tau_{c}$
\citep{Wri18}. Doing this for all targets presented in \citet{Def17} Table 3
and fitting the same piecewise power law as before, we find best fit parameters
of $Ro_{s} = 0.17\pm0.04$, $\log(R_{s}) = -4.140\pm0.067$, and
$k=-1.43\pm0.21$. Compared to the best fit parameters for our data, $Ro_{s}$
and the unsaturated regime's index, $k$, are consistent to within one sigma,
while the saturated value, $R_{s}$, differs. 

The mass ranges of our respective samples explain the differences in saturation
values between our work and that of \citet{Def17}. Our work focuses on
mid-to-late M-dwarfs and includes no stars above a mass of $0.5$ M$_{\odot}$
(Figure \ref{fig:massDistribution}).  The strength of Ca II H\&K
emission is known to decrease as stellar mass decreases \citep{Schrijver1987,
Rauscher2006, Hou17}. As \citet{Rauscher2006} note, this is the opposite as the
trend seen in H-alpha; the latter primarily reflects the increasing length of
time that lower M dwarfs remain active and rapidly rotating \citep{West2015,
Newton2016}.

A mass dependence can be seen in Figure 10 in \citet{Def17}, consistent
with expectations from the literature. If we clip the data from \citet{Def17}
Table 3 to the same mass range as our data-set ($M_{*} < 0.5M_{\odot}$) and fit
the same function as above, we find that all best fit parameters are consistent
to within one sigma between the two data-sets. 

\begin{figure}
    \centering
    \includegraphics[width=0.85\textwidth]{figures/magActivity/B2020vsAD2016_Masses.pdf}
	\caption{Distribution of masses between our sample and the sample presented
	in \citet{Def17}. Note how the two studies have approximately the same
	sample sizes; however, our sample is more tightly concentrated at lower
	masses \textbackslash later spectral classes.}
    \label{fig:massDistribution}
\end{figure}

We also compare our best fit $Ro_{s}$ to both those derived in
\citet{Newton2017} using $H_{\alpha}$ as an activity measure and those derived
in \citep{ Wri18, Magaudda2020} using $L_{X}/L_{bol}$ as an activity measure.
Works using $L_{X}/L_{bol}$ identify a similar, yet not consistent to within
one sigma result for $Ro_{s}$; while, the value of $k$ we find here is
consistent between all four works. Therefore, we find similar results not only
to other work using the same activity tracer, but also a power-law slope that
is consistent with work using different tracers. 

\section{Magnetic Activities in M dwarfs Closing Thoughts}\label{sec:magActivity-conclusions}
In this work we have approximately doubled the number of M-dwarfs with both
empirically measured $R'_{HK}$ with $M_{*} < 0.5 M_{\odot}$. This has enabled
us to more precisely constrain the rotation-activity relation. This
relationship is consistent with other measurements using $R'_{HK}$, and
$L_{X}$/$L_{bol}$; our data does not require a slope in the saturated regime.
Finally, we identify a mass dependence in the activity level of the saturated
regime, consistent with trends seen in more massive stars in previous works.



\chapter{The Red Giant Branch Bump}
The red giant branch bump (RGBB) is a feature experienced by many stars as they
ascend the RGB \citep{Riello2003}. During this period, as the core of the star
is contracting and heating the surronding matierial shell burning begins.
Regions of the star which were previously too cool to fuse hydrogesn (and so
therefore still have hydrogen fuel despite core depletion) begin to fuse. At
the same time, because of the increase energy output from the heating core and
shell burning the balance between the adiabatic and radiative temperature
gradients near the convective envlope of the star begins to flip. This results
in the convective envelope pushing deeper --- in mass fraction --- into the
star. As the convective envelope reaches further into the star hydrogen from
the stars outer layers may be effeciently mixed deeper. The convective envelope
only ever manages to reach a fraction of a stars mass fraction during this
phase of evolution; however, due to the efficient mixing, by the time it starts
to recede to the surface there is a radial discontinuity in the hyrdogen
concentration within the star. If shell burning reaches this far out then the
normal corse of RGB asension will be interupted by an anoumously large amount
of fuel. The star will remain burning that shell for longer than might
otherwise be predicted leading to a ``bump'' in the luminosity function. This
is known as the Red Giant Branch Bump.

The RGBB provides yet another view into the interior physics of a star and may
allow for calibration and testing of stellar models against observations.
Previous work by \citet{Joyce2016} has found that current
generation 1D stellar models, such as DSEP, may undersestimate the bump
luminosity for metal poor stars but that they generally models the bump well
for more metal rich stars.

In this chapter we will provide two breif views into the RGBB. First, we will
look at how different self consistent models of multiple populations in NGC
2808 (the same models disscused in Chapter \ref{chap:ngc2808}) effect the
RGBB location. Second, we will investigate how the updated low and high
temperature opacities which we have incorperated into DSEP effect the RGBB
location.

\section{The Red Giant Branch Bunch in NGC 2808}
NGC 2808 is made up of of anywhere from 2 - 5 separate stellar populations. In
Chapter \ref{chap:ngc2808} we discussed modeling efforts of the primordial
and the most enriched of these populations (A and E). Here we identify the
RGBB location in both populations and compare that to observations of the RGBB
luminosity.

Identification of the RGBB may be preformed with either isochrones or stellar
evolutionary tracks. For the same reasons laid out in \citet{Joyce2016} we use
evolutionary tracks in place of isochrones. This is primarily due to the
limited sampling of equivalent evolutionary points near the bump which can
result in the bump be interpolated over. 

We select two evolutionary tracks from the Population A (Y$\sim$0.24) and
Population E (Y $\sim$ 0.36) model sets each of a mass so that they reach the
red giant branch within 100 Myr of each other. The population A model is of
mass 0.857 M$_{\odot}$ while the population E model is of mass 0.625
M$_{\odot}$. Identification of the RGBB in the tracks is then straightforward.
First we preform bolometric corrections to take the tracks into WFC3/ACS
filters with the same distance modulus and extinction as we fit in Chapter
\ref{chap:ngc2808}, then we identify the maximum of the derivative of F814W
magnitude vs. age. This metric proves to reliably extract the center of the
RGBB. We identify the bump in population A at an F814W magnitude of approximately 16.5 while
we are unable to identify the gap in population E.

In order to verify this is a attribute of the population and not the particular
model we selected we preform stellar population synthesis using the best fit
isochrones from Chapter \ref{chap:ngc2808}. This is done using the population
synthesis module in \fidanka. Further, HUGS artificial star tests are used in
order to inject proper uncertainties and to model completeness. Figure
\ref{fig:LumFAE} shows both the synthetic population (left) and the luminosity
function of that population (right). We measure the magnitude of the RGBB by
first flattening the luminosity function shown in Figure \ref{fig:LumFAE} with
a second order polynomial. We then fit a Gaussian model --- with a prior of
16.5 as the mean of the RGBB magnitude --- to the flattened luminosity function.
We find that the RGBB in population A is located at F814W = 16.53$\pm$0.004
mag, this is in agreement with literature values for the magnitude of the RGBB
in NGC 2808. However, once again we do not see a bump in the synthetic
population E data. The pertinent question then becomes why do we not see a
RGBB in population E when compared to population A.

\begin{figure}
  \centering
  \includegraphics[width=0.85\textwidth]{figures/rgbb/ComparisonOfRGBBump.pdf}
  \caption{Luminosity function for a population A and E model. Note how there
  is a bump at approximately $M_{V} = 16.5$ for population A but not for E. The
  dashed line represents the literature value for the observed RGBB in NGC
  2808.}
  \label{fig:LumFAE}
\end{figure}

\subsection{Population E and the Case of the Missing Bump}
It is well established that stars with lower metallicities will have a less
pronounced bump \citep{Cassisi2002, Bjork2006}. Population E in fact is
dramatically more depleted in general metallicity than population A ([Fe/H] =
-1.34 and [Fe/H] =-0.94 respectively). Lower metallicities decrease the opacity
of a star and therefore make the coupling between the radiation field and the
fluid field weaker. This results in less efficient thermal transfer into the
fluid of the star. This results in a more shallow outer convective zone
(Figures \ref{fig:convEnvMass} \& \ref{fig:MfracVX}) which may not be able mix
new hydrogen fuel deep enough to induce the RGBB. Specifically, in population E
we see that the convection zone only reaches down 50\% of the mass of the model
(Figure \ref{fig:MfracVX}). Because of this, shell burning does not reach the
discontinuity in hydrogen mass fraction until later in the model's life. Because
the luminosity of the models is rapidly increasing during this phase of
evolution the timescales to burn hydrogen shrink rapidly as well. Therefore, in
population E even though there does seem to be a small discontinuity in the
hydrogen mass fraction, evolutionary time scales are so short that the
population will evolve through this period without a noticeable bump. The lack
of an observed RGB bump in Population E models --- and the commensurate lack of
effect that population E has on the location of the NGC 2808 bump --- validates
previous theoretical investigations of the bump which have ignored the presence
of multiple populations in NGC 2808 while modeling the bump.

\begin{figure}
  \centering
  \includegraphics[width=0.85\textwidth]{figures/rgbb/ConvectiveEnvelopeMass.pdf}
  \caption{Fractional mass of the convective envelope of a population A and E
  model vs. model age. Note how the population A model's convective envelope
  reaches approximately 60 percent of the star by mass while the population E
  convective envelope only reaches 40 percent of the star by mass.}
  \label{fig:convEnvMass}
\end{figure}

\begin{figure}
    \centering
    \includegraphics[width=\textwidth]{figures/rgbb/EvolutionAE.pdf}
    \caption{Evolution of the hydrogen mass fraction vs. the Lagrangian depth
    within Population A (top) and E (bottom) Stellar models. From left to right
    panels show snapshots at informative ages. Note how Population E does not
    exhibit the same deep bump in its hydrogen mass fraction as Population A
    does. Populations have been roughly mas calibrated so they reach the same
    evolutionary stages within 100 Myr of each other.}
    \label{fig:MfracVX}
\end{figure}

\section{Effects of Opacities on the Red Giant Branch Bump}

\input{chapters/rgb_bump/subsections/implications.tex}


\part{Conclusions}
\section{Conclusion}\label{sec:2808conclusion}
Here we have preformed the first chemically self-consistent modeling of the
Milky Way Globular Cluster NGC 2808. We find that, updated atmospheric boundary
conditions and opacity tables do not have a significant effect on the inferred
helium abundances of multiple populations. Specifically, we find that
population  has a helium mass fraction of 0.24, while population E has a helium
mass fraction of 0.39. Additionally, we find that the ages of these two populations 
agree within uncertainties. We only find evidence for two distinct stellar
populations, which is in agreement with recent work studying the number
of populations in NGC 2808 spectroscopic data.

We introduce a new software suite for globular cluster science,
\fidanka, which has been released under a permissive open source license.
\fidanka aims to provide a statistically robust set of tools for estimating the
parameters of multiple populations within globular clusters.


\cleardoublepage
\phantomsection
\addcontentsline{toc}{part}{Bibliography}
\bibliography{bib/ms} % Entries are in the refs.bib file
\bibliographystyle{aasjournal} % We choose the "plain" reference style


% Now the user can start typing
\end{document}
