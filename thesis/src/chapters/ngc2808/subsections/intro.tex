NGC 2808 is the prototype globular cluster to host Multiple Populations.
Various studies since 2007 have identified that it may host anywhere from 2-5
stellar populations. These populations have been identified both
spectroscopically \citep[i.e.][]{Carretta2004, Carretta2006, Carretta2010,
Gratton2011, Carretta2015, Hong2021} and photometrically
\citep[i.e.][]{Piotto2007, Piotto2015, Milone2015, Milone2017, Pasquato2019}.
Note that recent work \citep{Valle2022} calls into question the statistical
significance of the detections of more than 2 populations in the spectroscopic
data. Here we present new, chemically self-consistent modeling of the
photometry of the two extreme populations of NGC 2808 identified by
\citet{Milone2015}, populations A and E. We do not consider populations B, C,
or D identified in \citet{Milone2015} as the purpose of this work is to
identify if chemically self-consistent modelling results in a statisically
signifigant deviation in the infered helium abundance when compared to non
chemically self-consistent models. Use of the two populations in the NGC 2808
with the highest identified difference between their helium populations is
sufficent for to answer this question.  We use archival photometry from the
Hubble UV Globular Cluster Survey (HUGS) \citep{Piotto2015, Milone2017} in the
F275W and F814W passbands to characterize multiple populations in NGC 2808
\citep{Milone2015, Milone2015b} (This data is avalible at MAST:
\href{https://archive.stsci.edu/doi/resolve/resolve.html?doi=10.17909/T9810F}{10.17909/T9810F}).
Additionally, we present a likelihood analysis of the photometric data of NGC
2808 to determine the number of populations present in the cluster.


