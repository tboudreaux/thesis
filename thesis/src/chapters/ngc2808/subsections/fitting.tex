\section{Isochrone Fitting}\label{sec:isoFit}
We fit pairs of isochrones to the HUGS data for NGC 2808 using \fidanka, as
described in \S \ref{sec:fidanka}. Two isochrones, one for Population A and one
for Population E are fit simultaneously. These isochrones are constrained to
have distance modulus, $\mu$, and color excess, E(B-V) which agree to within
0.5\% and an ages which agree to within 1\%. Moreover, we constrain the mixing
length, $\alpha_{ML}$, for any two isochrones in a set to be within 0.5 of one
and other. For every isochrone in the set of combination of which fulfilling
these constraints $\mu$, $E(B-V)$, Age$_{A}$, and $Age_{B}$ are optimized to
reduce the $\chi^{2}$ distance ($\chi^{2} = \sum\sqrt{\Delta \text{color}^{2} +
\Delta \text{mag} ^{2}}$) between the fiducial lines and the isochrones.
Because we fit fiducial lines directly, we do not need to consider the binary
population fraction, $f_{bin}$, as a free parameter.

The best fit isochrones are shown in Figure \ref{fig:BestFitResults} and
optimized parameters for these are presented in Table \ref{tab:BestFitResults}.
The initial guess for the age of these populations was locked to 12 Gyr
and the initial Extinction was locked to 0.5 mag. The initial guess for the
distance modulus was determined at run time using a dynamic time warping
algorithm to best align the morphologies of the fiducial line with the target
isochrone. This algorithm is explained in more detail in the \fidanka
documentation under the function called \texttt{guess\_mu}. We find helium mass
fractions that are consistent with those identified in past literature
\citep[e.g.][]{Milone2015}. Note that our helium mass fraction grid has a
spacing of 0.03 between grid points and we are therefore unable to resolve
between certain proposed helium mass fractions for the younger sequence (for
example between 0.37 and 0.39). We also note that the best fit mixing
length parameter which we derive for populations A and E do not agree within
their uncertainties. This is not surprising as the much high mean molecular mass
of population E --- when compared to population A, due to population E's larger
helium mass fraction --- will result in a steeper adiabatic temperature
gradient.

\begin{figure*}
  \centering
  \includegraphics[width=0.9\textwidth]{figures/ngc2808/BestFitResults.pdf}
  \caption{Best fit isochrone results for NGC 2808. The best fit population A
  and E models are shown as black lines. The following 50 best fit models are
  presented as gray lines. The solid black line is fit to population A, while
  the dashed black line is fit to population E.}
  \label{fig:BestFitResults}
\end{figure*}

\begin{table*}
  \centering
  \begin{tabular}{c | c c c c c c}
    \hline
    Population & Age & Distance Modulus & Extinction & Y & $\alpha_{ML}$ & $\chi^{2}_{\nu}$\\
    & [Gyr] & & [mag] & & &\\
    \hline
    \hline
    A & 12.996$^{+0.87}_{-0.64}$ & 15.021 & 0.54 & 0.24 & 2.050 & 0.021\\
    E & 13.061$^{+0.86}_{-0.69}$ & 15.007 & 0.537 & 0.39 & 1.600 & 0.033 \\
    \hline
  \end{tabular}
  \caption{Best fit parameters derived from fitting isochrones to the fiducual
  lines derived from the NCG 2808 photometry. The one sigma uncertainty
  reported on population age were determined from the 16th and 84th percentiles
  of the distribution of best fit isochrones ages.}
  \label{tab:BestFitResults}
\end{table*}


Past literature \citep[e.g. ][]{Milone2015, Milone2018} have found helium mass
fraction variation from the low red-most to blue-most populations of $\sim 0.12$.
Here we find a helium mass fraction variation of 0.15 which, given the spacing
of the helium grid we use {\em is consistent with these past results}.

\subsection{The Number of Populartions in NGC 2808}
In order to estimate the number of populations which ideally fit the NGC 2808
F275W-F814W photometry without over-fitting the data we make use of silhouette
analysis \citep[][and in a similar manner to how \citet{Valle2022} perform
their analysis of spectroscopic data]{ROUSSEEUW198753}. We find the average
silhouette score for all tagged clusters identified using BGMM in all magnitude
bins over the CMD using the standard python module \texttt{sklearn}. Figure
\ref{fig:clusterAn} shows the silhouette analysis results and that two
populations fit the photometry most ideally. This is in line with what our BGMM
model predicts for the majority of the CMD.

\begin{figure}
  \centering
  \includegraphics[width=0.85\textwidth]{figures/ngc2808/ClusterAnalysis.pdf}
  \caption{Silhouette analysis for NGC 2808 F275W-F814W photometry. The
  Silhouette scores are an average of score for each magnitude bin. Positive
  scores indicate that the clustering algorithm produced well distinguished
  clusters while negative scores indicate clusters which are not well
  distinguished.}
  \label{fig:clusterAn}
\end{figure}

While we make use a purely CMD based approach in this work, other literature
has made use of Chromosome Maps. These consist of implicitly verticalized
pseudo colors. In the chromosome map for NGC 2808 there may be evidence for
more than two populations; however, the process of transforming magnitude
measurements into chromosome space results in dramatically increased
uncertainties for each star. We find a mean fractional uncertainty for
chromosome parameters of $\approx1$ when starting with magnitude measurements
having a mean best-case (i.e. uncertainty assumed to only be due to Poisson
statistics) fractional uncertainty of $\approx 0.0005$ (Figure
\ref{fig:chromMapUn}). Because of how \fidanka operates, i.e. resampling a
probability distribution for each star in order to identify clusters, we are
unable to make statistically meaningful statements from the chromosome map


\begin{figure}
  \centering
  \includegraphics[width=0.85\textwidth]{figures/ngc2808/ChromosomeMapFractionalErrorDist.pdf}
  \caption{Fractional uncertainty distribution of the chromosome map parameter
  space for targets in NGC 2808. Note that fractional uncertainties of the
  magnitudes which went into the production of this chromosome map were on the
  order of 0.0005 (the blue vertical line in both plots marks this). Further,
  we assumed that there was no uncertainty on the placement of the red and blue
  fiducial lines. If there were uncertainty on those placements then the mean
  of this distribution would be higher.}
  \label{fig:chromMapUn}
\end{figure}

\subsection{ACS-HUGS Photometric Zero Point Offset}
The Hubble legacy archive photometry used in this work is calibrated to the
Vega magnitude system. However, we have found that the photometry has a
systematic offset of $\sim0.026$ magnitudes in the F814W band when
compared to the same stars in the ACS survey (Figure \ref{fig:offset}). The
exact cause of this offset is unknown, but it is likely due to a difference in
the photometric zero point between the two surveys. A full correction of this
offset would require a careful re-reduction of the HUGS photometry, which is
beyond the scope of this work. We instead recognize a 0.02 inherent uncertainty
in the inferred magnitude of any fit when comparing to the ACS survey. This
uncertainty is small when compared to the uncertainty in the
distance modulus and should not affect the conclusion of this
paper. 

\begin{figure*}
  \centering
  \includegraphics[width=0.90\textwidth]{figures/ngc2808/photometricOffset.pdf}
  \caption{(left) CMD showing the photometric offset between the ACS and HUGS
  data for NGC 2808. CMDs have been randomly subsampled and colored by point
  density for clarity. (right) Mean difference between the color of the HUGS
  and ACS fiducual lines at the same magnitude. Note that the ACS data is
  systematically bluer than the HUGS data.}
  \label{fig:offset}
\end{figure*}

The oberved photometric offset between ACS and HUGS reductions introduces a
systematic uncertainty when comparing parameters derived from isochrone fits
to ACS data vs those fit to HUGS data. Specifically, this offset introduces a
$\sim 2 Gyr$ uncertainty when comparing ages between ACS and HUGS. Moreover,
for two isochrone of the same age, only separated by helium mass fraction, a
shift of the main sequence turn off of is also expected. Figure \ref{fig:HeMO}
shows this shift. Note a change in the helium mass fraction of a model by 0.03
results in an approximate 0.08 magnitude shift to the main sequence turn off
location. This means that the mean 0.026 magnitude offset we find in between
ACS and HUGS data corresponds to an additional approximate 0.01 uncertainty
in the derived helium mass fraction when comparing between these two data sets. 

\begin{figure}
  \centering
  \includegraphics[width=0.85\textwidth]{figures/ngc2808/HeliumMeanOffset.pdf}
  \caption{Main sequence turn off magnitude offset from a gauge helium mass
  fraction (Y=0.30 chosen). All main sequence turn off locations are measured
  at 12.3 Gyr}
  \label{fig:HeMO}
\end{figure}
