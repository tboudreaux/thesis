\section{fidanka}\label{sec:fidanka}
When fitting isochrones to the data we have four main criteria for any method

\begin{itemize}
	\item The method must be robust enough to work along the entire main sequence, turn off, and much of the subgiant and red giant branchs.
	\item Any method should consider photometric uncertainty in the fitting process.
	\item The method should be model independent, weighting any n number of populations equally.
	\item The method should be automated and require minimal intervention from the user.
\end{itemize}


We do not believe that any currently available software is a match for
our use case. Therefore, we elect to develop our own software suite, \fidanka.
\fidanka is a python package designed to automate much of the process of
measuring fiducial lines in CMDs, adhering to the four criteria we lay out
above. Primary features of \fidanka may be separated into three
categories: fiducial line measurement, stellar population synthesise, and
isochrone optimization/fitting. Additionally, there are utility functions which
are detailed in the \fidanka documentation.

\subsection{Fiducial Line Measurement}
\fidanka takes a iterative approach to measuring fiducial lines, the first step
of which is to make a ``guess'' as to the fiducial line. . This initial guess
is calculated by splitting the CMD into magnitude bins, with uniform numbers of
stars per bin (so that bins are cover a small magnitude range over densely
populated regions of the CMD while covering a much larger magnitude range in
sparsely populated regions of the CMD, such as the RGB). A unimodal Gaussian
distribution is then fit to the color distribution of each bin, and the
resulting mean color is used as the initial fiducial line guess. This rough
fiducial line will approximately trace the area of highest density. The initial
guess will be used to verticalze the CMD so that further algorithms can work in
1-D magnitude bins without worrying about weighting issues caused by varying
projections of the evolutionary sequence onto the magnitude axis.
Verticalization is preformed taking the difference between the guess fiducial
line and the color of each star in the CMD.

If \fidanka were to simply apply the same algorithm to the verticalized CMD
then the resulting fiducial line would likely be a re-extraction of the initial
fiducial line guess. To avoid this, we take a more robust, number density based
approach, which considers the distribution of stars in both color and magnitude
space simultaneously. For each star in the CMD we first using a
\texttt{introselect} partitioning algorithm to select the 50 nearest stars in F814W vs. F275W-F814W space.
To account for the case where the star is at an extreme edge of the CMD, those
50 stars include the star itself (such that we really select 49 stars + 1). We
use \texttt{qhull}\footnote{https://www.qhull.com}\citep{Barber1996, } to
calculate the convex hull of those 50 points. The number density at each star
then is defined as $50/A_{hull}$, where $A_{hull}$ is the area of the convex
hull. Because we use a fixed number of points per star, and a partitioning
algorithm as opposed to a sorting algorithm, this method scales like
$\mathcal{O}(n)$, where n is the number of stars in the CMD. This method also
intrinsically weights the density of of each star equally as the counting
statistics per bin are uniform. We are left with a CMD where each star
has a defined number density (Figure \ref{fig:densityMapDemo}).

\begin{figure*}
	\centering
	\includegraphics[width=0.9\textwidth]{figures/ngc2808/notebookFigures/DensityMapDemo.png}
	\label{fig:densityMapDemo}
	\caption{Density map demo showing density estimate over different parts of
	the evolutionary sequence. The left panel shows the density map over the
	entire evolutionary sequence, while the middle panel shows the density map
	over the main sequence and the right most panel shows the density map over
	the RGB. Figures in the top row are the raw CMD, while figures in the
	bottom row are colored by the density map.}
\end{figure*}

\fidanka can now exploit this density map to fit a better fiducial line to the
data, as the density map is far more robust to outliers. There are multiple
algorithms we implement to fit the fiducial line to the color-density profile
in each magnitude bin (Figure \ref{fig:densityBinsDemo}); they are explained in more detail in the \fidanka
documentation. However, of most relevance here is the Bayesian Gaussian Mixture
Modeling (BGMM) method. BGMM is a clustering algorithm which, for some fixed
number of n-dimensional Gaussian distributions, $K$, determines the mean, covariance, and
mixing probability (somewhat analogous to amplitude) of each $k^{th}$
distribution, such that the local lower bound of the evidence of each star
belonging strongly to a single distribution is maximized. 

\begin{figure}
	\centering
	\includegraphics[width=0.85\textwidth]{figures/ngc2808/notebookFigures/DensityBinsDemo.png}
	\label{fig:densityBinsDemo}
	\caption{CMD where points are colored by density. Lines show the
	density-color profile in each magnitude bin. In this figure adaptive
	binning targeted 1000 stars per bin}
\end{figure}

Maximization is preformed using the Dirichlet process, which is a
non-parametric Bayesian method of determining the number of Gaussian distributions, $K$,
which best fit the data \citep{Ferguson1973, scikit-learn}. Use of the Dirichlet process
allows for dynamic variation in the number of inferred populations from
magnitude bin to magnitude bin. Specifically, populations are clearly visually
separated from the lower main sequence through the turn off; however, at the
turn off and throughout much of the subgiant branch, the two visible
populations overlap due to their extremely similar ages \citep[i.e.][]{Jordan2002}. The Dirichlet process allows for the BGMM method to infer a single
population in these regions, while inferring two populations in regions where
they are clearly separated. More generally, the use of the Dirichlet process
removes the need for a prior on the exact number of populations to fit. Rather,
the user specifies a upper bound on the number of populations within the
cluster. An example bin (F814W = 20.6) is shown in Figure \ref{fig:BGMMDist}.

\begin{figure*}
	\centering
	\includegraphics[width=0.9\textwidth]{figures/ngc2808/BGMMMixingBin.pdf}
	\caption{Example of BGMM fit to a magnitude bin. The grey line shows the
	underlying color-density profile, while the black dashed-line shows the
	joint distribution of each BGMM component. The solid black lines show the
	two selected components.}
	\label{fig:BGMMDist}
\end{figure*}

\fidanka's BGMM method first breaks down the verticalized CMD into magnitude
bins with uniform numbers of stars per bin (here we adopt 250). Any stars left
over are placed into the final bin. For each bin a BGMM model with a maximum of
5 populations is fit to the color density profile. The number of populations is
then inferred from the weighting parameter (the mixing probability) of each
population. If the weighting parameter of any $k^{th}$ components less than
{\color{blue}0.05}, then that component is considered to be spurious and
removed. Additionally, if the number of populations in the bin above and the
bin below are the same, then the number of populations in the current bin is
forced to be the same as the number of populations in the bin above. Finally,
the initial guess fiducial line is added back to the BGMM inferred line. Figure
\ref{fig:vertFit} shows the resulting fiducial line(s) in each magnitude bin
for both a verticalized CMD and a non verticalized CMD.

\begin{figure}
	\centering
	\includegraphics[width=0.85\textwidth]{figures/ngc2808/vertFit.png}
	\caption{CMD where points are colored by density. Line trace the infered 
	fiducial line(s) in each magnitude bin.}
	\label{fig:vertFit}
\end{figure}

This method of fiducial line extraction effectively discriminated between
multiple populations long the main sequence and RGB of a cluster, while
simultaneously allowing for the presence of a single population along the MSTO
and subgiant branch. 

We can adapt this density map based BGMM method to consider photometric
uncertainties by adopting a simple Monte Carlo approach. Instead of measuring
the fiducial line(s) a single time, \fidanka can measure the fiducial line(s)
many times, resampling the data with replacement each time. For each resampling
\fidanka adds a random offset to each filter based on the photometric
uncertainties of each star. From these $n$ measurments the mean fiducial line
for each sequence can be identified along with upper and lower bound confidence
intervals in each magnitude bin.

\subsection{Stellar Population Synthesis}
In addition to measuring fiducial lines, \fidanka also includes a stellar
population synthesise module. This module is used to generate synthetic CMDs
from a given set of isochrones. This is of primary importance for binary
population modelling. The module is also used to generate synthetic CMDs for
the purpose of testing the fiducial line extraction algorithms against priors.

\fidanka uses MIST formatted isochrones \citep{Dotter2016} as input along
with distance modulus, B-V color excess, binary mass fraction, and bolometric
corrections. An arbitrarily large number of isochrones may be used to define an
arbitrary number of populations. Synthetic stars are samples from each
isochrone based on a definable probability (for example it is believed that
$\sim90\%$ of stars in globular clusters are younger population
\citep[e.g.][]{Suntzeff1996, Carretta2013}). Based on the metallicity, $\mu$, and E(B-V) of each
isochrone, bolometric corrections are taken from bolometric correction tables.
Where bolometric correction tables do not include exact metallicities or
extinctions a linear interpolation is preformed between the two bounding
values. 

\subsection{Isochrone Optimization}
The optimization routines in \fidanka will find the best fit distance modulus,
B-V color excess, and binary number fraction for a given set of isochrones. If
a single isochrone is provided then the optimization is done by minimizing the
$\chi^2$ of the perpendicular distances between an isochrone and a fiducial
line. If multiple isochrones are provided then those isochrones are first used
to run stellar population synthesis and generate a synthetic CMD. The
optimization is then done by minimizing the $\chi^2$ of both the perpendicular
distances between and widths of the observed fiducial line and the fiducial
line of the synthetic CMD.


\subsection{Fidanka Testing}
In order to validate fidanka we have run an series of injection recovery tests
using \fidanka's population synthesis routines to build various synthetic
populations and \fidanka's fiducial measurement routines to recover these
populations. Each population was generated using the initial mass function
given in \citep{Milone2012} for the redmost population ($\alpha=-1.2$).
Further, every population was given a binary population fraction of 10\%,
distance uniformly sampled between 5000pc and 15000pc, and a B-V color excess
uniformly sampled between 0 and 0.1. Finally, each synthetic population was
generated using a fixed age  uniformlly sampled between 7 Gyr and 14 Gyr. An
example synthetic population along with its associated best fit isochrone are
shown in Figure \ref{fig:ValidationBestFit}.

\begin{figure}
  \centering
  \includegraphics[width=0.85\textwidth]{figures/ngc2808/ExtractedIsoFit.pdf}
  \caption{Synthetic population generated by fidanka at 10000pc with E(B-V) = 0, and an age of 12 Gyr along with the best fitting isochrone. The best fit paremeters are derived to be $mu=15.13$, E(B-V)=0.001, and an age of 12.33 Gyr.}
  \label{fig:ValidationBestFit}
\end{figure}

For each trial we use \fidanka to measure the fiducial line and then optimize that fiducial line against the originating isochrone to esimate distance modulus, age, and color B-V excess. Figure \ref{fig:validationDist} is built from 1000 runs of these trials and show the mean and width of the percent error distributions for $\mu$, $E(B-V)$, and age. In general \fidanka is able to recover distance modulii effectively with age and E(B-V) reovery falling in line with other literature that does not cosider the CMD outside of the main sequence, main sequence turn off, sub giant, and red giant branches; specifically, it should be noted that \fidanka is not setup to model the horizontal branch.

\begin{figure}
  \centering
  \includegraphics[width=0.85\textwidth]{figures/ngc2808/DistributionOfErrors.pdf}
  \caption{Percent Error distribution for each of the three deriver parameters. Note that these values will be sensitive to the magnitude uncertainties of the photometry. Here we made use of the ACS artificial star tests to estimate the uncertanties. {\color{blue}Note that currently this is built with 100 runs, these take a long time so currently re running with 1000 runs.}}
  \label{fig:validationDist}
\end{figure}
