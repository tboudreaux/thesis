The red giant branch bump (RGBB) is a feature experienced by many stars as they ascend the RGB. During this period, as the core of
the star is contracting and heating the surronding matierial shell burning
begins. Regions of the star which were previously too cool to fuse hydrogesn
(and so therefore still have hydrogen fuel despite core depletion) begin to
fuse. At the same time, because of the increase energy output from the heating
core and shell burning the balance between the adiabatic and radiative
temperature gradients near the convective envlope of the star begins to flip.
This results in the convective envelope pushing deeper --- in mass fraction --- into the star. As the
convective envelope reaches further into the star hydrogen from the stars outer
layers may be effeciently mixed deeper. The convective envelope only ever
manages to reach a fraction of a stars mass fraction during this phase of evolution;
however, due to the efficient mixing, by the time it starts to recede to the
surface there is a radial discontinuity in the hyrdogen concentration within
the star. If shell burning reaches this far out then the normal corse of RGB
asension will be interupted by an anoumously large amount of fuel. The star
will remain burning that shell for longer than might otherwise be predicted
leading to a ``bump'' in the luminosity function. This is known as the Red
Giant Branch Bump.

The RGBB provides yet another view into the interior physics of a star and may
allow for calibration and testing of stellar models against observations.
Previous work by \citet{Joyce2016} has found that current
generation 1D stellar models, such as DSEP, may undersestimate the bump
luminosity for metal poor stars but that they generally models the bump well
for more metal rich stars.

In this chapter we will provide two breif views into the RGBB. First, we will
look at how different self consistent models of multiple populations in NGC
2808 (the same models disscused in Chapter \ref{chap:ngc2808}) effect the
RGBB location. Second, we will investigate how the updated low and high
temperature opacities which we have incorperated into DSEP effect the RGBB
location.
