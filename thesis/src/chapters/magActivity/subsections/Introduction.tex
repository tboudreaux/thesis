\section{Magnetic Activity in M dwarfs} \label{sec:magActivity-intro}
M-dwarfs are the most numerous stars in our galaxy; however, spun-up M-dwarfs
are more magnetically active when compared to larger and hotter stars
\citep{Haw91, Del98, Sch14}. The increase in activity may accelerate the
stripping of an orbiting planet's atmosphere \citep[e.g.][]{Owe16}, and may
dramatically impact habitability \citep{Shi16}. Therefore, it is essential to
understand the magnetic activity of M-dwarfs in order to constrain the
potential habitability and history of the planets that orbit them.
Additionally, rotation and activity may impact the detectability of hosted
planets \citep[e.g.][]{Rob14, Newton2016, Van16}.

Robust theories explaining the origin of solar-like magnetic fields exist and
have proven extensible to other regions of the main sequence \citep{Cha14}. The
classical $\alpha\Omega$ dynamo relies on differential rotation between layers
of a star to stretch a seed poloidal field into a toroidal field \citep{Par55,
Cam17}. Magnetic buoyancy causes the toroidal field to rise through the star.
During this rise, turbulent helical stretching converts the toroidal field back
into a poloidal field \citep{Par55}. Seed fields may originate from the
stochastic movement of charged particles within a star's atmospheres.

In non-fully convective stars the initial conversion of the toroidal field to a
poloidal field is believed to take place at the interface layer between the
radiative and convective regions of a star --- the tachocline \citep{Noy84,
Tom96, Dik99}. The tachocline has two key properties that allow it to play an
important role in solar type magnetic dynamos: 1), there are high shear
stresses, which have been confirmed by astroseismology \citep{Tho96}, and 2),
the density stratification between the radiative and convective zones serves to
``hold'' the newly generated toroidal fields at the tachocline for an extended
time. Over this time, the fields build in strength significantly more than they
would otherwise \citep{Par75}. This theory does not trivially extend to
mid-late M-dwarfs, as they are believed to be fully convective and consequently
do not contain a tachocline \citep{Cha97}. Moreover, fully convective M-dwarfs
are not generally expected to exhibit internal differential rotation
\citep[e.g.][]{barnes2004differential, barnes2005dependence}, though, some
models do produce it \citep{Yad13}.

Currently, there is no single accepted process that serves to build and
maintain fully convective M-dwarf magnetic fields in the same way that the
$\alpha$ and $\Omega$ processes are presently accepted in solar magnetic dynamo
theory. Three-dimensional magneto anelastic hydrodynamical simulations have
demonstrated that local fields generated by convective currents can self
organize into large scale dipolar fields. 
%These fields are similar to smaller scale structure which have been observed through Zeeman-Doppler imaging \citep{Brown2011, Yad15}
These models indicate that for a fully convective star to sustain a magnetic
field it must have a high degree of density stratification --- density
contrasts greater than 20 at the tachocline --- and a sufficiently large magnetic Reynolds
number\footnote{The Reynolds Number is the ratio of magnetic induction to
magnetic diffusion; consequently, a plasma with a larger magnetic Reynolds
number will  sustain a magnetic field for a longer time than a plasma with a
smaller magnetic Reynolds number.}.

An empirical relation between the rotation rate and the level of magnetic
activity has been demonstrated in late-type stars \citep{Skumanich1972, Pal81}. This is
believed to be a result of faster rotating stars exhibiting excess non-thermal
emission from the upper chromosphere or corona when compared to their slower
rotating counterparts. This excess emission is due to magnetic heating of the
upper atmosphere, driven by the underlying stellar dynamo.
The faster a star rotates, up to some saturation threshold, the more such emission is expected. However,
the dynamo process is not dependent solely on rotation; rather, it depends on
whether the contribution from the rotational period ($P_{rot}$) or convective
motion --- parameterized by the convective overturn time scale ($\tau_{c}$) ---
dominates the motion of a charge packet within a star. Therefore, the Rossby
Number ($Ro = P_{rot}/\tau_{c}$) is often used in place of the rotational
period as it accounts for both.

The rotation-activity relation was first discovered using the ratio of X-ray
luminosity to bolometric luminosity ($L_{X}/L_{bol}$) \citep{Pal81} and was
later demonstrated to be a more general phenomenon, observable through other
activity tracers, such as Ca II H\&K emission \citep{Vilhu1984}. This relation has
a number of important structural elements. \citet{Noy84} showed that magnetic
activity as a function of Rossby Number is well modeled as a piecewise power
law relation including a saturated and non-saturated regime. In the saturated
regime, magnetic activity is invariant to changes in Rossby Number; in the
non-saturated regime, activity decreases as Rossby Number increases. The
transition between the saturated and non-saturated regions occurs at $Ro \sim
0.1$ \citep[e.g.][]{Wri11}. Recent evidence may suggest that, instead of an
unsaturated region where activity is fully invariant to rotational period,
activity is more weakly, but still positively, correlated with rotation rate
\citep{Mamajek2008, Reiners2014, Leh20, Magaudda2020}. 

Previous studies of the Ca II H\&K rotation-activity relation
\citep[e.g.][]{Vau81, Sua15, Def17, Hou17} have focused on on spectral ranges
which both extend much earlier than M-dwarfs and which do not fully probe late
M-dwarfs. Other studies have relied on $v\sin(i)$ measurements
\citep[e.g.][]{Browning2010, Hou17}, which are not sensitive to the long
rotation periods reached by slowly rotating, inactive mid-to-late type M dwarfs
\citep[70-150 days:][]{Newton2016}. Therefore, these studies can present only
coarse constraints on the rotation activity relation in the fully convective
regime. The sample we present in this paper is focused on mid-to-late type M
dwarfs, with photometrically measured rotational periods, while maintaining of
order the same number of targets as previous studies.  Consequently, we provide
much finer constraints on the rotation-activity relation in this regime. 

One example of an application of the rotation-activity relation is as a means
of approximating stellar ages. Because as stars spin down, they move along
the rotation-activity relation \citep{Soderblom1991}. To calibrate this
relation, however, one needs a priori knowledge of a star's age and therefore
stars need to be in clusters where population statistics may be used to
accurately measure ages. This has proved doable for FGK type stars; however,
as is often the case, M-dwarfs pose some unique challenges. 

Firstly, the sample of all clusters in which we can observe M-dwarfs is
extremely small due to M-dwarfs' low luminosities. Secondly, open clusters
preferentially contain stars younger than the characteristic time it takes an
M-dwarf to spin down out of the saturated regime of the rotation-activity
relation \citep{West2009, Newton2016, Giacobbe2020}. Therefore, even in the
small set of clusters with measured ages and that contain observable
mid-to-late M-dwarfs, the unsaturated regimes in the rotation-activity
relation is not present. Currently, their has not been a successful
demonstration of using the M-dwarf rotation-activity relation to measure
ages.

We present a high resolution spectroscopic study of 53 mid-late M-dwarfs. We
measure Ca II H\&K strengths, quantified through the $R'_{HK}$ metric, which is
a bolometric flux normalized version of the Mount Wilson S-index. These
activity tracers are then used in concert with photometrically determined
rotational periods, compiled by \citet{Newton2017}, to generate a
rotation--activity relation for our sample. This paper is organized as follows:
Section \ref{sec:Observations} provides an overview of the observations and
data reduction, Section \ref{sec:Analysis} details the analysis of our data,
and Section \ref{sec:results} presents our results and how they fit within the
literature. 
