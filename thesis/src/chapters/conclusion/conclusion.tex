\chapter{So Long and Thanks for all the Fish}
Stellar models provide an extremley powerful tool to understand the universe.
They allow astronomers to probe the interiors of stars, infering physics based
on the limited number of observables we are granted. In this thesis I have
presented five projects which let us probe the physics of either single stars
or stellar populations.

First, I disscused multiple populations in the Globular Cluster NGC 2808 and
presented estimates for both the number of populations in the cluster and the
helium enhancement between populations. Unlike previous literature we only find
statistical evidence for two populations in the color-magnitude diagram for NCG
2808. For these two populations we find that models which maintain chemical
consistency through their radiative opacity, stellar strucutre, and atmospheric
boundary consitions predict helium enhancement's which are inagreement with
literature values (which have generally not been evolved in a chemically
self-consistent manner). This work, along with previous work by
\citet{Dotter2016} self-consistently modelling NGC 6752 seem to indicate that
full chemical self-consistency does not, at least at our current level of
observational constraint, have a signifigant impact on infered helium abundance
when compared to non self-consistent modeling. Future work in this regime may
focus on confirming this weak effect for other clusters and for clusters at
different metallicities.

Following the work with NGC 2808 this thesis spent two chapters centered on the
Gaia M Dwarf Gap. The Gap provides a rare oppertunity to study the interior
physics of stars for which other methods, such as seismology, are not feasible.
However, the gap still reativley newley discovered and models have not yet been
developed which fully capture its structure. If astronomers want to actually
make use of the gap to study the interiors of stars it is essential to first
have models which recreate its stucture. In the first chapter focused on the
gap we show that updated opacities are not in and of themselves sufficnent for
DSEP to recreate the gap structure. Specifically while we do see a minor
improvment in agreement of gap location between models evolved with OPLIB
compared to those evolved with OPAL we are unable to model the color dependency
of the gap. 

The second Gaia M Dwarf gap project investigates a novel correlation between
the gap and Ca II H\&K emission. Wheras previous work by \citet{Jao2023}
discovered a paucity of $H\alpha$ active stars near the gap magnitude, here we
discover a similar trend in Ca II H\&K emission. Moreover, using a random walk
based toy model we show that the underlying physics belived to cause the gap,
convective-kissing instabilities, could produce a similar qualitative trend in
activity to what we see in our dataset. This toy model is extemley simplistic
and in order to falsify the model that convective-kissing instabilities can
result in the increase in activity spread 3D magneto hydrodynamical models need
to be developed. The challenge here is that thermal time scale resolutions are
required  in order to resolve the rapid changed to magnetic toplogy driven
stellar structure changes while a temporal baseline of gigayears is required to
see the effects of multiple mixing events. 

Following the chapter studying correlations between Ca II H\&K emission to the
Gaia M dwarf gap we present a project studying the rotation-activity relation
in 53 mid-to-late type M dwarfs. This project constiutes the only purley
observational project in this thesis and the results which we found in this
work underpinned much of the previous project. We find that the maximal
magnetic activity which the M Dwarfs in our sample reach is in agreement with
previous literature.

Finally, we present a breif study of how various initial conditions may effect
the location and the magnitude of the Red Giant Branch bump. First we
investigate how the red giant branch bump may vary from one population in NGC
2808 to the other, finding that models of population E (the helium enhanced
population) do not predict a red giant branch bump at all, wherease models of
population A predict a bump at a luminosity consistent with observations of NGC
2808. This is attibutatble to the generally higher metallicity and consequently
higher opacities in population E which prevent the convective envelope reaching
as deep into the streucture of star during shell burning. Second, we study how
the updated OPLIB opacities effect the bump location, finding that while there
is a change to the identified bump location it is relativley minial and in the
wrong direction to explain the {\color{red} diagreement between models of the
bump location at for low metallicity populations when compared to
observations.}

This thesis has combined work which has been published in four seperate
articles along with a project which has not yet been submitted to a journal for
publication. Throughout this thesis we have focused on how models of low mass
stars may provide a controlled labratory for physics and we have presented
projects which further researchers ability to make these controlled statements
in the future. While this thesis has presented work which furthers the ability
to do controoled reaserch with both stellar models and obserbations of stars,
there is much work left to be done, from pinning down the exact formation
channels which lead to multiple populations in globular clusters, to explaining
the color dependency of the Gaia M dwarf gap.

