\section{Correlation}\label{sec:results}
Using Ca II H\&K emission data from \citet{Perdelwitz2021} and
\citet{Boudreaux2022} \citep[quantified using the $R'_{HK}$
metric][]{Middelkoop1982, Rutten1984} we investigate the correlation between
the Jao Gap magnitude and stellar magnetic activity. We are more statistically
limited here than past authors have been due to the requirement for high
resolution spectroscopic data when measuring Calcium emission.

The merged dataset is presented in Figure \ref{fig:initData}. The
sample overlap between \citet{Perdelwitz2021} and \citet{Boudreaux2022} is
small (only consisting of five targets). For those five targets there is an
approximately 1.5 percent average difference between measured $\log(R'_{HK})$
values with measurements from \citeauthor{Boudreaux2022} biased to be slightly
more negative than those from \citeauthor{Perdelwitz2021}

There is a visual discontinuity in the spread of stellar activity
below the Jao Gap magnitude. Further discussion of why there may be
disagreement between the observed magnitude of the gap and the discontinuity
which we identify may be found in Section \ref{sec:coincident}. In order to
quantify the significance of this discontinuity we measure the false alarm
probability of the change in standard deviation.

\begin{figure}
  \centering
  \includegraphics[width=0.85\textwidth]{figures/jaoMagActivity/CombinedDeviation.pdf}
  \caption{Merged Dataset from \citet{Perdelwitz2021, Boudreaux2022}. Note the
  increase in the spread of $R'_{HK}$ around the Jao Gap Magnitude (top).
  Standard deviation of Calcium emission data within each bin. Note the
  discontinuity near the Jao Gap Magnitude (bottom). The location of the Gap
  as identified in literature is shown by the hatched region ($\sim$ 10-10.5 $M_{G}$). Potential
  explanations for the disagreement in magnitude are discussed in detail
  in Section \ref{sec:coincident}.}
  \label{fig:initData}
\end{figure}

First we split the merged dataset into bins with a width of 0.5 mag. In each bin we
measure the standard deviation about the mean of the data. The results of this
are shown in Figure \ref{fig:initData} (bottom). In order to measure the false alarm
probability of this discontinuity we first resample the merged calcium
emission data based on the associated uncertainties for each datum as
presented in their respective publications. Then, for each of these ``resample
trials'' we measure the probability that a change in the standard deviation of
the size seen would happen purely due to noise. Results of this test are show in
in Figure \ref{fig:dist}. 

\begin{figure}
  \centering
  \includegraphics[width=0.85\textwidth]{figures/jaoMagActivity/fpDist.pdf}
  \caption{Probability distribution of the false alarm probability for the
  discontinuity seen in Figure \ref{fig:initData}. The mean of this
  distribution is $0.341\%\pm^{0.08}_{0.08}$.}
  \label{fig:dist}
\end{figure}

This rapid increase star-to-star variability would only arise due purely to
noise $0.3\pm0.08$ percent of the time and is therefore likely either a true
effect or an alias of some sample bias.

If the observed increase in variability is not due to a sample bias and rather
is a physically driven effect then there is an obvious similarity between these
findings and those of \citet{Jao2023}. Specifically we find a increase in
variability below the magnitude of the Gap. Moreover, this variability
increase is primarily driven by an increase in the number of low activity stars
(as opposed to an increase in the number of high activity stars). We can
further investigate the observed change in variability for only low activity
stars by filtering out those stars at or above the saturated threshold for
magnetic activity. \citet{Boudreaux2022} identify $\log(R'_{HK}) = -4.436$ as
the saturation threshold. We adopt this value and filter out all stars where
$\log(R'_{HK}) \geq -4.436$. Applying the same analysis to this reduced dataset
as was done to the full dataset we still find a discontinuity at the same
location (Figure \ref{fig:reduced}). This discontinuity is of a smaller
magnitude and consequently is more likely to be due purely to noise, with a
$7\pm0.2$ percent false alarm probability. This false alarm probability is
however only concerned with the first point after the jump in variability. If
we consider the false alarm probability of the entire high variability region
then the probability that the high variability region is due purely to noise
drops to $1.4\pm0.04$ percent.

\begin{figure}
  \centering
  \includegraphics[width=0.85\textwidth]{figures/jaoMagActivity/ReducedDeviation.pdf}
  \caption{Spread in the magnetic activity metric for the merged sample with
  any stars $\log(R'_{HK}) > -4.436$ filtered out.The location of the Gap
  as identified in literature is shown by the hatched region ($\sim$ 10-10.5 $M_{G}$).}
  \label{fig:reduced}
\end{figure}

Further, various authors have shown that the strength of Calcuium II
H\&K emission may evolve over month to year timescales
\citep[e.g.][]{Rauscher2006, Perdelwitz2021, Cretignier2024}. Targets from
\citet{Boudreaux2022} were observed an average of only four times and over year
long timescales. Therefore, the nominal $\log(R'_{HK})$ values derived in that
work may be biased by stellar variability. However, the scale of observed
variabilty in the activity metric is signifigantly smaller than the
star-to-star activity variability addressed here and therfore activity cycles
are not expected to be of particular relevance. Specifically, the amplitude of
variability is generally $\Delta \log(R'_{HK}) \lessapprox 0.2$ wherase in this
work we address variability on the order of $\Delta \log(R'_{HK}) \lessapprox 2$.

We observe a strong, likely statistically significant, discontinuity in the
star-to-star variability of Ca II H\&K emission below the magnitude
of the Jao Gap. However, modeling is required to determine if this discontinuity
may be due to the same underlying physics.

\subsection{Conicidence with the Jao Gap Magnitude}\label{sec:coincident}
While the observed increase in variability seen here does not seem to be
coincident with the Jao Gap --- instead appearing to be approximately 0.5 mag
fainter, in agreement with what is observed in \citet{Jao2023} --- a number of
complicating factors prevent us from falsifying that the these two features are
not coincident. \citeauthor{Jao2023} find, similar to the results presented
here, that the paucity of $H\alpha$ emission originates below the Gap.
Moreover, we use a 0.5 magnitude bin size when measuring the star-to-star
variability which injects error into the positioning of any feature in
magnitude space. We can quantify the degree of uncertainty the magnitude bin
choice injects by conducting Monte Carlo trials where bins are randomly shifted
redder or bluer. We conduct 10,000 trials where each trial involves sampling a
random shift to the bin start location from a normal distribution with a
standard deviation of 1 magnitude. For each trial we identify the discontinuity
location as the maximum value of the gradient of the standard deviation (this
is the derivative of the data in Figures \ref{fig:initData} \&
\ref{fig:reduced}). Some trials result in the maximal value lying at the 0th
index of the magnitude array due to edge effects, these trials are rejected
(and account for 11\% of the trials). The uncertainty in the identified
magnitude of the discontinuity due to the selected start point of the magnitude
bins reveals a $1\sigma = \pm$0.32 magnitude uncertainty in the location of the
discontinuity (Figure \ref{fig:GapLocationMC}). Finally, all previous studies
of the M dwarf Gap \citep{Jao2018, Jao2021, Mansfield2021, Boudreaux2022,
Jao2023} demonstrate that the Gap has a color dependency, shifting to fainter
magnitudes as the population reddens and consequently an exact magnitude range
is ill-defined. Therefore we cannot falsify the model that the discontinuity in
star-to-star activity variability is coincident with the Jao Gap magnitude.

\begin{figure}
  \centering
  \includegraphics[width=0.85\textwidth]{figures/jaoMagActivity/GapLocationMC.pdf}
  \caption{Probability density distribution of discontinuity location as
  identified in the merged dataset. The dashed line represents the mean of the
  distribution while the shaded region runs from the 16th percentile to the
  84th percentile of the distribution. This distribution was built from 10,000
  independent samples where the discontinuity was identified as the highest
  value in the gradient of the standard deviation.The location of the Gap
  as identified in literature is shown by the hatched region ($\sim$ 10-10.5 $M_{G}$).}
  \label{fig:GapLocationMC}
\end{figure}

\subsection{Rotation}
It is well known that star's magnetic activity tend to be correlated with their
rotational velocity \citep{Vaughan1981, Newton2016, Astudillo-Defru2017,
Houdebine2017, Boudreaux2022}; therefore, we investigate whether there is a
similar correlation between Gap location and rotational period in our dataset.
All targets from \citet{Boudreaux2022} already have published rotational
periods; however, targets from \citet{Perdelwitz2021} do not necessarily have
published periods. Therefore, in previously published work collaborators of ours have derived photometric rotational periods for
these targets \citep{Boudreaux2024}. Given the inherent heterogeneity of M Dwarf stellar
surfaces \citep{Boisse2011, Robertson2020} one is able to determine the
rotational period of a star through the analysis of active regions. Various
methodologies can be employed for this purpose, including the examination of
photometry and light curves \citep[e.g.,][]{Newton2016}, and the observation of
temporal changes in the strength of chromospheric emission lines such as Ca II
H \& K or H$\alpha$ \citep[e.g.,][]{2019A&A...623A..24F,2023MNRAS.518.3147K}.
Photometric periods for targets from \citet{Perdelwitz2021} were determined in \citet{Boudreaux2024} by \href{https://orcid.org/0000-0001-9828-3229}{Aylin Garcia Soto}. All rotational periods are presented in Table \ref{tab:dataset}. Our
final sample contains 187 stars with measured rotational periods. We derive new
rotational periods for 7 of these. 

{\footnotesize
  \begin{longtable}{ccccccccc}
  \caption{All data used from P2021 and B2022.}\label{tab:dataset} \\
  \hline
  ID & G Mag & V Mag & K Mag & $\log(R'_{HK})$ & e\_Log(R'$_{HK}$) & Ro & prot & r\_prot \\
   & $\mathrm{mag}$ & $\mathrm{mag}$ & $\mathrm{mag}$ &  &  &  & $\mathrm{d}$ & \\
  \hline
  \endfirsthead  % Header for the first page

  \multicolumn{9}{c}{{\bfseries \tablename\ \thetable{} -- continued from previous page}} \\
  \hline
  ID & G Mag & V Mag & K Mag & $\log(R'_{HK})$ & e\_Log(R'$_{HK}$) & Ro & prot & r\_prot \\
   & $\mathrm{mag}$ & $\mathrm{mag}$ & $\mathrm{mag}$ &  &  &  & $\mathrm{d}$ & \\
  \hline
  \endhead  % Header for the rest of the pages

  \hline
  \multicolumn{9}{r}{{Continued on next page}} \\ 
  \endfoot  % Footer for all but the last page

  \hline
  \endlastfoot  % Footer for the last page
    2MASS J00094508-4201396 & 12.14 & 13.659 & 8.223 & -4.339 & 0.001 & 0.009 & 0.859 & Bou22 \\
    2MASS J00310412-7201061 & 12.301 & 13.648 & 8.445 & -5.388 & 0.003 & 0.928 & 80.969 & Bou22 \\
    2MASS J01040695-6522272 & 12.447 & 13.95 & 8.532 & -4.489 & 0.001 & 0.006 & 0.624 & Bou22 \\
    2MASS J02004725-1021209 & 12.778 & 14.113 & 9.092 & -4.791 & 0.001 & 0.188 & 14.793 & Bou22 \\
    2MASS J02014384-1017295 & 13.026 & 14.477 & 9.189 & -4.54 & 0.001 & 0.034 & 3.152 & Bou22 \\
    2MASS J02125458+0000167 & 12.096 & 13.58 & 8.168 & -4.635 & 0.001 & 0.048 & 4.732 & Bou22 \\
    2MASS J02411510-0432177 & 12.251 & 13.79 & 8.246 & -4.427 & 0.001 & 0.004 & 0.4 & Bou22 \\
    2MASS J03100305-2341308 & 12.23 & 13.5 & 8.567 & -4.234 & 0.001 & 0.028 & 2.083 & Bou22 \\
    2MASS J03205178-6351524 & 12.087 & 13.433 & 8.195 & -5.629 & 0.004 & 1.029 & 91.622 & Bou22 \\
    2MASS J05015746-0656459 & 10.649 & 12.196 & 6.736 & -5.005 & 0.002 & 0.875 & 88.5 & Bou22 \\
    2MASS J06000351+0242236 & 9.901 & 11.31 & 6.042 & -4.547 & 0.002 & 0.02 & 1.809 & Bou22 \\
    2MASS J06022261-2019447 & 12.02 & 13.289 & 8.374 & -5.698 & 0.006 & 1.285 & 95.0 & Bou22 \\
    2MASS J06105288-4324178 & 10.996 & 12.23 & 7.312 & -5.251 & 0.004 & 0.726 & 53.736 & Bou22 \\
    2MASS J07401183-4257406 & 12.031 & 13.835 & 7.768 & -4.336 & 0.001 & 0.002 & 0.307 & Bou22 \\
    2MASS J08184619-4806172 & 13.092 & 14.37 & 9.351 & -4.283 & 0.001 & 0.021 & 1.653 & Bou22 \\
    2MASS J08443891-4805218 & 12.434 & 13.932 & 8.562 & -5.668 & 0.002 & 1.348 & 129.513 & Bou22 \\
    2MASS J09342791-2643267 & 12.544 & 13.992 & 8.619 & -4.341 & 0.001 & 0.007 & 0.694 & Bou22 \\
    2MASS J09524176-1536137 & 12.173 & 13.463 & 8.507 & -5.632 & 0.003 & 1.317 & 99.662 & Bou22 \\
    2MASS J11075025-3421003 & 13.828 & 15.02 & 9.407 & -4.225 & 0.001 & 0.069 & 7.611 & Bou22 \\
    2MASS J11381671-7721484 & 12.816 & 14.78 & 8.521 & -5.503 & 0.003 &  &  &  \\
    2MASS J11575352-2349007 & 13.28 & 14.77 & 9.355 & -4.295 & 0.001 & 0.031 & 3.067 & Bou22 \\
    2MASS J12102834-1310234 & 12.298 & 13.801 & 8.412 & -4.689 & 0.001 & 0.443 & 42.985 & Bou22 \\
    2MASS J12384914-3822527 & 11.226 & 12.74 & 7.386 & -6.069 & 0.004 &  &  &  \\
    2MASS J12440075-1110302 & 12.635 & 14.18 & 8.674 & -4.405 & 0.001 & 0.02 & 2.099 & Bou22 \\
    2MASS J13442092-2618350 & 11.984 & 13.253 & 8.285 & -5.963 & 0.005 & 2.032 & 154.885 & Bou22 \\
    2MASS J14211512-0107199 & 11.82 & 12.95 & 8.093 & -5.584 & 0.004 & 1.279 & 91.426 & Bou22 \\
    2MASS J14253413-1148515 & 12.237 & 13.51 & 8.391 & -4.764 & 0.001 & 0.301 & 25.012 & Bou22 \\
    2MASS J14294291-6240465 & 8.985 & 11.13 & 4.384 & -5.005 & 0.001 & 0.394 & 83.5 & Bou22 \\
    2MASS J14340491-1824106 & 12.738 & 14.32 & 8.708 & -4.609 & 0.001 & 0.275 & 30.396 & Bou22 \\
    2MASS J15165576-0037116 & 13.001 & 14.472 & 9.105 & -4.07 & 0.007 &  &  &  \\
    2MASS J15290145-0612461 & 12.601 & 14.01 & 8.781 & -4.201 & 0.001 & 0.095 & 8.434 & Bou22 \\
    2MASS J16204186-2005139 & 12.326 & 13.801 & 8.419 & -4.39 & 0.001 & 0.029 & 2.814 & Bou22 \\
    2MASS J16352464-2718533 & 12.751 & 14.2 & 8.998 & -5.596 & 0.002 & 1.407 & 122.656 & Bou22 \\
    2MASS J16570570-0420559 & 10.913 & 12.25 & 7.12 & -4.307 & 0.001 & 0.014 & 1.212 & Bou22 \\
    2MASS J18494929-2350101 & 9.126 & 10.43 & 5.37 & -4.524 & 0.001 & 0.036 & 2.869 & Bou22 \\
    2MASS J19204795-4533283 & 10.753 & 12.23 & 6.845 & -5.839 & 0.004 &  &  &  \\
    2MASS J20035892-0807472 & 12.15 & 13.4 & 8.286 & -5.653 & 0.004 & 1.025 & 84.991 & Bou22 \\
    2MASS J20091824-0113377 & 12.673 & 14.45 & 8.512 & -4.377 & 0.001 & 0.01 & 1.374 & Bou22 \\
    2MASS J21390081-2409280 & 12.078 & 13.4 & 8.359 & -6.194 & 0.008 & 1.186 & 94.254 & Bou22 \\
    2MASS J22480446-2422075 & 11.129 & 12.624 & 7.206 & -4.412 & 0.001 & 0.005 & 0.466 & Bou22 \\
    2MASS J23071524-2307533 & 12.376 & 13.674 & 8.738 & -5.278 & 0.003 & 0.684 & 51.204 & Bou22 \\
    Karmn J00184+440 & 9.687 & 11.04 & 5.948 & -5.439 & 0.0 &  &  &  \\
    Karmn J01013+613 & 9.846 & 10.87 & 6.483 & -5.171 & 0.0 & 0.636 & 34.7 & SM18 \\
    Karmn J01125-169 & 10.437 & 12.074 & 6.42 & -4.585 & 0.0 & 0.612 & 69.2 & DA19 \\
    Karmn J01433+043 & 9.907 & 10.915 & 6.516 & -4.997 & 0.0 &  &  &  \\
    Karmn J02002+130 & 10.681 & 12.298 & 6.648 & -5.131 & 0.0 &  &  &  \\
    Karmn J02123+035 & 9.206 & 10.042 & 6.077 & -5.126 & 0.0 &  &  &  \\
    Karmn J02358+202 & 9.709 & 10.628 & 6.328 & -4.86 & 0.0 & 0.615 & 31.9 & DA19 \\
    Karmn J02442+255 & 9.467 & 10.563 & 5.961 & -5.166 & 0.0 & 0.627 & 38.7 & DA19 \\
    Karmn J03133+047 & 12.101 & 14.001 & 7.833 & -4.602 & 0.0 & 0.83 & 126.2 & New16a \\
    Karmn J03181+382 & 9.415 & 10.28 & 6.164 & -4.961 & 0.0 & 1.654 & 77.2 & DA19 \\
    Karmn J03217-066 & 10.362 & 11.314 & 6.983 & -4.472 & 0.0 & 0.4 & 21.1 & DA19 \\
    Karmn J03473-019 & 10.506 & 11.537 & 6.933 & -4.026 & 0.0 & 0.063 & 3.88 & Rev20 \\
    Karmn J04290+219 & 7.704 & 8.3 & 4.875 & -4.429 & 0.0 & 0.81 & 25.4 & DA19 \\
    Karmn J04376+528 & 7.953 & 8.648 & 5.047 & -4.39 & 0.0 &  &  &  \\
    Karmn J04376-110 & 9.408 & 10.331 & 6.091 & -5.034 & 0.0 &  &  &  \\
    Karmn J04429+189 & 9.004 & 9.951 & 5.607 & -4.806 & 0.0 & 0.765 & 40.7 & DA19 \\
    Karmn J04429+214 & 10.727 & 11.93 & 7.101 & -5.841 & 0.0 & 0.679 & 47.8 & DA19 \\
    Karmn J04520+064 & 10.714 & 12.018 & 6.942 & -5.007 & 0.0 &  &  &  \\
    Karmn J04538-177 & 9.938 & 10.894 & 6.598 & -5.081 & 0.0 &  &  &  \\
    Karmn J05019+011 & 11.513 & 12.859 & 7.677 & -4.068 & 0.0 & 0.024 & 2.08 & Rev20 \\
    Karmn J05019-069 & 10.649 & 12.196 & 6.736 & -4.779 & 0.0 & 0.875 & 88.5 & Kir12 \\
    Karmn J05033-173 & 10.562 & 11.735 & 6.936 & -5.095 & 0.0 &  &  &  \\
    Karmn J05062+046 & 11.992 & 13.46 & 8.067 & -4.658 & 0.0 & 0.009 & 0.9 & Rev20 \\
    Karmn J05127+196 & 9.815 & 10.746 & 6.47 & -5.148 & 0.0 &  &  &  \\
    Karmn J05314-036 & 7.108 & 7.968 & 3.9 & -4.67 & 0.0 & 0.745 & 33.8 & DA19 \\
    Karmn J05365+113 & 8.195 & 8.898 & 5.269 & -4.001 & 0.0 & 0.326 & 11.51 & CC23 \\
    Karmn J05366+112 & 11.197 & 12.58 & 7.406 & -4.164 & 0.0 &  &  &  \\
    Karmn J05421+124 & 10.114 & 11.509 & 6.389 & -5.068 & 0.0 &  &  &  \\
    Karmn J06000+027 & 9.901 & 11.31 & 6.042 & -4.274 & 0.0 & 0.02 & 1.8 & Rev20 \\
    Karmn J06105-218 & 7.313 & 8.125 & 4.17 & -4.594 & 0.0 & 0.642 & 27.3 & DA19 \\
    Karmn J06371+175 & 8.878 & 9.593 & 5.862 & -4.871 & 0.0 &  &  &  \\
    Karmn J07274+052 & 8.576 & 9.872 & 4.857 & -5.301 & 0.0 & 1.194 & 93.5 & SM17b \\
    Karmn J07287-032 & 10.296 & 11.456 & 6.704 & -5.26 & 0.0 &  &  &  \\
    Karmn J07319+362N & 10.506 & 11.738 & 6.755 & -4.29 & 0.0 & 0.213 & 16.4 & DA19 \\
    Karmn J07386-212 & 10.611 & 11.712 & 7.063 & -5.276 & 0.0 &  &  &  \\
    Karmn J07393+021 & 8.898 & 9.589 & 5.872 & -4.523 & 0.0 &  &  &  \\
    Karmn J07582+413 & 10.666 & 13.912 & 6.878 & -5.335 & 0.0 &  &  &  \\
    Karmn J08126-215 & 10.672 & 12.13 & 6.705 & -4.977 & 0.0 &  &  &  \\
    Karmn J08161+013 & 9.133 & 10.091 & 5.766 & -5.051 & 0.0 & 0.773 & 40.7 & DA19 \\
    Karmn J08293+039 & 10.443 & 11.437 & 7.026 & -4.509 & 0.0 &  &  &  \\
    Karmn J08409-234 & 10.765 & 11.975 & 7.028 & -5.041 & 0.0 &  &  &  \\
    Karmn J09161+018 & 11.689 & 13.03 & 7.96 & -3.772 & 0.0 & 0.017 & 1.4 & Rev20 \\
    Karmn J09163-186 & 9.807 & 10.739 & 6.492 & -4.559 & 0.0 &  &  &  \\
    Karmn J09307+003 & 10.502 & 11.71 & 6.871 & -5.225 & 0.0 &  &  &  \\
    Karmn J09360-216 & 9.892 & 10.906 & 6.475 & -5.448 & 0.0 & 1.328 & 74.3 & SM15 \\
    Karmn J09411+132 & 9.443 & 10.37 & 6.128 & -4.824 & 0.0 &  &  &  \\
    Karmn J09439+269 & 10.849 & 13.962 & 7.19 & -5.235 & 0.0 & 0.064 & 13.7 & DA19 \\
    Karmn J09447-182 & 11.082 & 12.455 & 7.257 & -5.399 & 0.0 &  &  &  \\
    Karmn J09511-123 & 9.242 & 10.016 & 6.15 & -4.773 & 0.0 &  &  &  \\
    Karmn J09561+627 & 8.231 & 9.18 & 5.2 & -4.392 & 0.0 & 0.336 & 14.501 & Bou24 \\
    Karmn J10122-037 & 8.332 & 9.264 & 5.015 & -4.555 & 0.0 & 0.429 & 21.6 & DA19 \\
    Karmn J10167-119 & 9.934 & 10.998 & 6.452 & -5.037 & 0.0 &  &  &  \\
    Karmn J10251-102 & 9.276 & 10.139 & 6.032 & -4.583 & 0.0 & 0.226 & 10.483 & Bou24 \\
    Karmn J10289+008 & 8.676 & 9.65 & 5.311 & -4.957 & 0.0 & 0.617 & 32.72 & CC23 \\
    Karmn J10350-094 & 10.991 & 12.11 & 7.393 & -5.203 & 0.0 &  &  &  \\
    Karmn J10396-069 & 10.27 & 11.281 & 6.818 & -4.993 & 0.0 &  &  &  \\
    Karmn J10508+068 & 10.291 & 11.675 & 6.371 & -5.041 & 0.0 & 0.692 & 64.0 & DA19 \\
    Karmn J11000+228 & 8.977 & 10.02 & 5.503 & -5.124 & 0.0 & 0.092 & 5.429 & Bou24 \\
    Karmn J11026+219 & 8.811 & 9.572 & 5.688 & -5.073 & 0.0 & 0.331 & 13.53 & CC23 \\
    Karmn J11033+359 & 6.551 & 7.52 & 3.34 & -5.284 & 0.0 & 0.991 & 48.0 & Kir07 \\
    Karmn J11110+304W & 9.055 & 9.983 & 5.734 & -4.688 & 0.0 &  &  &  \\
    Karmn J11126+189 & 9.884 & 10.85 & 6.579 & -4.587 & 0.0 &  &  &  \\
    Karmn J11302+076 & 12.401 & 13.5 & 8.899 & -4.827 & 0.0 & 0.63 & 38.84 & CC23 \\
    Karmn J11417+427 & 10.582 & 13.795 & 6.822 & -5.321 & 0.0 & 0.296 & 71.5 & DA19 \\
    Karmn J11421+267 & 9.582 & 10.613 & 6.073 & -5.274 & 0.0 & 0.749 & 44.6 & DA19 \\
    Karmn J11467-140 & 10.616 & 11.7 & 7.069 & -4.75 & 0.0 &  &  &  \\
    Karmn J11476+002 & 11.967 & 13.34 & 8.098 & -4.452 & 0.0 & 0.13 & 11.6 & DA19 \\
    Karmn J11477+008 & 9.601 & 11.153 & 5.654 & -5.194 & 0.0 & 1.576 & 163.0 & DA19 \\
    Karmn J11511+352 & 8.854 & 9.802 & 5.606 & -4.579 & 0.0 & 0.467 & 22.8 & DA19 \\
    Karmn J12100-150 & 10.742 & 12.083 & 6.863 & -5.365 & 0.0 &  &  &  \\
    Karmn J12111-199 & 10.588 & 11.709 & 7.044 & -5.357 & 0.0 &  &  &  \\
    Karmn J12189+111 & 11.926 & 13.9 & 7.57 & -4.705 & 0.0 & 0.003 & 0.5 & DA19 \\
    Karmn J12230+640 & 10.552 & 13.481 & 7.122 & -5.072 & 0.0 & 0.194 & 32.9 & DA19 \\
    Karmn J12248-182 & 10.29 & 11.272 & 6.95 & -5.339 & 0.0 &  &  &  \\
    Karmn J12312+086 & 8.951 & 9.664 & 5.892 & -4.529 & 0.0 &  &  &  \\
    Karmn J12350+098 & 10.453 & 11.405 & 7.153 & -5.177 & 0.0 & 0.725 & 36.6 & DA19 \\
    Karmn J12373-208 & 11.958 & 13.28 & 8.085 & -5.344 & 0.0 &  &  &  \\
    Karmn J12388+116 & 10.336 & 11.502 & 6.691 & -5.03 & 0.0 &  &  &  \\
    Karmn J12479+097 & 10.105 & 11.395 & 6.362 & -5.463 & 0.0 &  &  &  \\
    Karmn J13283-023W & 10.186 & 11.336 & 6.613 & -5.189 & 0.0 & 0.701 & 46.42 & Rae20 \\
    Karmn J13299+102 & 8.205 & 9.029 & 5.036 & -4.81 & 0.0 & 0.69 & 30.0 & SM17b \\
    Karmn J13427+332 & 10.72 & 13.896 & 6.982 & -5.147 & 0.0 &  &  &  \\
    Karmn J13450+176 & 9.108 & 9.75 & 6.22 & -4.964 & 0.0 &  &  &  \\
    Karmn J13457+148 & 7.611 & 8.5 & 4.415 & -5.065 & 0.0 & 1.141 & 52.3 & SM15 \\
    Karmn J13458-179 & 10.598 & 11.872 & 6.902 & -5.24 & 0.0 &  &  &  \\
    Karmn J13591-198 & 11.396 & 12.88 & 7.445 & -4.248 & 0.0 & 0.033 & 3.3 & Rev20 \\
    Karmn J14010-026 & 8.863 & 9.707 & 5.683 & -4.733 & 0.0 & 0.992 & 43.9 & SM17b \\
    Karmn J14257+236E & 9.205 & 9.973 & 6.091 & -4.654 & 0.0 & 0.432 & 17.6 & DA19 \\
    Karmn J14257+236W & 9.006 & 9.72 & 5.973 & -4.553 & 0.0 & 2.942 & 111.0 & DA19 \\
    Karmn J14294+155 & 9.733 & 10.676 & 6.393 & -4.929 & 0.0 & 0.847 & 43.5 & SM18 \\
    Karmn J14307-086 & 8.721 & 9.392 & 5.769 & -4.609 & 0.0 &  &  &  \\
    Karmn J14310-122 & 10.668 & 11.903 & 6.961 & -5.581 & 0.0 &  &  &  \\
    Karmn J14342-125 & 9.895 & 11.317 & 5.939 & -5.264 & 0.0 &  &  &  \\
    Karmn J15013+055 & 11.075 & 12.265 & 7.432 & -5.113 & 0.0 & 0.06 & 4.223 & Bou24 \\
    Karmn J15095+031 & 10.378 & 11.473 & 6.858 & -5.128 & 0.0 & 0.06 & 3.756 & Bou24 \\
    Karmn J15194-077 & 9.422 & 10.56 & 5.837 & -5.698 & 0.0 & 2.002 & 132.5 & SM15 \\
    Karmn J15218+209 & 9.124 & 10.003 & 5.756 & -4.015 & 0.0 & 0.068 & 3.4 & Rev20 \\
    Karmn J15369-141 & 11.369 & 12.726 & 7.572 & -5.608 & 0.0 &  &  &  \\
    Karmn J15598-082 & 9.601 & 10.487 & 6.343 & -4.58 & 0.0 & 0.422 & 20.0 & SM18 \\
    Karmn J16028+205 & 11.167 & 12.56 & 7.369 & -5.123 & 0.0 &  &  &  \\
    Karmn J16254+543 & 9.14 & 10.07 & 5.833 & -5.331 & 0.0 & 1.535 & 76.8 & DA19 \\
    Karmn J16303-126 & 8.802 & 10.072 & 5.075 & -5.24 & 0.0 & 1.536 & 119.0 & DA19 \\
    Karmn J16327+126 & 11.072 & 12.158 & 7.675 & -5.287 & 0.0 &  &  &  \\
    Karmn J16462+164 & 10.592 & 11.68 & 7.094 & -5.115 & 0.0 &  &  &  \\
    Karmn J16554-083N & 10.444 & 11.759 & 6.724 & -5.253 & 0.0 & 0.082 & 6.52 & DA19 \\
    Karmn J16570-043 & 10.913 & 12.25 & 7.12 & -4.425 & 0.0 & 0.007 & 0.55 & DA19 \\
    Karmn J16581+257 & 8.824 & 9.655 & 5.624 & -4.541 & 0.0 & 0.536 & 23.8 & DA19 \\
    Karmn J17052-050 & 9.178 & 10.071 & 5.975 & -5.076 & 0.0 & 1.088 & 50.2 & DA19 \\
    Karmn J17071+215 & 10.557 & 11.677 & 7.042 & -5.197 & 0.0 &  &  &  \\
    Karmn J17303+055 & 8.533 & 9.433 & 5.422 & -4.61 & 0.0 &  &  &  \\
    Karmn J17378+185 & 8.743 & 9.577 & 5.572 & -5.157 & 0.0 &  &  &  \\
    Karmn J17578+046 & 8.194 & 9.511 & 4.524 & -5.359 & 0.0 & 1.882 & 145.0 & TP19 \\
    Karmn J18051-030 & 8.519 & 9.36 & 5.306 & -4.737 & 0.0 & 2.838 & 127.8 & SM15 \\
    Karmn J18319+406 & 10.81 & 12.046 & 7.184 & -5.05 & 0.0 & 0.7 & 50.2 & DA19 \\
    Karmn J18353+457 & 9.107 & 9.83 & 6.082 & -4.737 & 0.0 & 0.9 & 34.0 & DA19 \\
    Karmn J18363+136 & 11.154 & 12.46 & 7.367 & -4.852 & 0.0 & 0.613 & 50.2 & DA19 \\
    Karmn J18409-133 & 9.772 & 10.63 & 6.546 & -4.647 & 0.0 &  &  &  \\
    Karmn J18419+318 & 10.201 & 13.126 & 6.722 & -5.466 & 0.0 &  &  &  \\
    Karmn J18498-238 & 9.126 & 10.43 & 5.37 & -4.435 & 0.0 & 0.035 & 2.85 & CC23 \\
    Karmn J18580+059 & 8.461 & 9.217 & 5.357 & -4.538 & 0.0 & 0.874 & 35.2 & DA19 \\
    Karmn J19070+208 & 9.815 & 10.774 & 6.521 & -5.083 & 0.0 &  &  &  \\
    Karmn J19072+208 & 9.805 & 10.771 & 6.517 & -5.041 & 0.0 & 0.075 & 3.8 & DA19 \\
    Karmn J19169+051N & 8.1 & 9.115 & 4.673 & -4.862 & 0.0 & 0.817 & 46.0 & DA19 \\
    Karmn J19216+208 & 11.934 & 13.42 & 7.935 & -5.511 & 0.0 & 1.296 & 133.0 & DA19 \\
    Karmn J19346+045 & 8.714 & 9.326 & 5.918 & -4.273 & 0.0 & 0.416 & 12.9 & DA19 \\
    Karmn J20405+154 & 11.837 & 13.41 & 7.75 & -5.482 & 0.0 & 0.934 & 106.0 & DA19 \\
    Karmn J20450+444 & 9.829 & 10.704 & 6.533 & -4.881 & 0.0 & 0.413 & 19.9 & DA19 \\
    Karmn J20525-169 & 10.082 & 11.458 & 6.199 & -5.191 & 0.0 & 0.75 & 67.6 & DA19 \\
    Karmn J20567-104 & 10.419 & 11.487 & 6.884 & -4.678 & 0.0 & 0.156 & 9.63 & DA19 \\
    Karmn J21019-063 & 10.183 & 11.276 & 6.692 & -5.024 & 0.0 &  &  &  \\
    Karmn J22020-194 & 10.831 & 12.03 & 7.185 & -5.316 & 0.0 &  &  &  \\
    Karmn J22021+014 & 8.403 & 9.146 & 5.322 & -4.471 & 0.0 & 0.748 & 29.5 & DA19 \\
    Karmn J22096-046 & 9.237 & 10.366 & 5.594 & -5.042 & 0.0 & 0.576 & 39.2 & SM15 \\
    Karmn J22125+085 & 10.908 & 11.982 & 7.472 & -5.387 & 0.0 &  &  &  \\
    Karmn J22137-176 & 12.067 & 13.672 & 8.115 & -4.646 & 0.0 &  &  &  \\
    Karmn J22231-176 & 11.551 & 13.3 & 7.319 & -4.468 & 0.0 & 0.031 & 4.256 & Bou24 \\
    Karmn J22330+093 & 9.542 & 10.373 & 6.356 & -4.853 & 0.0 & 0.273 & 12.038 & Bou24 \\
    Karmn J22468+443 & 9.005 & 10.26 & 5.299 & -4.148 & 0.0 & 0.057 & 4.35 & Rev20 \\
    Karmn J22503-070 & 9.133 & 9.847 & 6.104 & -4.497 & 0.0 &  &  &  \\
    Karmn J22532-142 & 8.875 & 10.192 & 5.01 & -5.168 & 0.0 & 0.94 & 81.0 & DA19 \\
    Karmn J22559+178 & 9.656 & 10.558 & 6.449 & -4.495 & 0.0 & 0.581 & 27.0 & SM18 \\
    Karmn J22565+165 & 7.795 & 8.638 & 4.523 & -4.698 & 0.0 & 0.847 & 39.5 & DA19 \\
    Karmn J23216+172 & 10.359 & 11.71 & 6.507 & -5.074 & 0.0 & 0.856 & 74.7 & DA19 \\
    Karmn J23340+001 & 10.177 & 11.16 & 6.828 & -5.36 & 0.0 &  &  &  \\
    Karmn J23381-162 & 10.343 & 11.288 & 7.003 & -5.559 & 0.0 & 1.2 & 61.7 & Wat06 \\
    Karmn J23492+024 & 8.153 & 8.993 & 5.043 & -5.462 & 0.0 & 1.177 & 49.9 & SM18 \\

\end{longtable}

}

One might expect a decrease in mean rotational period around the magnitude of
the Gap, due to the slight decrease in magnetic activity. However, there is no
statistically significant correlation between rotational period and G
magnitude which we can detect given our sample size (Figure
\ref{fig:rotationalSignifigance}). Rotational period is however, not the ideal
parametrization to use, as magnetic activity is more directly related to the
Rossby number ($Ro$). Using the empirical calibration presented in
\citet{Wright2018} (Equation \ref{eqn:tauc}) we find the mixing timescale for
each star such that the Rossby Number is defined as $Ro = P_{rot}/\tau_{c}$.

\begin{equation}\label{eqn:tauc}
  \tau_{c} = 0.64 + 0.25 * (V-K)
\end{equation}

When we compare Rossby number to G magnitude (Figure \ref{fig:rossby}) we find
that there may be a slight paucity of rotation coincident with the decrease in
spread of the activity metric. We quantify the statistical significance of this
drop by building a Gaussian kernel density estimator (kde) based on the data
outside of this range, and then resampling that kde 10000 times for each data
point in the theorized paucity range. The false alarm probability that that drop
is due to noise is then the product of the fraction of samples which are less
than or equal to the value of each data point. We find that there is a 0.022
percent probability that this dip is due purely to noise.


\begin{figure}
  \centering
  \includegraphics[width=0.85\textwidth]{figures/jaoMagActivity/RotationSignifigance.pdf}
  \caption{Rotational Periods against G magnitude for all stars with rotational
  periods (top). Standard deviation of rotational period within magnitude bin (bottom).The location of the Gap
  as identified in literature is shown by the hatched region ($\sim$ 10-10.5 $M_{G}$).}
  \label{fig:rotationalSignifigance}
\end{figure}

\begin{figure}
  \centering
  \includegraphics[width=0.85\textwidth]{figures/jaoMagActivity/Rossby.pdf}
  \caption{Rossby number vs. G magnitude for all stars with rotational periods
  and V-K colors on Simbad. Dashed lines represent the hypothesized region of decreased rotation.The location of the Gap
  as identified in literature is shown by the hatched region ($\sim$ 10-10.5 $M_{G}$).}
  \label{fig:rossby}
\end{figure}


\subsection{Limitations}
There are two primary limitation of our dataset. First, we only have 264 star
in our dataset (with measured $R'_{HK}$, 187 with rotational periods) limiting
the statistical power of our analysis. This is primarily due to the relative
difficulty of obtaining Ca II H\&K measurements compared to obtaining $H\alpha$
measurements. Reliable measurements require both high spectral resolutions (R
$\sim$ 16000) and a comparatively blue wavelength range \footnote{wrt. to what
many spectrographs cover. There is no unified resource listing currently
commissioned spectrographs; however, it is somewhat hard to source glass which
transmits well at H\&K wavelengths limiting the lower wavelength of most
spectrographs.}.

Additionally, the sample we do have does not extend to as low mass as would be
ideal. This presents a degeneracy between two potential causes for the observed
increased star-to-star variability. One option, as presented above and
elaborated on in the following section, is that this is due to kissing
instabilities. However, another possibility is that this increased variability
is intrinsic to the magnetic fields of fully convective stars. This alternate
option may be further supported by the shape of the magnetic activity spread vs.
G magnitude relation. Convective kissing instabilities are not expected to
continue to much lower masses than the fully convective transition mass. The
fact that the increase in variance which we observe continues to much fainter
magnitudes would therefore be somewhat surprising in a purely convective kissing
instability driven framework (though the degeneracy between potentially
physically driven increase in variance and increase in variance due to the
noise-magnitude relation complicates attempts to constrain this.) There is
limited discussion in the literature of overall magnetic field strength
spanning the fully convective transition mass; however, \citet{Shulyak2019}
present estimated magnetic field strengths for 47 M dwarfs, spanning a larger
area around the convective transition region and their dataset does not
indicate a inherently increased variability for fully convective stars.
