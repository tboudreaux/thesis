\section{The Jao Gap and Magnatism}\label{sec:intro}
The magnetic activity of M dwarfs is of particular interest to many astronomers
due to the theorised links between habitability and the magnetic environment
which a planet resides within \citep[e.g.][]{Lammer2012,Gallet2017,
Kislyakova2017}. M dwarfs are known to be more magnetically active than earlier
type stars \citep{Saar1985,Astudillo-Defru2017,Wright2018} while simultaneously
this same high activity calls into question the canonical magnetic dynamo
believed to drive the magnetic field of solar-like stars (the $\alpha\Omega$
dynamo) \citep{Shulyak2015}. One primary challenge which M dwarfs pose is that
stars less than approximately 0.35 M$_{\odot}$ are composed of a single
convective region. This denies any dynamo model differential rotation between
adjacent levels within the star. Alternative dynamo models have been proposed,
such as the $\alpha^{2}$ dynamo along with modifications to the $\alpha\Omega$
dynamo which may be predictive of M dwarf magnetic fields \citep{Chabrier2006,
Kochukhov2021, Kleeorin2023}.

Despite this work, very few studies have dived specifically into the magnetic
field of M dwarfs at or near the convective transition region . This is not
surprising as that only spans approximately a 0.2 magnitude region
in the Gaia BP-RP color magnitude diagram and is therefore populated by a
relatively small sample of stars. 

\citet{Jao2023} identify the Jao Gap as a strong discontinuity point for
magnetic activity in M dwarfs. Two primary observations from their work are
that the Gap serves as a boundary where very few active stars, in their sample
of 640 M dwarfs, exist below the Gap and that the overall downward trend of
activity moving to fainter magnitudes is anomalously high in within the 0.2 mag
range of the Gap. \citeauthor{Jao2023} Figures 3 and 13 make this paucity in
H$\alpha$ emission particularly clear. Based on previous work from
\citet{Spada2020, Curtis2020, Dungee2022} the authors propose that the
mechanism resulting in the reduced fraction of active stars within the Gap is
that as the radiative zone dissipates due to core expansion, angular momentum
from the outer convective zone is dumped into the core resulting in a faster
spin down than would otherwise be possible. Effectively the core of the star
acts as a sink, reducing the amount of angular momentum which needs to be lost
by magnetic breaking for the outer convective region to reach the same angular
velocity. Given that H$\alpha$ emission is strongly coupled magnetic activity
in the upper chromosphere \citep{Newton2016, Kumar2023} and that a star's angular velocity
is a primary factor in its magnetic activity, a faster spin down will serve to
more quickly dampen H$\alpha$ activity.

In addition to H$\alpha$ the Calcium Fraunhaufer lines may be used to trace the
magnetic activity of a star. These lines originate from magnetic heating of the
lower chromosphere driven by magnetic shear stresses within the star.
Both \citet{Perdelwitz2021} and \citet{Boudreaux2022} present calcium emission
measurements for stars spanning the Jao Gap. In this paper we search for
similar trends in the Ca II H\& K emission as \citeauthor{Jao2023} see in the
H$\alpha$ emission. In Section \ref{sec:results} we investigate the empirical
star-to-star variability in emission and quantify if this could be due to noise
or sample bias; in Section \ref{sec:modeling} we present a simplified toy model
which shows that the mixing events characteristic of convective kissing
instabilities could lead to increased star-to-star variability in activity as
is seen empirically.

% Stellar modeling has been successful in reproducing the Jao Gap
% \citep[e.g.][]{Feiden2021,Mansfield2021} and, with these models, we have begun
% to understand which parameters constrain the Jao Gap's location. For example,
% it is now well documented that metallicity affects the Jao Gap's color, with
% higher metallicity stellar populations showing the Jao Gap at consistently
% higher masses / bluer colors \citep{Mansfield2021}.
%
% {\color{red} EXPAND THIS, READ SOME OTHER Gap PAPERS TO SEE WHAT THEY DO}

% Both \citeauthor{Feiden2021} and \citeauthor{Mansfield2021} demonstrate the Jao
% Gap's location sensitivity to age, evolving to higher mass regions of the
% mass-luminosity relation with population age. Per \citet{Mansfield2021} the
% degree of this location evolution also does not seem to be strongly sensitive
% to metallicity. 
