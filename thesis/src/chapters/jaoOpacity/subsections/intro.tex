\section{The Jao Gap}\label{sec:JOintro}
Due to the initial mass requirements of the molecular clouds which collapse to form
stars, star formation is strongly biased towards lower mass, later spectral
class stars when compared to higher mass stars. Partly as a result of this
bias and partly as a result of their extremely long main-sequence lifetimes,
M Dwarfs make up approximately 70 percent of all stars in the galaxy. Moreover,
some planet search campaigns have focused on M Dwarfs due to the relative ease
of detecting small planets in their habitable zones \citep[e.g.][]{Nut08}.
M Dwarfs then represent both a key component of the galactic stellar population
as well as the possible set of stars which may host habitable exoplanets.
Given this key location M Dwarfs occupy in modern astronomy it is important to
have a thorough understanding of their structure and evolution.

\citet{Jao2018} discovered a novel feature in the Gaia Data Release 2 (DR2)
$G_{BP}-G_{RP}$ color-magnitude-diagram. Around $M_{G}=10$ there is an
approximately 17 percent decrease in stellar density of the sample of stars
\citet{Jao2018} considered. Subsequently, this has become known as either the
Jao Gap, or Gaia M Dwarf Gap. Following the initial detection of the Gap in DR2
the Gap has also potentially been observed in 2MASS \citep{Skrutskie2006,
Jao2018}; however, the significance of this detection is quite weak and it
relies on the prior of the Gap's location from Gaia data. Further, the Gap is
also present in Gaia Early Data Release 3 (EDR3) \citep{Jao2021}. These EDR3
and 2MASS data sets then indicate that this feature is not a bias inherent to
DR2.

The Gap is generally attributed to convective instabilities in the cores of
stars straddling the fully convective transition mass (0.3 - 0.35 M$_{\odot}$)
\citep{Baraffe2018}. These instabilities interrupt the normal, slow, main
sequence luminosity evolution of a star and result in luminosities lower
than expected from the main sequence mass-luminosity relation \citep{Jao2020}.

The Jao Gap, inherently a feature of M Dwarf populations, provides an enticing
and unique view into the interior physics of these stars \citep{Feiden2021}.
This is especially important as, unlike more massive stars, M Dwarf seismology
is unfeasible due to the short periods and extremely small
magnitudes which both radial and low-order low-degree non-radial seismic waves
are predicted to have in such low mass stars \citep{Rodriguez-Lopez2019}. The
Jao Gap therefore provides one of the only current methods to probe the
interior physics of M Dwarfs.

Despite the early success of modeling the Gap some issues remain.
\citet{Jao2020, Jao2021} identify that the Gap has a wedge shape which has not been
successful reproduced by any current modeling efforts and which implies a
somewhat unusual population composition of young, metal-poor stars. Further,
\citet{Jao2020} identify substructure, an additional over density of stars,
directly below the Gap, again a feature not yet fully captured by current
models. 

All currently published models of the Jao Gap make use of OPAL high temperature
radiative opacities. Here we investigate the effect of using the more
up-to-date OPLIB high temperature radiative opacities and whether these opacity
tables bring models more in line with observations. In Section \ref{sec:JaoGap}
we provide an overview of the physics believed to result in the Jao Gap, in
Section \ref{sec:opac} we review the differences between OPAL and OPLIB and
describe how we update DSEP to use OPLIB opacity tables. Section
\ref{sec:modeling} walks through the stellar evolution and population synthesis
modeling we perform. Finally, in Section \ref{sec:results} we present our
findings. 

Stellar modeling has been successful in reproducing the Jao Gap
\citep[e.g.][]{Feiden2021,Mansfield2021} and, with these models, we have begun
to understand which parameters constrain the Jao Gap's location. For example,
it is now well documented that metallicity affects the Jao Gap's color, with
higher metallicity stellar populations showing the Jao Gap at consistently
higher masses / bluer colors \citep{Mansfield2021}.

Both \citeauthor{Feiden2021} and \citeauthor{Mansfield2021} demonstrate the Jao
Gap's location sensitivity to age, evolving to higher mass regions of the
mass-luminosity relation with population age. Per \citet{Mansfield2021} the
degree of this location evolution also does not seem to be strongly sensitive
to metallicity. 
