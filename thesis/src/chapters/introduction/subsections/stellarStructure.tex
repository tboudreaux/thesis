\chapter{Stellar Structure Evolution}
There are currently many stellar structure codes \citep[e.g.][]{Dotter2008,
Kovetz2009, Paxton2011} which integrate the equations of stellar structure ---
in addition to equations of state and lattices of nuclear reaction rates ---
over time to track the evolution of an individual star. The Dartmouth Stellar
Evolution Program (DSEP) \citep{Chaboyer2001, Bjork2006, Dotter2008} is one
such, well tested, stellar evolution program.

DSEP solves the equations of stellar structure using the Henyey method
\citep{Henyey1964}. This is a relaxation technique making use of a
Newton–Raphson root finder and therefore requires some initial guess to relax
towards a solution. This guess will be either some initial, polytropic, model
or the solution from the previous timestep.  In order to evolve a model through
time DSEP alternates between solving for reaction rates and the structure
equations. At some temperature and pressure from the solution to the structure
equations DSEP finds the energy generation rate due to proton-proton chains,
the CNO cycle, and the tripe-alpha process from known nuclear cross sections.
These reaction rates yield both photon and neutrino luminosities as well as
chemical changes over some small time step. Thermodynamic variables are
calculated using an equation of state routine which is dependent on the initial
model mass. All the updated physical quantities (pressure, luminosity, mean
molecular mass, temperature) are then used to solve the structure equations
again. This process of using a solution to the structure equations to calculate
reaction rates which then inform the next structure solution continues until
DSEP can no longer find a solution.  This can happen as the stellar structure
equations are extremely stiff. In addition, for finite radial mesh sizes,
discontinuities can occur.

While other stellar evolution programs, such as the widely used Modules for
Experimentation in Stellar Astrophysics (MESA) \citep{Paxton2011}, consider a
more complex handling of nuclear reaction rate calculations, and are
consequently more applicable to a wider range of spectral classes than DSEP,
DSEP has certain advantages over these other programs that make it well suited
for certain tasks, such as low-mass modeling. For one, DSEP generally can
evolve models much more rapidly than MESA and has a smaller memory footprint
while doing it. This execution time difference is largely due to the fact that
DSEP makes some simplifying assumptions due to its focus only on models with
initial masses between 0.1 and 5 M$_{\odot}$ compared to MESA’s more general
approach.  Moreover, MESA elects to take a very careful handling of numeric
uncertainty, going so far as to guarantee byte-to-byte similarity of the same
model run on different architectures \citep{Paxton2011}. DSEP on the other hand
makes no such guarantee. Rather, models evolved using DSEP will be accurate
down to some arbitrary, user controllable, tolerance but beyond that point may
vary from one computer to another. Despite this trade off in generality and
precision, the current grid of isochrones generated by DSEP \citep{Dotter2008},
has been heavily cited since its initial release in 2008, proving that there is
a place for a code as specific as DSEP.
