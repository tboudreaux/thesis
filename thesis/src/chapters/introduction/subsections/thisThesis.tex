% \chapter{This Thesis}
\section{This Thesis}
Astronomy is somewhat of an outlier in the natural sciences in so far as we do
not have access to laboratory conditions --- at least most of the time, notable
exceptions include experiments such as those run with the Z Pulsed Power
Facility \citep[e.g.][]{Falcon2013}, computational simulations, and certain
cosmochemistry work \citep[e.g.][]{MacPherson2011}. Throughout this thesis we
will discuss five projects which either use low mass, $< 2$ M$_{\odot}$, as
laboratories to test various physics in a controlled manner or present work
which will in future make these stars use as labratories more feasible.

First we will how features present in statistical samples of stars may be used
to infer various parameters of those same statistical samples. This section
includes careful and chemically consistent fitting of isochrones to the
color-magnitude diagram of NGC 2808 to infer accurate helium abundances.
Additionally, this section includes multiple projects focused on the Gaia M
Dwarf, or Jao, Gap. The Jao gap provides a unique view into the interior
physics of fully convective stars.

The second part of this thesis will focus on how properties of individual stars
which may be used to study their physics in a controlled manner. Here we will
discuss both the rotation-activity relation of M Dwarfs and how various
parameters may effect the location of the Red Giant Branch Bump. 
