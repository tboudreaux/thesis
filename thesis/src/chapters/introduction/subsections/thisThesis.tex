\section{This Thesis}
So far I have provided an extremely abridged overview of the history of stellar
physics and more generally our our understanding of stars. In a way these
preceding chapters and sections are the most important part of this work for
myself. As with all doctoral theses the chapters which follow will be very
in-the-weeds, detail oriented, and at times abstract. Despite this, or perhaps
because of it, I feel it important to reaffirm the reasons I have for pursing a
degree, writing this thesis, and doing astronomy research. {\em This thesis is
at its core a work of personal interest}. With that said there seems little
need for me to spend the readers time with further self reflection.

Astronomy is somewhat of an outlier in the natural sciences in so far as we do
not have access to laboratory conditions --- at least most of the time, notable
exceptions include experiments such as those run with the Z Pulsed Power
Facility \citep[e.g.][]{Falcon2013}, computational simulations, and certain
cosmochemistry work \citep[e.g.][]{MacPherson2011}. Throughout this thesis we
will discuss five projects which either use low mass stars, $< 2$ M$_{\odot}$, as
laboratories to test various physics in a controlled manner or present work
which will in future make these stars use as laboratories more feasible.

First we will determine how features present in statistical samples of stars may be used
to infer various parameters of those same statistical samples. This section
includes careful and chemically consistent fitting of isochrones to the
color-magnitude diagram of NGC 2808 to infer accurate helium abundances.
Additionally, this section includes multiple projects focused on the Gaia M
Dwarf, or Jao, Gap. The Jao gap provides a unique view into the interior
physics of fully convective stars.

The second part of this thesis will focus on how properties of individual stars
which may be used to study their physics in a controlled manner. Here we will
discuss both the rotation-activity relation of M Dwarfs and how various
parameters may effect the location of the Red Giant Branch Bump. 

Many of the work presented in this thesis has either already been published 
or is under review for publication (Table \ref{tab:publications}) 

\begin{table}
\centering
\scriptsize
  \begin{tabular}{l p{3cm} c p{2cm} c c}
    \hline
    \rowcolor{gray!50}
    Chapter & Paper Title & DOI & Paper Authors & Date Published & State\\
    \hline
    \hline
    Ch. 3 & Chemically Self-Consistent Modeling of the Globular Cluster NGC 2808 and its Effects on the Inferred Helium abundance of Multiple Stellar Populations & N/A & Boudreaux, Emily M., Chaboyer, Brian C., Ash, Amanda., Edaes Hoh, Renata., Feiden, Gregory.& N/A & {\em Under Review} \\
    \rowcolor{gray!25}
    Ch. 4 & Updated High-temperature Opacities for the Dartmouth Stellar Evolution Program and Their Effect on the Jao Gap Location & 10.3847/1538-4357/acb685 & Boudreaux, Emily M., Chaboyer, Brian C. & February 2023 & {\em Published} \\
    Ch. 5 & Correlations between Ca II H\&K Emission and the Gaia M dwarf Gap & 10.48550/arXiv.2402.14984 & Boudreaux, Emily M., Garcia Soto, Aylin., Chaboyer, Brian C. & February 2024 & {\em In Pres.} \\
    \rowcolor{gray!25}
    Ch. 6 & The Ca II H\&K Rotation-Activity Relation in 53 mid-to-late type M-Dwarfs & 10.3847/1538-4357/ac5cbf & Boudreaux, Emily M., Newton, Elisabeth E., Mondrik, Nicholas., Charbonneau, David., Irwin, Jonathan. & April 2022 & {\em Published}
  \end{tabular}
  \caption{List of published papers and their equivalent chapters in this thesis. Note that chapter 7 has not been submitted for publication and therefore does not appear in this table.}
  \label{tab:publications}
\end{table}
