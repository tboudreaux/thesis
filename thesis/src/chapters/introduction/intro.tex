\chapter{A Breif History of Humans \& Stars}
\begin{quote}\em
  That's the wonderful thing about crayons. They can take you to more places than a starship.
\end{quote}
\begin{flushright}
-- Guinan, Star Trek the Next Generation, Season 6, Episode 7.
\end{flushright}
\section{A post prelude prelude}
I will start by saying that I am not trained in history and perhapse it is
self-indulgent to being with such an ameture overview of another academic
field. However, all of the abstraction which we surruond ourselves with in so
many of the physical sciences may sometimes sever us from underlying
motivations as to why we are interested. It is true that we often speak to the
public and to funding agencies about grand ambition and infinite possibility;
however, our day-to-day lives are not defined by these things. We spend so much
of our time either immersed in the fine details or prostelatizing about greater
purpose that I find that I often feel discconected from the real reason I am
interested in astronomy. I study astronomy because I am interested in space,
there is no greater reason, nothing more substantial than a five year old
sitting with their father outside their preschool asking about why the stars
and moon are the way that they are. With that being said, please bear with me
as I walk us through a breif history of some of the places stars have had in
human civilization and the evolution of our understanding of them.

\section{Pre-Modern Astronomy}
\subsection{Stars In Ancient Times}
The ansestors of humans have no doubt been looking at the sky since before they
were humans; however, there are limited ways to study these prehistoric
astronomers. Some reminatnts of ealy human (pre-history, $\gtrapprox$ 5000 yrs
ago) stargazing to remain, largely in the form of earthn works \addcite. The
earliest records we have show stars playing an imporant role in regligious
practices, navigation, and time keeping \addcite. These early understandings
recognized differences between the fixed stars and the wandring stars though
it would be many more centuries and millenia before the full extent of those
differences became clear \addcite.

As far back as the 1300s B.C.E there were accurate star charts being produced
by ancient Egyptian, Babylonian, and Chinese astronomers. Fourth century B.C.E
Greeks develope star catalougs and many of the moden day English names of stars
still derive from these early catalougs.

The understanding of fixed stars at this time was not one of objects similar to
our sun; rather, they were thought to be seperate things altogether. Various
cosmologies supposed that stars were points on a celestial sphere which
surrounded the earth, such as the Babylonian cosmology with posited that the
stars existed in a heaven of their own which was interlocked with 2 other
heavens.

\subsection{Astronomy Becomes a Science}
The ancient Greeks, Babylonians, and Chinese contributed greatly to our
understanding of the universe; however, it was later work, largely by Islamic
and Indian scholars which form the basis of what we think of as modern
astronomy. These astronomers were not yet conceptualizing the universe the way
that we model it today; however, they began systamatized observations of the
universe in ways which would be recognizable. One of the main driving forces of
astronomy during this period was nagivation and to that end new instrumentation
was developed and early observatories were built. In 934 C.E. \textit{The Book
of Fixed Stars} was written by Abd al-Rahman al-Sufi. A expansion of the kinds
of catalougs which began over a millenia earlier and allowed for more precise
navigation of ships. 

By the 10th Century C.E. Islamic scholars had found evidence that the Earth was not,
contrary to Ptolomeic ideas, stationary. In the following centuries multiple 
critisicms of Ptolemeic theories were published, though none going quite as far
as doubting the geocentric model of the universe. One primary change to these
ancient theories was the regognition that the Earth rotates about its axis 
as opposed to the heavens rotaing around earth \addcite. 

\subsection{The Birth of Modern Astronomy}
Modern Astronomy does not have a single birth, and different cultures at
different points in time have contributed in various ways to our modern
understanding. Often, in western culture we think of 15th and 16th century
astronomers, such as Johannas Kepler and Tycho Brahe as the first astronomers.
No doubt that Brahe infulence pushed the field to systamatize certain
observations more and Kepler's theoretical work was some of the earliest which
presented a reasonable model of orbital motion. However, outside of Europe
astronomers such as Uluga Beg (1394 $\rightarrow$ 1449) and Wang Zehenyi (1768
$\rightarrow$ 1797) we building early observatories, improving star catalougs,
and devloping models for both lunar and solar eclipses.


Despite Astronomy's extrenley long history peoples perceptions of what
stars were had remained relativly fixed over time. There had been some early
astronomers (notably from Aristotle) who proposed that stars might be far away
suns; however, this did not translate to mainstream of global acceptance. This began
to change during the early modern period and it became more common to view stars
as far away stars potentially with their own planetary systems. Even during this
period though, what a star is, what the sun is, physically, was not understood.

\section{Stars in the Modern World}
With the advent of spectroscopy and more formalized and globalized astronomy
during the 18th and 19th centuries a physically driven picture of stars emerged.
Spectroscopy revealed similar trends between our sun and other stars and new
physics and chemistry allowed astronomers to start pieceing together that stars
were made, primarily, of extremley hot hydrogen and helium \addcite.

By the early 20th century astronomers were close to figuring out what energy
powered stars. The most prominant theory being heat generated from
gravitational collapse (which had innitially been propsed as a theory of star
formation, but not energy generation by Kant and Laplace \addcite). Lord Kelvin
and Hermann von Helmholtz proposed that a stars energy could originate from
gravitational collapse. They derived that the sun could power itself for 19
million years \addcite (Aside \ref{gravCollapse}). However, as
geological records settled on an age in the billions of years for earth it
became clear that some other mechanism must provide energy to a star.


\begin{sidebar}{Kelvin-Helmholtz Lifespan}{gravCollapse}
  The total binding energy of a star may be approximated as
  \begin{equation*}
    U = -\frac{3}{5}\frac{GM^{2}}{R} \approx -2.3 \times 10^{41} \text{J}
  \end{equation*}
  From the flux of the sun and its radius we can find that its approximate
  energy output is $4\times 10^{26}$ Watts. Then assuming all of that
  energy is being produced by gravaitational collapse the total 
  time the sun could burn would be given by the ratio of the total
  energy to the luminosity.
  \begin{equation*}
    t = \frac{|U|}{L} \approx 19 \times 10^{6} \text{yr}
  \end{equation*}
\end{sidebar}

Arthur Eddington, often considered one of the founders of stellar astrophyics,
concerned with discrepency between geological records of Earths age an the suns maximum life
span, as predicted by the Kelvin-Helmholtz mechanism, proposed in the 1920s
that a stars energy might primarily originate from the fusion of hydrogen into
helium \addcite. Early estimates of the maxium lifespan of the sun if it were
powered by hydrogen fusion prooved to be sufficient for the Earths lifespan to
make sense. In the following decades the sun and other stars would become
important test beds for physics too extreme to be studied in labratory conditions
on earth, a theme similar to one we will return to multiple times during
this thesis.

While the mechanism which powers the sun was still being debated mathematical
models which would evolve into those still used today were being developed.
Over the last half of the 19th and first decade of the 20th centuries Lane,
Ritter, and Emden codified the earliest mathematical model of stellar
structure, the polytrope (Equation \ref{eqn:polytrope}), in \textit{Gaskugeln}
(Gas Balls) \citep{Emden1907}.

\begin{align}\label{eqn:polytrope}
	\frac{d}{d\xi}\left(\xi^{2}\frac{d\theta}{d\xi}\right) = -\xi^{2}\theta^{n}
\end{align}

Where $\xi$ and $\theta$ are dimensionless parameterizations of radius and
temperature respectively, and $n$ is known as the polytropic index. Despite
this early work, it wasn't until the late 1930s and early 1940s that the full
set of equations needed to describe the structure of a steady state,
radially-symmetric, star (known as the equations of stellar structure) began to
take shape as the specific fusion chanies (primarily the proton-proton chains
and the Carbon-Nitrogen-Oxygen cycle) were, seriously considered as energy
generation mechanisms \citep{Cowling1966}. Since then, and especially with the
proliferation of computers in astronomy, the equations of stellar structure
have proven themselves an incredibly predictive set of models.  

Today stars and stellar physics form the basis of much of astronomy \addcite.
Despite being some of the smallest objects studied by astronomers stars provide
the majority of the luminsosity of the universe \addcite. They make up key
rungs on the distance ladder \addcite, and they provide key constraints on
topics as varied as fundamental nuclear physics \addcite and the search for
habital planets and life \addcite. Stars then represent not just an interesting
object of study in their own right but also, in the right conditions, labratories
where we can study astronomical objects in a controlled manner.

Astronomy is somewhat of an outlier in the nautral sciences in so far as we do
not have access to labratory conditions --- at least most of the time, notable
exceptions include experiments such as those run with the Z-machine,
computational simulations, and certain cosmochemistry work. Throughout this
thesis we will disscus five projects which use low mass, $< 2$ M$_{\odot}$, as
labratories to test various physics in a controlled manner. First we will how
features present in statisticall samples of stars may be used to measure
population ages 

\chapter{Stellar Structure Evolution}
There are currently many stellar structure codes \citep[e.g.][]{Dotter2008,
Kovetz2009, Paxton2011} which integrate the equations of stellar structure ---
in addition to equations of state and lattices of nuclear reaction rates ---
over time to track the evolution of an individual star. The Dartmouth Stellar
Evolution Program (DSEP) \citep{Chaboyer2001, Bjork2006, Dotter2008} is one
such, well tested, stellar evolution program.

DSEP solves the equations of stellar structure using the Henyey method
\citep{Henyey1964}. This is a relaxation technique making use of a
Newton–Raphson root finder and therefore requires some initial guess to relax
towards a solution. This guess will be either some initial, polytropic, model
or the solution from the previous timestep.  In order to evolve a model through
time DSEP alternates between solving for reaction rates and the structure
equations. At some temperature and pressure from the solution to the structure
equations DSEP finds the energy generation rate due to proton-proton chains,
the CNO cycle, and the tripe-alpha process from known nuclear cross sections.
These reaction rates yield both photon and neutrino luminosities as well as
chemical changes over some small time step. Thermodynamic variables are
calculated using an equation of state routine which is dependent on the initial
model mass. All the updated physical quantities (pressure, luminosity, mean
molecular mass, temperature) are then used to solve the structure equations
again. This process of using a solution to the structure equations to calculate
reaction rates which then inform the next structure solution continues until
DSEP can no longer find a solution.  This can happen as the stellar structure
equations are extremely stiff. In addition, for finite radial mesh sizes,
discontinuities can occur.

While other stellar evolution programs, such as the widely used Modules for
Experimentation in Stellar Astrophysics (MESA) \citep{Paxton2011}, consider a
more complex handling of nuclear reaction rate calculations, and are
consequently more applicable to a wider range of spectral classes than DSEP,
DSEP has certain advantages over these other programs that make it well suited
for certain tasks, such as low-mass modeling. For one, DSEP generally can
evolve models much more rapidly than MESA and has a smaller memory footprint
while doing it. This execution time difference is largely due to the fact that
DSEP makes some simplifying assumptions due to its focus only on models with
initial masses between 0.1 and 5 M$_{\odot}$ compared to MESA’s more general
approach.  Moreover, MESA elects to take a very careful handling of numeric
uncertainty, going so far as to guarantee byte-to-byte similarity of the same
model run on different architectures \citep{Paxton2011}. DSEP on the other hand
makes no such guarantee. Rather, models evolved using DSEP will be accurate
down to some arbitrary, user controllable, tolerance but beyond that point may
vary from one computer to another. Despite this trade off in generality and
precision, the current grid of isochrones generated by DSEP \citep{Dotter2008},
has been heavily cited since its initial release in 2008, proving that there is
a place for a code as specific as DSEP.


\chapter{Stars as Stellar Labratories}

\begin{itemize}
  \item Overview of what makes stars so important to human history
  \item Overview of what makes stars so important to the history of the universe 
  \item General overview of the history of stellar stucture
    \begin{itemize}
      \item History of thought on stars
      \item First attempts to describe what stars were
      \item Early modern attempts to describe stars (i.e. gravity collapse)
      \item Start of modern stellar astrophysics
      \item history of stellar structure equations
      \item history of computer models
      \item history and provinence of DSEP
    \end{itemize}
  \item Why is it important to understand stellar physics going forward
  \item Switch gears to talk about GC because of how they are
    \begin{itemize}
      \item The roll GC play in understanding the universe
      \item labs to test stellar physics!
    \end{itemize}
  \item Talk about Jao Gap
    \begin{itemize}
      \item Talk about the Jao Gap as a stellar labratory (interiors of low mass stars)
    \end{itemize}
  \item Talk about the RGBB, again a lab for stellar physics
  \item breif overview of what will be presented in the thesis
\end{itemize}
