Low mass stars account for approximately 70 percent of the stellar populations
\citep{Conroy2012}; yet, due to their small sizes and cool temperatures they
account for only a small fraction of the galaxies luminosity function
\citep{Laughlin1997}. Moreover, due to the lack of labratory conditions
avalible to astronomy and astrophysics low mass stars can provide a rare
controlled enviroment for calibrations of numerical models. Consequently,
across multiple domains there has been signifigant interest in these key
astronomical objects. In this thesis I present three projects which have
further revealed properties of low mass stars and pushed the extent where
these low mass stars may be used as labratories. Firstly, I present chemically
self consistent stellar evolutionary models of the globular clusters NGC 2808.
Due to the age of this cluster, these models are dominated by low mass stars.
We find that chemical consistency between a stars structural and atmospheric
models makes only a trivial difference in model predictions. Secondly, I
present a detailed investigation into the Gaia M Dwarf Gap (the Jao Gap)
looking at how the Jao Gap's theoretical location is effected by high
temperature radiative opacity source and how the physics which drives the Jao
Gap's formation may also drive perturbations to stellar magnetic field
strength. A detailed understanding of the Jao Gap's underlying physics may
provide an important calibration point for M dwarf convectivive parameters. The
work presented in this thesis brings the field of astronomy closer to being
able to use those calibrations. Finally, this thesis investigates the relation
between the red giant branch bump (RGBB) in both NGC 2808 accross multiple
populations and across multiple opacity sources. Similar to the Jao Gap, the
RGBB provides a calibration point for convective parameters in stars on the red
giant branch. We find that the helium enriched population in NGC 2808 does not
show a detectable RGBB, validating previous theoretical studies of the RGBB
which did not consider multiple populations in their modeling. 

