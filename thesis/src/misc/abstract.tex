Low mass stars account for approximately 70 percent of the stellar populations; yet, due to their small sizes and cool temperatures they account for only {\color{red} NUMBER} percent of the galaxies luminosity function. Consequently, across multiple domains there has been a dearth of interest in these key astronomical objects. In this thesis I present two projects which have further revealed properties of low mass stars. Firstly, I present chemically self consistent models of the globular clusters NGC 2808 and NGC 6752. Due to the ages of both clusters, these models are dominated by low mass stars. We find that chemical consistency between a stars structural and atmospheric models makes only a trivial difference in model predictions. Secondly, I present a detailed investigation into the Gaia M Dwarf Gap (the Jao Gap) looking at how the Jao Gap's theoretical location is effected by high temperature radiative opacity source and how the physics which drives the Jao Gap's formation may also drive perturbations to stellar magnetic field strength.
