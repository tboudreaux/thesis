When I was in pre-school I told my father that I would be a professor of
astronomy. I remember the day, walking out of the Boyd school, having just
looked through a large picture book of the --- then --- nine planets in our
solar system. At the time I did not have a concept of what it meant to be an
astronomer. I did not understand what it meant to study space in any way other
than looking at images in a large cardboard book. In a very real way that day
was the most important in my life and it undoubtedly set a trajectory for me
which I have obsessively held onto for the past 22 years. I have held onto
that goal; however, my ability to, in some small part with this thesis, achieve
the goal that that 4 year old told their father has not primarily been a
function of my ability. Rather, many many people in my life have supported me,
helped me, loved me, let me lean on them, and been there for me when I need
them. I would not have been able to write this thesis without each and every
one of them and I am immeasurably grateful for the people in my life.

First of all I would like to speak to my mother and father, Karol and Don. You
are the most wonderful parents I could ever imagine. From the time when your
young child made the wild statement that they wanted to be an astronomer,
through many detentions and late nights at Westminster, through far too many
hours at Encore and comically long commutes in high school you have both been
the most supportive parents anyone ever could be. You have relentlessly
supported and loved me in ways that, I hope, have made me a better person, 
a better researcher, and a better child. I love you so much and I am so
grateful. Thank you.

Secondly, I would like to thank the educators, teachers, and professors who
I have interacted with in my life. I have been blessed and privileged to
have such a consistently good set of educational role models in my life. From
Mr. David Majewski in seventh and eighth grade science, to Ms. Leslie Zeigler 
in high school biology. My childhood teachers shaped my love of science and
helped me develop that goal that four year old me had into a firm sense of
what astronomy is. I want to thank Ms. Zeigler in particular for helping me 
grow my love of science and supporting that during a time in life when so 
many people get disillusioned with it. I truly believe that if not for her
I would not have remained interested in astronomy through high-school.

When I left high school for college I was worried that I would not be able to
build the same kind of relationships with educators there than I had had
until that point. I could not have been more wrong. I first met Dr. Brad Barlow
the day before his wedding, in October 2014, and from the moment we met he went
above and beyond any possible expectation to try to help me become an astronomer.
Brad brought me observing at SOAR when I was still a senior in High School, he
took me to AAS my first year in college, he co authored 3 papers with me and
took me around the world to various conferences. The friendship I developed with
Brad was the most important, by far, of my time in college and I immensely
grateful for that. More than the professional development opportunities which
Brad provided to me I want to thank him for his consistent willingness to
engage with me on random thoughts I had. I think chatting outside the Slane
center over lunch about various research ideas or topics I had just learned
about in class is what began, in earnest, to develop my ability to think about
science analytically and critically. That is the greatest gift that a professor
could ever give to their student. Thank you Brad.

Outside of education I have also been exceptionally privileged in the friends
whom I have met. I spent much of my childhood quite isolated, by choice.
However, working at Encore with Sarah, Annika, Kara, and Paddy was one of the
most important experiences of my life. Whereas Leslie and Brad helped shape who
I am as a scientist, my friends served to shape who I am as a person. There can
never be words strong enough to thank you all for that. I love you all very
deeply. 

Much as I was worried about my ability to form mentor-mentee connections in
college I was worried about my ability to make friends in grad school. This
concern could not have been more misplaced. I think there is likely no one in
grad school with a better set of friends and colleagues. My joint cohort mates
and house mates Aylin and Weishi are wonderful. We have lived together for 5
years now and I cannot imagine better people to live with. They are funny, fun,
engaging, and wonderful. I am so glad we will be able to keep living together
and I am so glad that we have. Aside from those I live with Steph and Rayna
have been incredible friends. Steph is always there for support or laughs, or
cooking and Rayna is always there for a good laugh or a sandwich. Thank you
all so much for being such wonderful friends and I cannot wait to see what you
accomplish with your careers.

Keighley Rockcliffe is one of the most amazing people I have ever met. I
remember being so scared of her when I first starting attending group meetings
my first year. She is a brilliant scientist, an incredibly empathetic and kind
person, and one of my best friends in the world. I could not have finished grad
school without her and I want to thank her from the bottom of my heart for
always being there for me. During grad school I went through a lot of change in
my life, I transitioned, I adopted a dog, I lost a dog, I struggled with
depression, and I struggled with anxiety. For every single struggle Keighley has
been there to provide a shoulder to cry on, a ear to listen, and an arm to
support. Thank you Keighley, you are a wonderful person, I love you so much and
I cannot wait to see the impact you make on the world.

Towards the end of my time in graduate school I was lucky enough to meet another
one of the most amazing people. I met Isabel in August of 2023 and even though
we have only known each other for 8 months, they have been some of the most
wonderful 8 months of my life. You have made the stress of the end of grad
school so much more bearable and you have made the stress of transitioning so
much more bearable. Shortly after we met I lost Jordy unexpectedly and
suddenly. That was likely the hardest day of my life and Isabel was there for
me in a way that is so much more than could ever have been expected. I could
not have continued in grad school after that without Isabel. I love you Isabel
and I am so excited for the coming years together.

I adopted Jordy in February of 2021 and she was the best Beagle, the best dog, and
the best companion someone could hope for. Jordy was the littlest beagle and the
stinkiest girl. She brought so much love and happiness to everyone life. I 
had planned to walk at graduation with her but unfortunately in October of 2023 
Jordy passed away from cancer. I miss you Jordy and I love you little girl.

Finally, this work was conducted under the supervision of Brian Chaboyer and I
would not have been able to complete this thesis without his continued support.
I would like to that Brian immensely for his continued support and for being a
wonderful teacher. I have not always been the fastest worker, spending far too
long getting distracted with small side projects. However, Brian has always
been both supportive and a provided a guiding hand to get those projects back
on track and turned into publishable material. Thank you Brian.
